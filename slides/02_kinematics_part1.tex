\documentclass{beamer}

% Shared Beamer configuration for Principles of Robot Autonomy decks
\usetheme{Boadilla}
%\usecolortheme{seagull}

% Use Palatino to match the book's Tufte style
\usepackage{mathpazo}
\usefonttheme{serif}

\usepackage{amsmath}
\usepackage{amssymb}
\usepackage{bm}
\usepackage{graphicx}
\usepackage{enumitem}
\usepackage{xparse}
\usepackage{array}

% Increase vertical spacing between list items for readability.
\setlist[itemize]{itemsep=0.75em}
\setlist[enumerate]{itemsep=0.75em}

% Define bold symbol macros if not already defined
\providecommand{\x}{\boldsymbol{x}}
\providecommand{\z}{\boldsymbol{z}}
\providecommand{\q}{\boldsymbol{q}}
\providecommand{\bxi}{\boldsymbol{\xi}}
\providecommand{\bu}{\boldsymbol{u}}
\providecommand{\bmu}{\boldsymbol{\mu}}
\providecommand{\bSigma}{\boldsymbol{\Sigma}}
\providecommand{\y}{\boldsymbol{y}}
\providecommand{\m}{\mathcal{M}}
\providecommand{\bc}{\boldsymbol{c}}
\providecommand{\R}{\mathbb{R}}

% Animated lists by default, with override via optional overlay argument.
\beamerdefaultoverlayspecification{<+->}

% Paragraphs reveal sequentially by default.
\AtBeginDocument{%
  \ifx\paragraph\undefined
    \newcommand{\paragraph}[1]{\textbf{#1}\par\onslide<+->{} }
  \else
    \let\origparagraph\paragraph
    \renewcommand{\paragraph}[1]{\origparagraph{#1}\par\onslide<+->{} }
  \fi
}

% Add other shared packages or theme settings above this line as needed.

\author[COMP 4766]{Adapted from ``Principles of Robot Autonomy'' by D. Gammelli, J. Lorenzetti,
K. Luo, G. Zardini, and M. Pavone}
\date{\today}


\title{Mobile Robot Kinematics: Part 1}

\begin{document}

\begin{frame}[plain]
\titlepage
\end{frame}

\begin{frame}{Source Material}
    Covering the following sections from ``Principles of Robot Autonomy'':
    \begin{itemize}
    \item 1.1:
    \begin{itemize}
        \item 1.1.1 covered lightly (ignoring linearity and time-invariance)
        \item 1.1.2 covering only Example 1.1.3
    \end{itemize}
%    \item 1.2:
%    \begin{itemize}
%        \item 1.2.1 - 1.2.4 covered fully
%        \item 1.2.5 not covered
%    \end{itemize}
    \end{itemize}
\end{frame}


\begin{frame}{Motion Planning Needs Models}
\begin{itemize}
\item Motion planning and control are fundamental components of robotic autonomy.
\item But to do either, we need some model of robot motion.
\item We start by discussing \emph{state} and \emph{state-space models}.
\item We will then look at the distinction between \emph{dynamics} and \emph{kinematics}.
\item Many robots are subject to \emph{kinematic constraints} which are crucial to understand.
\end{itemize}
\end{frame}


\begin{frame}{State Space Models}
    A state space model is a framework to describe a system (e.g. a moving robot) evolving over time.
    \begin{block}{State}
    The state of a system at time $t_0$ is a minimal
set of variables $\mathbf{x}(t_0)$ such that, given the control input $\mathbf{u}(t)$ for all $t \geq t_0$, the
future evolution of the system's state $\mathbf{x}(t)$ for all $t \geq t_0$ is uniquely determined.
    \end{block}
    \pause

    \begin{itemize}
    \item Both $\mathbf{x}(t)$ and $\mathbf{u}(t)$ are vectors. $\mathbf{x}(t)$ is the \emph{state vector} which is $n-$dimensional. 
    
    \item The state for a mobile robot will often be $\mathbf{x}(t) = [x, y, \theta]$.
    \pause

    \item $(x, y)$ refers to the robot's $x-$ and $y-$ coordinates relative to some coordinate frame, and $\theta$ is the robot's rotation relative to that frame.

    \item Note: $\mathbf{x}$ is a vector, whereas $x$ is a scalar.
    \end{itemize}
\end{frame}


\begin{frame}
    What variables we consider part of the state is dependent both on the scenario and the level of modelling.
    \pause

    \vspace{1em}
    \begin{tabular*}{\textwidth}{@{\extracolsep{\fill}}p{0.22\textwidth}p{0.22\textwidth}p{0.22\textwidth}p{0.22\textwidth}}
        \raggedright Mobile robot & \raggedright ...with linear ($v$) and angular ($\omega$) speeds & \raggedright ...with landmarks & \raggedright\arraybackslash Arm with 6 joint angles \\
        \noalign{\smallskip}
            $\mathbf{x} = \begin{bmatrix} 
                x \\ 
                y \\ 
                \theta
            \end{bmatrix} 
            $
& 
            $\mathbf{x} = \begin{bmatrix} 
                x \\ 
                y \\ 
                \theta \\ 
                v \\ 
                \omega 
            \end{bmatrix}
            $
& 
            $\mathbf{x} = \begin{bmatrix} 
                x \\ 
                y \\ 
                \theta \\ 
                l_{1,x} \\ 
                l_{1,y} \\ 
                l_{2,x} \\ 
                l_{2,y} \\ 
                \vdots 
            \end{bmatrix}
            $
& 
            $\mathbf{x} = \begin{bmatrix} 
                q_1 \\ 
                q_2 \\ 
                \vdots \\ 
                q_6 \\ 
                \dot{q}_1 \\ 
                \dot{q}_2 \\ 
                \vdots \\ 
                \dot{q}_6 
            \end{bmatrix} 
            $
    \end{tabular*}
\end{frame}


\begin{frame}
\begin{itemize}
\item $\mathbf{u}(t)$ is considered the system's input.
\item The inputs could be forces, torques, voltages, commands, or whatever we have direct control over.
\item $\mathbf{u}(t)$ is $m-$dimensional.
\item The system may also have some output $\mathbf{y}(t)$ which typically comes from sensor data (e.g. GPS or camera pixels).
\item $\mathbf{y}(t)$ is $q-$dimensional.
\item The process of inferring the state from the system's output is known as \emph{state estimation}.
\end{itemize}
\end{frame}


\begin{frame}
\begin{block}{Model}
    A state-space model describes the evolution of a system through \emph{state equations} and \emph{observation equations}.
\end{block}
    \pause\vspace{0.5cm}

    State equations are differential equations of the following form:
    \pause

    $$\dot{\mathbf{x}} = f(\mathbf{x}(t), \mathbf{u}(t)),$$
    \pause
    where $\dot{\mathbf{x}} = \frac{dx(t)}{dt}$.
    \pause

    \vspace{0.5cm}
    Observations equations have this form:

    $$\mathbf{y}(t) = h(\mathbf{x}(t), \mathbf{u}(t))$$
\end{frame}


\begin{frame}{Linear Models}
    A state-space model which is \emph{linear} can be expressed the following form:
    \pause
        \begin{align*}
            \dot{\mathbf{x}}(t) &= A(t)\mathbf{x}(t) + B(t)\mathbf{u}(t) \\
            \mathbf{y}(t) &= C(t)\mathbf{x}(t) + D(t)\mathbf{u}(t)
        \end{align*}
        \pause
        \begin{itemize}
            \item $A(t)$ is the dynamics matrix ($n \times n$).
            \item $B(t)$ is the control matrix ($n \times m$).
            \item $C(t)$ is the output/sensor matrix ($q \times n$).
            \item $D(t)$ is the direct matrix ($q \times m$).
        \end{itemize}
\end{frame}


\begin{frame}
    If $A(t)$, $B(t)$, $C(t)$ and $D(t)$ are actually time-invariant, we can express the model more compactly,
        \begin{align*}
            \dot{\mathbf{x}}(t) &= A\mathbf{x}(t) + B\mathbf{u}(t) \\
            \mathbf{y}(t) &= C\mathbf{x}(t) + D\mathbf{u}(t).
        \end{align*}
    We will often leave off the ``$(t)$'' even when it applies (e.g. $\mathbf{x}$ is always going to be a function of time, otherwise we don't need the model). 
        \begin{align*}
            \dot{\mathbf{x}} &= A\mathbf{x} + B\mathbf{u} \\
            \mathbf{y} &= C\mathbf{x} + D\mathbf{u}.
        \end{align*}
\end{frame}


\begin{frame}{Example: Double-Integrator}
    Consider a mass in space (i.e. no friction, no gravity). We can push the mass in a fixed direction, through an input force $u$.
    \pause
    \begin{center}
        \includegraphics[width=0.4\linewidth]{figs/ch01/mass_in_space.pdf}
    \end{center}

    $s$ is the one-dimensional position of the mass.
    \pause\vspace{0.5cm}

    The system is governed by Newton's Second Law of Motion:
    $$u = m \ddot{s}$$

    \pause
    To recover $s$ we would need to integrate $\ddot{s}$ twice, hence this is referred to as a double-integrator system.
\end{frame}


\begin{frame}
    To describe this second-order system, we define two state variables:
    \vspace{0.3cm}
    \begin{itemize}
        \item $x_1 = s$ (position)
        \item $x_2 = \dot{s}$ (speed)
    \end{itemize}
    \pause\vspace{0.3cm}
    The derivatives of our states are:
    \begin{itemize}
        \item $\dot{x}_1 = \dot{s} = x_2$
        \item $\dot{x}_2 = \ddot{s} = \frac{1}{m}u$
    \end{itemize}
    \pause\vspace{0.3cm}
    We can now write this in matrix form.
    \vspace{0.5cm}
    $$\begin{bmatrix} \dot{x}_1 \\ \dot{x}_2 \end{bmatrix} = \begin{bmatrix} 0 & 1 \\ 0 & 0 \end{bmatrix} \begin{bmatrix} x_1 \\ x_2 \end{bmatrix} + \begin{bmatrix} 0 \\ \frac{1}{m} \end{bmatrix} u$$
    \pause\vspace{0.3cm}
    This is the state equation.
\end{frame}


\begin{frame}
    What about the observation equation? Actually, we can't obtain this without some definition for the output, $\mathbf{y}$.
    \pause\vspace{0.3cm}

    Lets define our output $y$ (a scalar) as the position $s$:
    $$y = \begin{bmatrix} 1 & 0 \end{bmatrix} \begin{bmatrix} x_1 \\ x_2 \end{bmatrix} + [0] u$$

    We now have the complete state-space model:
    \pause\vspace{0.3cm}
    \begin{itemize}
        \item Dynamics Matrix: $\mathbf{A} = \begin{bmatrix} 0 & 1 \\ 0 & 0 \end{bmatrix}$
        \item Control Matrix: $\mathbf{B} = \begin{bmatrix} 0 \\ \frac{1}{m} \end{bmatrix}$
        \item Output Matrix: $\mathbf{C} = \begin{bmatrix} 1 & 0 \end{bmatrix}$
        \item Direct Matrix: $\mathbf{D} = [0]$
    \end{itemize}
\end{frame}

\end{document}
