\documentclass{beamer}

% Shared Beamer configuration for Principles of Robot Autonomy decks
\usetheme{Boadilla}
%\usecolortheme{seagull}

% Use Palatino to match the book's Tufte style
\usepackage{mathpazo}
\usefonttheme{serif}

\usepackage{amsmath}
\usepackage{amssymb}
\usepackage{bm}
\usepackage{graphicx}
\usepackage{enumitem}
\usepackage{xparse}
\usepackage{array}

% Increase vertical spacing between list items for readability.
\setlist[itemize]{itemsep=0.75em}
\setlist[enumerate]{itemsep=0.75em}

% Define bold symbol macros if not already defined
\providecommand{\x}{\boldsymbol{x}}
\providecommand{\z}{\boldsymbol{z}}
\providecommand{\q}{\boldsymbol{q}}
\providecommand{\bxi}{\boldsymbol{\xi}}
\providecommand{\bu}{\boldsymbol{u}}
\providecommand{\bmu}{\boldsymbol{\mu}}
\providecommand{\bSigma}{\boldsymbol{\Sigma}}
\providecommand{\y}{\boldsymbol{y}}
\providecommand{\m}{\mathcal{M}}
\providecommand{\bc}{\boldsymbol{c}}
\providecommand{\R}{\mathbb{R}}

% Animated lists by default, with override via optional overlay argument.
\beamerdefaultoverlayspecification{<+->}

% Paragraphs reveal sequentially by default.
\AtBeginDocument{%
  \ifx\paragraph\undefined
    \newcommand{\paragraph}[1]{\textbf{#1}\par\onslide<+->{} }
  \else
    \let\origparagraph\paragraph
    \renewcommand{\paragraph}[1]{\origparagraph{#1}\par\onslide<+->{} }
  \fi
}

% Add other shared packages or theme settings above this line as needed.

\author[COMP 4766]{Adapted from ``Principles of Robot Autonomy'' by D. Gammelli, J. Lorenzetti,
K. Luo, G. Zardini, and M. Pavone}
\date{\today}


\title{Chapter 12 - Modern Computer Vision}

\begin{document}

\frame{\titlepage}

\begin{frame}{Deep Learning Era}
\begin{itemize}
\item Modern perception stacks lean on CNNs trained with abundant data and compute.
\item Architectural staples: convolutions, nonlinearities, pooling, then dense heads.
\item Robotics uses CNNs for both semantic understanding and geometric cues.
\end{itemize}
\end{frame}
% --- Original text start ---
% Chapter lead-in on CNN motivation and component list.
% --- Original text end ---

\begin{frame}{Canonical CNN Blueprint}
\begin{itemize}
\item Early CNNs (e.g., LeNet-5) already stacked conv, pooling, and FC layers.
\item Depth and parameter sharing regularize training versus plain MLPs.
\item Modern variants scale channel counts but keep the same motif.
\end{itemize}
\pause
\begin{center}
    \includegraphics[width=0.85\linewidth]{../tex/figs/ch12_figs/lecun.png}
\end{center}
\end{frame}
% --- Original text start ---
% Figure of LeCun et al. network and regularization discussion.
% --- Original text end ---

\begin{frame}{Convolution Filters}
\begin{itemize}
\item Learned filters slide over RGB patches with stride and optional padding.
\item Each filter implements $f(x)=w^\top x + b$ before ReLU or similar activation.
\item Filter sizes, strides, and count are key hyperparameters.
\end{itemize}
\pause
\begin{center}
    \includegraphics[width=0.75\linewidth]{../tex/figs/ch12_figs/convolution.png}
\end{center}
\end{frame}
% --- Original text start ---
% Convolution layer explanation and Figure convfilter.
% --- Original text end ---

\begin{frame}{Activation Maps}
\begin{itemize}
\item Filtering plus nonlinearity produces feature maps highlighting learned cues.
\item Multiple filters per layer stack activation maps along the channel axis.
\item Later layers consume these maps to detect higher-level patterns.
\end{itemize}
\pause
\begin{center}
    \includegraphics[width=0.6\linewidth]{../tex/figs/ch12_figs/convolution2.png}
\end{center}
\end{frame}
% --- Original text start ---
% Activation map description and Figure convolution2.
% --- Original text end ---

\begin{frame}{Feature Hierarchies}
\begin{itemize}
\item Early filters resemble edge detectors; mid layers capture textures.
\item Deep filters respond to object parts and semantic motifs.
\item Hierarchy emerges automatically during training.
\end{itemize}
\pause
\begin{center}
    \includegraphics[width=0.7\linewidth]{../tex/figs/ch12_figs/features.png}
\end{center}
\end{frame}
% --- Original text start ---
% Zeiler and Fergus visualization discussion.
% --- Original text end ---

\begin{frame}{Why Convolutions Work}
\begin{itemize}
\item Parameter sharing keeps models compact versus dense layers.
\item Sparse interactions focus computation on local, meaningful evidence.
\item Translation equivariance lets filters fire anywhere in the image.
\item Flexible input sizing comes for free when using shared kernels.
\end{itemize}
\end{frame}
% --- Original text start ---
% Bullet list on benefits (sharing, sparsity, equivariance, varying size).
% --- Original text end ---

\begin{frame}{Pooling Layers}
\begin{itemize}
\item Max or mean pooling summarize activations over small windows.
\item Downsampling increases invariance to small shifts and cuts compute.
\item Pooling preserves channel count while shrinking spatial grids.
\end{itemize}
\pause
\begin{center}
    \includegraphics[width=0.55\linewidth]{../tex/figs/ch12_figs/maxpool.png}
\end{center}
\end{frame}
% --- Original text start ---
% Pooling explanation and Figure maxpool.
% --- Original text end ---

\begin{frame}{Fully Connected Heads}
\begin{itemize}
\item Flattened feature maps feed a small stack of dense layers.
\item These heads perform the final classification or regression task.
\item Only a fraction of layers are dense; most parameters live in conv blocks.
\end{itemize}
\end{frame}
% --- Original text start ---
% Fully connected section.
% --- Original text end ---

\begin{frame}{End-to-End Learning}
\begin{itemize}
\item CNNs jointly learn feature extraction and decision layers.
\item Deep architectures surpassed hand-engineered pipelines post-2012.
\item Classical features still help when structure is well understood.
\end{itemize}
\end{frame}
% --- Original text start ---
% Performance discussion on end-to-end vs classical.
% --- Original text end ---

\begin{frame}{Object Localization}
\begin{itemize}
\item Predict bounding boxes $(x,y,w,h)$ plus class labels per object.
\item Split heads: one for classification, one for regression with $l_2$ loss.
\item Combine via multi-task objective to share conv backbones.
\end{itemize}
\pause
\begin{center}
    \includegraphics[width=0.6\linewidth]{../tex/figs/ch12_figs/boundingbox.png}
\end{center}
\end{frame}
% --- Original text start ---
% Bounding box explanation and Figure boundingbox.
% --- Original text end ---

\begin{frame}{Multiple Objects}
\begin{itemize}
\item Variable counts of objects make fixed-size outputs tricky.
\item Slide CNNs over candidate crops; include a "background" class.
\item Region proposals or grid partitions reduce redundant evaluations.
\end{itemize}
\end{frame}
% --- Original text start ---
% Discussion on regions of interest and sliding CNN filters.
% --- Original text end ---

\begin{frame}{Reminder: CNN Toolbox}
\begingroup
\setbeamercolor{block title}{bg=gray!20,fg=black}
\setbeamercolor{block body}{bg=gray!10,fg=black}
\begin{block}{Key Takeaways}
\begin{itemize}
\item Stack conv + pooling blocks to harvest spatially local cues efficiently.
\item Dense heads plus multi-task losses cover class labels and box regression.
\item Use proposals or grids to scale detection when many objects appear.
\end{itemize}
\end{block}
\endgroup
\end{frame}
% --- Original text start ---
% Summary tying together convolutional building blocks and detection usage.
% --- Original text end ---

\end{document}
