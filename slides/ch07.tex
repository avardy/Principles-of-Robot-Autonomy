\documentclass{beamer}

% Shared Beamer configuration for Principles of Robot Autonomy decks
\usetheme{Boadilla}
%\usecolortheme{seagull}

% Use Palatino to match the book's Tufte style
\usepackage{mathpazo}
\usefonttheme{serif}

\usepackage{amsmath}
\usepackage{amssymb}
\usepackage{bm}
\usepackage{graphicx}
\usepackage{enumitem}
\usepackage{xparse}
\usepackage{array}

% Increase vertical spacing between list items for readability.
\setlist[itemize]{itemsep=0.75em}
\setlist[enumerate]{itemsep=0.75em}

% Define bold symbol macros if not already defined
\providecommand{\x}{\boldsymbol{x}}
\providecommand{\z}{\boldsymbol{z}}
\providecommand{\q}{\boldsymbol{q}}
\providecommand{\bxi}{\boldsymbol{\xi}}
\providecommand{\bu}{\boldsymbol{u}}
\providecommand{\bmu}{\boldsymbol{\mu}}
\providecommand{\bSigma}{\boldsymbol{\Sigma}}
\providecommand{\y}{\boldsymbol{y}}
\providecommand{\m}{\mathcal{M}}
\providecommand{\bc}{\boldsymbol{c}}
\providecommand{\R}{\mathbb{R}}

% Animated lists by default, with override via optional overlay argument.
\beamerdefaultoverlayspecification{<+->}

% Paragraphs reveal sequentially by default.
\AtBeginDocument{%
  \ifx\paragraph\undefined
    \newcommand{\paragraph}[1]{\textbf{#1}\par\onslide<+->{} }
  \else
    \let\origparagraph\paragraph
    \renewcommand{\paragraph}[1]{\origparagraph{#1}\par\onslide<+->{} }
  \fi
}

% Add other shared packages or theme settings above this line as needed.

\author[COMP 4766]{Adapted from ``Principles of Robot Autonomy'' by D. Gammelli, J. Lorenzetti,
K. Luo, G. Zardini, and M. Pavone}
\date{\today}


\title{Chapter 07 - Robot Sensors}

\begin{document}

\begin{frame}[plain]
\titlepage
\end{frame}

\begin{frame}{Perception in Autonomy}
\begin{itemize}
\item Autonomy couples perception, planning, and control; sensors power the ``see'' loop.
\item Robots blend diverse exteroceptive and proprioceptive inputs to stay aware of both world and self.
\item Upcoming chapters dive into sensing hardware plus the pipelines that interpret measurements.
\end{itemize}
\end{frame}
% --- Original text start ---
% Intro paragraphs on see-think-act and sensor role across perception chapters.
% --- Original text end ---

\begin{frame}{Sensor Taxonomy}
\begin{itemize}
\item Proprioceptive sensors observe internal robot states (motors, joints, voltages).
\item Exteroceptive sensors probe the environment (ranges, light intensity, sound levels).
\item Interpretation burden is typically higher for exteroceptive data because semantics must be inferred.
\end{itemize}
\end{frame}
% --- Original text start ---
% Definitions of proprioceptive and exteroceptive sensing.
% --- Original text end ---

\begin{frame}{Active vs Passive Behavior}
\begin{itemize}
\item Passive sensors rely on ambient energy (thermometers, cameras) and inherit environmental variability.
\item Active sensors emit probe energy (ultrasonic, lidar) and must model round-trip propagation.
\item Classification matters because illumination control, interference, and safety constraints differ.
\end{itemize}
\end{frame}
% --- Original text start ---
% Definitions of passive and active sensors plus environmental dependencies.
% --- Original text end ---

\begin{frame}{Design-Spec Metrics}
\begin{itemize}
\item Dynamic range compares upper/lower input limits, often via $\text{DR}=10\log_{10}(r)$ in dB.
\item Resolution sets the minimum distinguishable increment; digital sensors can decouple it from dynamic range.
\item Linearity states whether output scales proportionally with input across the operating range.
\item Bandwidth (Hz) caps how fast fresh measurements arrive, constraining robot speed and feedback rates.
\end{itemize}
\end{frame}
% --- Original text start ---
% Design specification metrics list (dynamic range, resolution, linearity, bandwidth).
% --- Original text end ---

\begin{frame}{In Situ Performance}
\begin{itemize}
\item Sensitivity trades signal amplification against noise amplification; cross-sensitivity captures off-axis effects.
\item Error $e = m - v$ compares measured value $m$ to truth $v$; accuracy $a = 1 - |m-v|/v$ summarizes agreement.
\item Precision tracks repeatability; a sensor can be precise yet inaccurate if bias dominates.
\end{itemize}
\end{frame}
% --- Original text start ---
% In situ metrics (sensitivity, cross-sensitivity, error, accuracy, precision).
% --- Original text end ---

\begin{frame}{Systematic vs Random Errors}
\begin{itemize}
\item Systematic errors stem from deterministic, modelable causes (calibration drift, temperature bias).
\item Random errors arise from stochastic influences (specular sonar reflections, camera black-level noise).
\item Practical robots face blended regimes where environment-dependent effects blur these categories.
\end{itemize}
\end{frame}
% --- Original text start ---
% Distinction between systematic and random sensor errors.
% --- Original text end ---

\begin{frame}{Reminder: Error Analysis Mindset}
\begingroup
\setbeamercolor{block title}{bg=gray!20,fg=black}
\setbeamercolor{block body}{bg=gray!10,fg=black}
\begin{block}{Concept Recap}
\begin{itemize}
\item Catalog systematic sources first, then model random terms (often Gaussian) only where justified.
\item Propagate both contributions through downstream estimators to predict accuracy and precision.
\item Revisit assumptions as environments change; ``random'' sonar failures may become systematic at fixed poses.
\end{itemize}
\end{block}
\endgroup
\end{frame}
% --- Original text start ---
% Discussion on error analysis challenges and environment-dependent behavior.
% --- Original text end ---

\begin{frame}{Modeling Uncertainty}
\begin{itemize}
\item Probability distributions approximate unresolved random errors; unimodal Gaussian assumptions ease math.
\item Real sensors (e.g., sonar, stereo) often exhibit multi-modal, biased noise when failure modes activate.
\item Always align the assumed distribution with observed operating regimes to avoid brittle estimators.
\end{itemize}
\end{frame}
% --- Original text start ---
% Section on modeling uncertainty and pitfalls of unimodal assumptions.
% --- Original text end ---

\begin{frame}{Encoder Essentials}
\begin{itemize}
\item Encoders convert mechanical rotation into digital pulses for wheel or joint sensing.
\item Optical quadrature designs illuminate slotted discs and read phase-shifted waveforms.
\item Bandwidth comfortably exceeds mobile-robot shaft speeds, enabling precise proprioception.
\end{itemize}
\pause
\begin{center}
    \includegraphics[width=0.65\linewidth]{../book/figs/ch07_figs/wheel_encoder.png}
\end{center}
\end{frame}
% --- Original text start ---
% Encoder description and Figure \ref{fig:encoder}.
% --- Original text end ---

\begin{frame}{Encoder Resolution Notes}
\begin{itemize}
\item Resolution (cycles per revolution) dictates minimum angular increment; $2000$ CPR $\rightarrow 0.18^\circ$ steps.
\item Quadrature sensing multiplies counts by four by tracking both channel edges.
\item Controlled mounting and predictable environments keep encoder systematic errors negligible.
\end{itemize}
\end{frame}
% --- Original text start ---
% Discussion of CPR, quadrature, and practical accuracy assumptions.
% --- Original text end ---

\begin{frame}{Heading Sensors Overview}
\begin{itemize}
\item Compasses read Earth’s magnetic field (Hall or flux-gate) but suffer from disturbances and indoor biases.
\item Gyroscopes preserve orientation in an inertial frame and complement compasses for dead reckoning.
\item Blending these modalities via filtering mitigates individual weaknesses.
\end{itemize}
\end{frame}
% --- Original text start ---
% Heading sensor section comparing compasses and gyros.
% --- Original text end ---

\begin{frame}{Mechanical Gyroscopes}
\begin{itemize}
\item Fast-spinning rotors on gimbals resist reorientation; friction-induced torques limit long-term accuracy.
\item Reactive torque follows $\tau = I \omega \Omega$, tying drift to inertia and spin rates.
\item High-grade units can drift $0.1^\circ$ over 6 hours but remain costly relative to MEMS.
\end{itemize}
\pause
\begin{center}
    \includegraphics[width=0.6\linewidth]{../book/figs/ch07_figs/gyro.png}
\end{center}
\end{frame}
% --- Original text start ---
% Mechanical gyro description, equation, and Figure \ref{fig:gyro}.
% --- Original text end ---

\begin{frame}{Optical Gyros and Compasses}
\begin{itemize}
\item Ring-laser gyros send counter-propagating beams; frequency split satisfies $\delta f \propto \text{angular velocity}$.
\item Resolutions below $10^{-4}$ deg/hr and $>100$ kHz bandwidth support high-end navigation.
\item Digital compasses remain attractive for cost-sensitive outdoor robots despite vibration sensitivity.
\end{itemize}
\end{frame}
% --- Original text start ---
% Optical gyro paragraph plus compass trade-offs.
% --- Original text end ---

\begin{frame}{Accelerometer Modeling}
\begin{itemize}
\item Spring-mass-damper analogy obeys $F_{\text{applied}} = m \ddot{x} + c \dot{x} + kx$.
\item Steady deflection yields $a_{\text{applied}} = kx/m$ once transients decay.
\item MEMS devices sense proof-mass displacement via capacitive or piezoelectric transducers.
\end{itemize}
\end{frame}
% --- Original text start ---
% Accelerometer equations and MEMS description.
% --- Original text end ---

\begin{frame}{IMU Processing Pipeline}
\begin{itemize}
\item Combine tri-axial gyros and accelerometers to recover orientation, velocity, and position.
\item Rotate measured accelerations into the navigation frame, subtract gravity, then integrate.
\item Initialization at rest provides zero-velocity priors before double integration.
\end{itemize}
\pause
\begin{center}
    \includegraphics[width=0.65\linewidth]{../book/figs/ch07_figs/imu.png}
\end{center}
\end{frame}
% --- Original text start ---
% IMU block diagram discussion and Figure \ref{fig:IMU}.
% --- Original text end ---

\begin{frame}{IMU Drift Considerations}
\begin{itemize}
\item Gyro drift corrupts estimated attitude, which then contaminates gravity removal.
\item Acceleration biases integrate into velocity drift and ultimately large position errors.
\item Practical systems fuse IMUs with external references (GNSS, vision) to reset accumulated error.
\end{itemize}
\end{frame}
% --- Original text start ---
% Discussion of drift propagation and need for aiding sensors.
% --- Original text end ---

\begin{frame}{Beacons and GNSS}
\begin{itemize}
\item Known-position beacons (GNSS, motion-capture) provide absolute pose via relative measurements.
\item Standard GNSS needs four satellites to solve $(x,y,z)$ plus receiver clock bias.
\item Differential methods refine accuracy when reference infrastructure is available.
\end{itemize}
\end{frame}
% --- Original text start ---
% Beacon subsection describing GNSS positioning and enhancements.
% --- Original text end ---

\begin{frame}{Active Ranging Landscape}
\begin{itemize}
\item Time-of-flight sensors measure signal transit time (ultrasonic, lidar, ToF cameras).
\item Geometric systems infer range via optical triangulation or structured light patterns.
\item Direct distances feed both mapping and localization pipelines.
\end{itemize}
\pause
\begin{center}
    \includegraphics[width=0.58\linewidth]{../book/figs/ch07_figs/Velodyne_64e.png}
\end{center}
\end{frame}
% --- Original text start ---
% Active ranging overview and Figure \ref{fig:active_range}.
% --- Original text end ---

\begin{frame}{Time-of-Flight Details}
\begin{itemize}
\item Distance obeys $d = c t$ with $c$ set by propagation medium (sound vs electromagnetic).
\item Laser rangefinders demand sub-nanosecond timing, while ultrasonics relax electronics but limit update rate.
\item Key uncertainty sources: arrival-time detection, electronics jitter, beam dispersion, target reflectivity, medium variation, and relative motion.
\end{itemize}
\end{frame}
% --- Original text start ---
% Time-of-flight derivation and list of quality factors.
% --- Original text end ---

\begin{frame}{Geometric Active Ranging}
\begin{itemize}
\item Optical triangulation projects a collimated beam and senses reflection location on a linear detector.
\item Structured-light systems emit known patterns (points, lines, textures) and triangulate per-pixel depth.
\item Accuracy hinges on calibration of projector/receiver baselines and robust feature decoding.
\end{itemize}
\end{frame}
% --- Original text start ---
% Geometric active ranging explanation.
% --- Original text end ---

\begin{frame}{Other Sensing Modalities}
\begin{itemize}
\item Radar exploits Doppler shifts for relative velocity; tactile arrays support physical interaction.
\item Vision sensors capture rich appearance cues but need heavy computation for semantics.
\item Sensor portfolios should reflect task demands, cost, and environmental challenges.
\end{itemize}
\end{frame}
% --- Original text start ---
% Brief mention of radar, tactile, and vision sensors.
% --- Original text end ---

\begin{frame}{Why Vision?}
\begin{itemize}
\item Cameras deliver massive information throughput for dynamic-scene understanding.
\item Advances in computer vision and image processing unlock tasks from feature tracking to recognition.
\item Remaining challenge: distill raw pixels into geometric and semantic primitives for planning.
\end{itemize}
\end{frame}
% --- Original text start ---
% Motivation for computer vision and mention of applications.
% --- Original text end ---

\begin{frame}{Digital Camera Hardware}
\begin{itemize}
\item CCD sensors use light-sensitive capacitors; CMOS integrates per-pixel amplifiers for low power.
\item Exposure/shutter speed governs integration time before readout.
\item Sensor choice balances image quality, cost, and power across robotic platforms.
\end{itemize}
\end{frame}
% --- Original text start ---
% Digital camera description (CCD vs CMOS, exposure).
% --- Original text end ---

\begin{frame}{Image Formation Basics}
\begin{itemize}
\item Objects scatter multi-wavelength light; raw rays would blur without an aperture barrier.
\item Pinhole setups admit a sparse subset of rays, yielding sharp but dim images.
\item Surface properties (reflection, absorption) dictate perceived color and brightness.
\end{itemize}
\pause
\begin{center}
    \includegraphics[width=0.78\linewidth]{../book/figs/ch07_figs/pinhole.png}
\end{center}
\end{frame}
% --- Original text start ---
% Image formation discussion and Figure \ref{fig:LightRays}.
% --- Original text end ---

\begin{frame}{Pinhole Camera Geometry}
\begin{itemize}
\item Camera frame origin $O$ sits one focal length $f$ in front of the image plane.
\item Rays through $O$ intersect the plane at point $p$, producing inverted imagery.
\item Coordinate relation: $x = f X/Z$, $y = f Y/Z$ for scene point $P=(X,Y,Z)$.
\end{itemize}
\pause
\begin{center}
    \includegraphics[width=0.7\linewidth]{../book/figs/ch07_figs/pinholecamera.png}
\end{center}
\end{frame}
% --- Original text start ---
% Pinhole camera model explanation and Figure \ref{fig:PinholeMath}.
% --- Original text end ---

\begin{frame}{Thin Lens Insights}
\begin{itemize}
\item Lenses replace tiny apertures, refracting rays to focus more light while retaining sharpness.
\item Similar-triangle geometry yields the thin-lens equation $\tfrac{1}{z} + \tfrac{1}{Z} = \tfrac{1}{f}$.
\item Depth-of-field captures tolerable focus ranges; focusing at infinity recovers the pinhole limit.
\end{itemize}
\pause
\begin{center}
    \includegraphics[width=0.68\linewidth]{../book/figs/ch07_figs/thinlens.png}
\end{center}
\end{frame}
% --- Original text start ---
% Thin lens model discussion and Figure \ref{fig:Lens}.
% --- Original text end ---

\begin{frame}{Next Steps}
\begin{itemize}
\item Audit your robot's sensor stack for coverage across proprioception, exteroception, and reference beacons.
\item Capture calibration datasets to characterize both systematic biases and random noise regimes.
\item Preview Chapter 08 to deepen camera modeling and calibration techniques.
\end{itemize}
\end{frame}
% --- Original text start ---
% Closing remarks transitioning toward camera models in later sections.
% --- Original text end ---

\end{document}
