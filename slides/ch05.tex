\documentclass{beamer}

% Shared Beamer configuration for Principles of Robot Autonomy decks
\usetheme{Boadilla}
%\usecolortheme{seagull}

% Use Palatino to match the book's Tufte style
\usepackage{mathpazo}
\usefonttheme{serif}

\usepackage{amsmath}
\usepackage{amssymb}
\usepackage{bm}
\usepackage{graphicx}
\usepackage{enumitem}
\usepackage{xparse}
\usepackage{array}

% Increase vertical spacing between list items for readability.
\setlist[itemize]{itemsep=0.75em}
\setlist[enumerate]{itemsep=0.75em}

% Define bold symbol macros if not already defined
\providecommand{\x}{\boldsymbol{x}}
\providecommand{\z}{\boldsymbol{z}}
\providecommand{\q}{\boldsymbol{q}}
\providecommand{\bxi}{\boldsymbol{\xi}}
\providecommand{\bu}{\boldsymbol{u}}
\providecommand{\bmu}{\boldsymbol{\mu}}
\providecommand{\bSigma}{\boldsymbol{\Sigma}}
\providecommand{\y}{\boldsymbol{y}}
\providecommand{\m}{\mathcal{M}}
\providecommand{\bc}{\boldsymbol{c}}
\providecommand{\R}{\mathbb{R}}

% Animated lists by default, with override via optional overlay argument.
\beamerdefaultoverlayspecification{<+->}

% Paragraphs reveal sequentially by default.
\AtBeginDocument{%
  \ifx\paragraph\undefined
    \newcommand{\paragraph}[1]{\textbf{#1}\par\onslide<+->{} }
  \else
    \let\origparagraph\paragraph
    \renewcommand{\paragraph}[1]{\origparagraph{#1}\par\onslide<+->{} }
  \fi
}

% Add other shared packages or theme settings above this line as needed.

\author[COMP 4766]{Adapted from ``Principles of Robot Autonomy'' by D. Gammelli, J. Lorenzetti,
K. Luo, G. Zardini, and M. Pavone}
\date{\today}


\title{Chapter 05 - Search-Based Motion Planning}

\begin{document}

\frame{\titlepage}

\begin{frame}{From Optimal Control to Search}
\begin{itemize}
\item Prior chapters planned in state space using dynamics-aware optimal control or flatness.
\item Here we move to configuration-space planners that emphasize collision-free navigation (e.g., LaValle 2006).
\item These planners supply high-level paths that later controllers can track.
\end{itemize}
\end{frame}
% --- Original text start ---
% Previous chapters addressed ... new set of algorithms ... configuration space rather than state space.
% --- Original text end ---

\begin{frame}{Motion Planning Problem Recap}
\begin{itemize}
\item Goal: find actions that move from start to goal while obeying constraints, maybe optimizing cost.
\item Previously we parameterized trajectories by state $\x$ and control $\bu$ linked via dynamics.
\item That viewpoint made sense for low-level control but complicates obstacle reasoning.
\end{itemize}
\pause
\[
\dot{\x} = f(\x, \bm{u})
\]
\end{frame}
% --- Original text start ---
% Recall the general definition ... trajectory parameterized by state ... differential equations.
% --- Original text end ---

\begin{frame}{Configurations vs States}
\begin{itemize}
\item Configuration $\bm{q}$ captures the robot's geometric degrees of freedom and derives from $\x$.
\item Choosing $C$-space tailored to collision checking can ignore dynamics during planning.
\item Geometric paths in $C$-space can later be time-parameterized and tracked by lower-level controllers.
\end{itemize}
\end{frame}
% --- Original text start ---
% Configuration $\q$ derivable from state ... useful for geometric path planning.
% --- Original text end ---

\begin{frame}{Motivating Example: L-Shaped Robot}
\begin{itemize}
\item The robot lives in a 2D workspace with obstacles and needs a collision-free path.
\item $\bm{q} = [x, y, \theta]^\top$ suffices to capture translation and heading.
\item Planning directly in $C$-space lets us ignore velocities until we execute the path.
\end{itemize}
\pause
\begin{center}
    \includegraphics[width=0.8\linewidth]{../tex/figs/ch05_figs/2d_ws_obstacles.png}
\end{center}
\end{frame}
% --- Original text start ---
% Example describing L-shaped robot in 2D workspace with obstacles and configuration path planning.
% --- Original text end ---

\begin{frame}{Manifold Structure Matters}
\begin{itemize}
\item Orientation lives on $\mathcal{S}^1$, so headings separated by $2\pi k$ are equivalent.
\item Recognizing the manifold lets the planner choose clockwise or counter-clockwise rotations.
\item Without this, certain goals would appear unreachable due to artificial discontinuities.
\end{itemize}
\pause
\begin{center}
    \includegraphics[width=0.8\linewidth]{../tex/figs/ch05_figs/S1_obstacle.png}
\end{center}
\end{frame}
% --- Original text start ---
% Discussion of $\mathcal{S}^1$ manifold and Figure \ref{fig:rotational-dof-fig}.
% --- Original text end ---

\begin{frame}{Planner Families}
\begin{itemize}
\item Grid-based planners discretize $C$-space into finitely many cells and search the resulting graph.
\item Combinatorial planners keep the space continuous and construct roadmaps inside $C_{free}$.
\item Both ultimately reduce to graph search but differ in how vertices are chosen.
\end{itemize}
\end{frame}
% --- Original text start ---
% Chapter introduces grid-based methods and combinatorial planners.
% --- Original text end ---

\begin{frame}{Grid-Based Discretization}
\begin{itemize}
\item Divide $C$-space into a grid, marking forbidden cells that intersect obstacles.
\item Assume the robot can move between adjacent free cells, abstracting away dynamics.
\item Resolution controls fidelity: too coarse and narrow passages disappear; too fine and graphs explode.
\end{itemize}
\pause
\begin{center}
    \includegraphics[width=0.65\linewidth]{../tex/figs/ch05_figs/2d_ws_grid.png}
\end{center}
\end{frame}
% --- Original text start ---
% Description of grid discretization and Figure \ref{fig:grid}.
% --- Original text end ---

\begin{frame}{Graphs from Grids}
\begin{itemize}
\item Each free cell becomes a vertex $v \in V$; adjacency induces edges $E$.
\item Planning is now a discrete graph search from $q_I$ to $q_G$.
\item Graph size grows exponentially with dimension, so grids suit low-DOF robots best.
\end{itemize}
\end{frame}
% --- Original text start ---
% Mapping grids to graphs and discussing pros/cons of discretization.
% --- Original text end ---

\begin{frame}{Label-Correcting Recipes}
\begin{itemize}
\item Maintain cost-of-arrival $C(q)$ for each vertex and track a frontier set $Q$ of candidates.
\item Iteratively pop a vertex, relax each outgoing edge, and push improved neighbors.
\item Terminate when $Q$ empties; recovered parents trace an optimal path if one exists.
\end{itemize}
\end{frame}
% --- Original text start ---
% Description of label correcting algorithms and finite termination.
% --- Original text end ---

\begin{frame}{Cost-of-Arrival Accounting}
\begin{itemize}
\item Initialize $C(q_I)=0$, $C(q)=\infty$ otherwise to encode lack of knowledge.
\item Relaxing edge $(q,q')$ proposes a candidate arrival cost.
\item Accept it only if it improves $q'$ and remains below the best-known goal cost.
\end{itemize}
\pause
\[
\tilde{C}(q') = C(q) + C(q,q')
\]
\end{frame}
% --- Original text start ---
% Definition of cost-of-arrival and candidate updates.
% --- Original text end ---

\begin{frame}{Search Orders}
\begin{itemize}
\item Depth-first: treat $Q$ like a stack (last-in/first-out) to dive along one branch at a time.
\item Breadth-first: treat $Q$ like a queue to expand all nodes by uniform depth.
\item Best-first: prioritize the smallest $C(q)$, i.e., Dijkstra's optimistic strategy.
\end{itemize}
\pause
\begin{center}
    \includegraphics[width=0.42\linewidth]{../tex/figs/ch05_figs/depth_first.PNG}\hspace{1em}
    \includegraphics[width=0.42\linewidth]{../tex/figs/ch05_figs/breadth_first.PNG}
\end{center}
\end{frame}
% --- Original text start ---
% Paragraphs on depth-first, breadth-first, and best-first/Dijkstra.
% --- Original text end ---

\begin{frame}{Reminder: Priority Queues}
\begingroup
\setbeamercolor{block title}{bg=gray!20,fg=black}
\setbeamercolor{block body}{bg=gray!10,fg=black}
\begin{block}{Concept Recap}
\begin{itemize}
\item A priority queue stores frontier vertices with retrieval keyed by a scalar priority.
\item Implementations (binary heaps, pairing heaps) make extract-min and decrease-key nearly logarithmic.
\item Efficient queues are essential for best-first and A* searches on large grids.
\end{itemize}
\end{block}
\endgroup
\end{frame}
% --- Original text start ---
% Frontier set $Q$ often treated as a priority queue in label-correcting algorithms.
% --- Original text end ---

\begin{frame}{A* Intuition}
\begin{itemize}
\item Augment Dijkstra's greedy cost with a heuristic cost-to-go $h(q)$ toward $q_G$.
\item Heuristic must underestimate the true cost-to-go to preserve optimality.
\item Total priority blends what we have paid with what we expect to pay next.
\end{itemize}
\pause
\[
y(q) = C(q) + h(q)
f(q) = C(q) + h(q)
\]
\end{frame}
% --- Original text start ---
% Definition of cost-to-go, heuristics, and A* modifications.
% --- Original text end ---

\begin{frame}{A* Workflow}
\begin{itemize}
\item Initialize $C(q_I)=0$, $f(q_I)=h(q_I)$, and push $q_I$ to $Q$.
\item Repeatedly pop the vertex with minimal $f(q)$, expand neighbors, and update parents.
\item Stop once $q_G$ is popped; if $h=0$, this collapses to Dijkstra.
\end{itemize}
\end{frame}
% --- Original text start ---
% Algorithm \ref{alg:astar} procedure and note about $h=0$ reducing to Dijkstra.
% --- Original text end ---

\begin{frame}{From Obstacles to Free Space}
\begin{itemize}
\item Combinatorial planners explicitly construct the free subset $C_{free} \subset C$.
\item The robot's footprint is grown onto obstacles so the point robot can move in $C$-space.
\item Once $C_{free}$ is known we can drop explicit geometry and work purely in configuration space.
\end{itemize}
\pause
\begin{center}
    \includegraphics[width=0.58\linewidth]{../tex/figs/ch05_figs/obs_padding.png}
\end{center}
\end{frame}
% --- Original text start ---
% Introduction of $C_{free}$ and Figure \ref{fig:collision-free-space-fig}.
% --- Original text end ---

\begin{frame}{Roadmaps in $C_{free}$}
\begin{itemize}
\item Build a roadmap graph whose vertices and edges lie entirely inside $C_{free}$.
\item Flexibility: vertices can be placed anywhere, avoiding the resolution limits of grids.
\item Once $q_I$ and $q_G$ connect to the roadmap, standard graph search yields a path.
\end{itemize}
\pause
\begin{center}
    \includegraphics[width=0.62\linewidth]{../tex/figs/ch05_figs/2d_ws_Cfree.png}
\end{center}
\end{frame}
% --- Original text start ---
% Description of roadmaps, set S, and Figure \ref{fig:combinatorial-planning}.
% --- Original text end ---

\begin{frame}{Cell Decomposition}
\begin{itemize}
\item Partition $C_{free}$ into easy-to-traverse cells (ideally convex) and connect adjacent ones.
\item Vertical cell decomposition in 2D yields trapezoids/segments amenable to roadmap placement.
\item Applicable to polygonal obstacles; higher-dimensional decompositions grow complex quickly.
\end{itemize}
\pause
\begin{center}
    \includegraphics[width=0.7\linewidth]{../tex/figs/ch05_figs/2d_cell_decomp.png}
\end{center}
\end{frame}
% --- Original text start ---
% Example \ref{ex:celldecomp} detailing vertical cell decomposition and figure.
% --- Original text end ---

\begin{frame}{Other Roadmap Strategies}
\begin{itemize}
\item Maximum-clearance (Voronoi) roadmaps bias paths toward the medial axis to stay far from obstacles.
\item Minimum-distance roadmaps hug obstacle boundaries, which is rarely desirable without smoothing.
\item Choice depends on safety margins, sensing noise, and follow-on controller capabilities.
\end{itemize}
\end{frame}
% --- Original text start ---
% Mention of maximum clearance vs minimum distance roadmaps.
% --- Original text end ---

\begin{frame}{Exercises}
\begin{itemize}
\item Implement the A* grid-based motion planning assignment.
\item Practice configuring heuristics, grids, and parent tracing in simple 2D worlds.
\item Repository: \url{https://github.com/PrinciplesofRobotAutonomy/AA274A_HW2}
\end{itemize}
\end{frame}
% --- Original text start ---
% Exercise description referencing Problem 1: A* Motion Planning.
% --- Original text end ---

\end{document}
