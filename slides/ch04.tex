\documentclass{beamer}

% Shared Beamer configuration for Principles of Robot Autonomy decks
\usetheme{Boadilla}
%\usecolortheme{seagull}

% Use Palatino to match the book's Tufte style
\usepackage{mathpazo}
\usefonttheme{serif}

\usepackage{amsmath}
\usepackage{amssymb}
\usepackage{bm}
\usepackage{graphicx}
\usepackage{enumitem}
\usepackage{xparse}
\usepackage{array}

% Increase vertical spacing between list items for readability.
\setlist[itemize]{itemsep=0.75em}
\setlist[enumerate]{itemsep=0.75em}

% Define bold symbol macros if not already defined
\providecommand{\x}{\boldsymbol{x}}
\providecommand{\z}{\boldsymbol{z}}
\providecommand{\q}{\boldsymbol{q}}
\providecommand{\bxi}{\boldsymbol{\xi}}
\providecommand{\bu}{\boldsymbol{u}}
\providecommand{\bmu}{\boldsymbol{\mu}}
\providecommand{\bSigma}{\boldsymbol{\Sigma}}
\providecommand{\y}{\boldsymbol{y}}
\providecommand{\m}{\mathcal{M}}
\providecommand{\bc}{\boldsymbol{c}}
\providecommand{\R}{\mathbb{R}}

% Animated lists by default, with override via optional overlay argument.
\beamerdefaultoverlayspecification{<+->}

% Paragraphs reveal sequentially by default.
\AtBeginDocument{%
  \ifx\paragraph\undefined
    \newcommand{\paragraph}[1]{\textbf{#1}\par\onslide<+->{} }
  \else
    \let\origparagraph\paragraph
    \renewcommand{\paragraph}[1]{\origparagraph{#1}\par\onslide<+->{} }
  \fi
}

% Add other shared packages or theme settings above this line as needed.

\author[COMP 4766]{Adapted from ``Principles of Robot Autonomy'' by D. Gammelli, J. Lorenzetti,
K. Luo, G. Zardini, and M. Pavone}
\date{\today}


\title{Chapter 04 - Optimal Control and Trajectory Optimization}

\begin{document}

\frame{\titlepage}

\begin{frame}{Why Revisit Optimal Control?}
\begin{itemize}
\item Chapter 4 returns to optimal control to connect planning, control, and numerical solvers.
\item The goal is to contrast indirect and direct methods and expose how discretization choices matter.
\item We will also study a worked double-integrator example and point to practice problems.
\end{itemize}
\end{frame}
% --- Original text start ---
% Previously, the idea of using \textit{optimal control} ... This chapter, the optimal control problem is revisited in more detail, including a brief discussion on the use of both \textit{indirect} and \textit{direct} methods.
% --- Original text end ---

\begin{frame}{Continuous-Time Optimal Control Problem}
\begin{itemize}
\item An OCP minimizes terminal and running costs while respecting nonlinear dynamics and initial data.
\item State $\x(t) \in \mathbb{R}^n$ and control $\bm{u}(t) \in \mathbb{R}^m$ define the trajectory to be optimized.
\item Solving the problem yields an open-loop policy $\bm{u}^*(t)$ parameterized by time and the initial condition.
\end{itemize}
\pause
\[
\begin{aligned}
\min_{\bm{u},\x}\; & h(\x(t_f),t_f) + \int_{t_0}^{t_f} g(\x(t),\bm{u}(t),t)\,dt \\
\text{s.t.}\; & \dot{\x}(t) = a(\x(t),\bm{u}(t),t), \quad \x(t_0) = \x_0
\end{aligned}
\]
\end{frame}
% --- Original text start ---
% Consider an optimal control problem (OCP) ... equations (1) ... \bm{u}^*(t) = f(\x(t_0), t).
% --- Original text end ---

\begin{frame}{Why The Problem Is Hard}
\begin{itemize}
\item OCPs are infinite-dimensional because they optimize over continuous functions of time.
\item Numerical solutions require discretization, but the stage where discretization enters defines the method family.
\item Understanding the structure helps decide when a controller can remain open-loop versus needs feedback augmentation.
\end{itemize}
\end{frame}
% --- Original text start ---
% Unfortunately, this optimization problem is particularly challenging to solve since it is \textit{infinite-dimensional} ... Methods for solving (1) can be categorized as either \textit{indirect} or \textit{direct}.
% --- Original text end ---

\begin{frame}{Indirect vs Direct Perspectives}
\begin{itemize}
\item Indirect: derive necessary optimality conditions ("optimize first"), then discretize those conditions for computation.
\item Direct: discretize the original problem immediately ("discretize first"), producing a nonlinear program (NLP).
\item Both paths inevitably use numerical solvers, but their modeling burdens differ.
\end{itemize}
\end{frame}
% --- Original text start ---
% Indirect methods follow a "first optimize, then discretize" approach ... Direct methods follow a "first discretize, then optimize" approach.
% --- Original text end ---

\begin{frame}{Refresher: Equality-Constrained Optimization}
\begin{itemize}
\item Before attacking infinite-dimensional problems, recall finite-dimensional Lagrange multiplier conditions.
\item The constraints enter through multipliers $\bm{\lambda}$ that augment the cost into a Lagrangian $L(\x,\bm{\lambda})$.
\item Stationarity in both $\x$ and $\bm{\lambda}$ gives the necessary optimality conditions (NOCs).
\end{itemize}
\pause
\[
\begin{aligned}
L(\x,\bm{\lambda}) &= f(\x) + \sum_{i=1}^m \lambda_i h_i(\x), \\
\nabla_{\x} L(\x^*,\bm{\lambda}^*) &= 0, \quad \nabla_{\bm{\lambda}} L(\x^*,\bm{\lambda}^*) = 0
\end{aligned}
\]
\end{frame}
% --- Original text start ---
% Consider the equality-constrained finite-dimensional optimization problem ... define the Lagrangian ... NOCs are given as ... equation (4).
% --- Original text end ---

\begin{frame}{Hamiltonian and Costates}
\begin{itemize}
\item Extending the Lagrangian idea leads to the Hamiltonian for the OCP.
\item Costates $\bm{p}(t)$ play the role of multipliers enforcing the dynamics across time.
\item The Hamiltonian couples running cost $g$ and dynamics $a$ to encode first-order conditions compactly.
\end{itemize}
\pause
\[
H(\x, \bm{u}, \bm{p}, t) = g(\x, \bm{u}, t) + \bm{p}^{\top} a(\x, \bm{u}, t)
\]
\end{frame}
% --- Original text start ---
% Analogously ... define the Hamiltonian $H(\x(t), \bu(t), \p(t), t) := g(\x(t), \bu(t), t) + \p^\top(t) a(\x(t), \bu(t), t)$.
% --- Original text end ---

\begin{frame}{Pontryagin Necessary Conditions}
\begin{itemize}
\item Stationarity of the Hamiltonian produces coupled state, costate, and input equations.
\item Together with boundary conditions, these differential-algebraic equations characterize optimal trajectories.
\item Solving them exactly is rarely possible, motivating numerical boundary-value solvers.
\end{itemize}
\pause
\[
\begin{aligned}
\dot{\x}^*(t) &= \frac{\partial H}{\partial \bm{p}}(\cdot), &
\dot{\bm{p}}^*(t) &= -\frac{\partial H}{\partial \x}(\cdot), &
0 &= \frac{\partial H}{\partial \bm{u}}(\cdot)
\end{aligned}
\]
\end{frame}
% --- Original text start ---
% The NOCs are then given by ... equation (6) ... consisting of $2n$ differential equations and $m$ algebraic equations.
% --- Original text end ---

\begin{frame}{Boundary Condition Menu}
\begin{itemize}
\item Initial state $\x(t_0)=\x_0$ always supplies $n$ boundary conditions.
\item Depending on whether final time/state are fixed or free, additional conditions arise from the terminal cost gradient and Hamiltonian.
\item Free final time problems add one more condition, leading to $2n+1$ constraints overall.
\end{itemize}
\end{frame}
% --- Original text start ---
% Identifying unique solutions ... boundary conditions (7) ... summaries for the four scenarios (fixed/free final time/state).
% --- Original text end ---

\begin{frame}{Two-Point Boundary Value Problems}
\begin{itemize}
\item The coupled state-costate equations with split boundary data define a two-point boundary value problem (TPBVP).
\item Solvers such as \texttt{bvp4c} (MATLAB) or \texttt{scikits.bvp\_solver} (Python) target the standard TPBVP form.
\item When the OCP does not match the standard form, transformations can recover it.
\end{itemize}
\pause
\[
\dot{\mathbf{z}} = g(\mathbf{z}, t), \quad l(\mathbf{z}(t_0), \mathbf{z}(t_f)) = 0
\]
\end{frame}
% --- Original text start ---
% Most solvers ... standard form (8) ... some problems may not directly fit ... convert using methods from Ascher & Russell.
% --- Original text end ---

\begin{frame}{Free Final Time Trick}
\begin{itemize}
\item Introduce the normalized time $\tau = t / t_f$ so the horizon becomes $[0,1]$ even when $t_f$ is unknown.
\item Replace every $\frac{d(\cdot)}{dt}$ with $t_f \frac{d(\cdot)}{d\tau}$ and promote $t_f$ to a state with $\dot{r}=0$.
\item Substitute $r$ for $t_f$ in all equations and boundary conditions to use standard TPBVP solvers.
\end{itemize}
\end{frame}
% --- Original text start ---
% In optimal control settings ... define new variable $\tau = t/t_f$ ... steps (1)-(3) ... tricking the solver to think final time is 1.
% --- Original text end ---

\begin{frame}{Example Setup: Free Final-Time Double Integrator}
\begin{itemize}
\item Dynamics: $\ddot{x} = u$ with boundary conditions $x(0)=10$, $\dot{x}(0)=0$, $x(t_f)=0$, $\dot{x}(t_f)=0$.
\item Cost combines arrival time and control effort via weights $\alpha$ and $\beta$.
\item Converting to first-order form yields $\dot{x}_1 = x_2$, $\dot{x}_2 = u$ with $\x = [x_1\;x_2]^\top$.
\end{itemize}
\pause
\[
J = \tfrac{1}{2}\alpha t_f^2 + \int_{0}^{t_f} \tfrac{1}{2}\beta u^2(t)\,dt
\]
\end{frame}
% --- Original text start ---
% Consider a double integrator system ... boundary conditions ... cost function ... rewrite as first-order system ... boundary conditions for $x_1$, $x_2$.
% --- Original text end ---

\begin{frame}{Example NOCs and Boundary Conditions}
\begin{itemize}
\item Hamiltonian $H = \tfrac{1}{2}\beta u^2 + p_1 x_2 + p_2 u$ yields linear dynamics for states and costates.
\item Free final time with fixed final state imposes both terminal state constraints and a Hamiltonian condition.
\item $u^*(t)$ emerges directly from the stationarity condition $\beta u^* + p_2^* = 0$.
\end{itemize}
\pause
\[
\begin{aligned}
\dot{x}_1^* &= x_2^*, & \dot{x}_2^* &= u^*, & \dot{p}_1^* &= 0, & \dot{p}_2^* &= -p_1^*, & u^* &= -\tfrac{1}{\beta}p_2^*
\end{aligned}
\]
\end{frame}
% --- Original text start ---
% Now that the problem has been introduced ... Hamiltonian ... NOCs ... boundary conditions ... optimal control from stationarity.
% --- Original text end ---

\begin{frame}{Example Insights}
\begin{itemize}
\item Integrating the costate dynamics gives affine functions parameterized by constants $C_1, C_2$.
\item State trajectories become polynomials in time whose coefficients follow from boundary conditions.
\item The optimal arrival time scales as $t_f = (1800 \beta / \alpha)^{1/5}$, revealing the trade-off between effort and time.
\end{itemize}
\pause
\[
\begin{aligned}
p_1^* &= C_1, & p_2^* &= -C_1 t + C_2, & u^*(t) &= \tfrac{C_1}{\beta}t - \tfrac{C_2}{\beta}
\end{aligned}
\]
\end{frame}
% --- Original text start ---
% Analytical solution section ... expressions for $p_1^*$, $p_2^*$, $x_1^*$, $x_2^*$, constants, and $t_f = (1800 \beta/\alpha)^{1/5}$ ... comments on limiting cases.
% --- Original text end ---

\begin{frame}{Direct Methods Overview}
\begin{itemize}
\item Direct transcription converts the continuous OCP into a finite-dimensional NLP.
\item Decision variables become sampled states and controls defined on a mesh $t_0,\dots,t_N$.
\item Dynamics, costs, and constraints must each be approximated consistently at those nodes.
\end{itemize}
\end{frame}
% --- Original text start ---
% Unlike indirect methods, direct methods do not require ... Instead these methods directly discretize ... nonlinear programming problem.
% --- Original text end ---

\begin{frame}{Forward Euler Discretization}
\begin{itemize}
\item A simple transcription uses forward Euler to approximate the dynamics update between nodes.
\item With known steps $h_i = t_{i+1}-t_i$, the discrete model propagates states from controls.
\item This substitution replaces the continuous differential constraint in the NLP.
\end{itemize}
\pause
\[
\x_{i+1} = \x_i + h_i a(\x_i, \bm{u}_i, t_i)
\]
\end{frame}
% --- Original text start ---
% Applying a forward Euler time discretization ... equation (21) ... recursive expression for dynamics.
% --- Original text end ---

\begin{frame}{Finite-Dimensional NLP}
\begin{itemize}
\item The discrete decision set consists of $\{\x_i,\bm{u}_i\}_{i=0}^{N}$ while $\x_0$ stays fixed by the initial condition.
\item The integral cost becomes a weighted sum (e.g., Newton--Cotes trapezoidal rule).
\item Together, they define the nonlinear program solved by generic NLP solvers.
\end{itemize}
\pause
\[
\begin{aligned}
\min_{\x_i,\bm{u}_i}\; & h(\x_N,t_N) + \sum_{i=0}^{N-1} h_i g(\x_i,\bm{u}_i,t_i) \\
\text{s.t.}\; & \x_{i+1} = \x_i + h_i a(\x_i,\bm{u}_i,t_i),\; i=0,\dots,N-1
\end{aligned}
\]
\end{frame}
% --- Original text start ---
% Rewriting the original OCP ... approximation of the integral ... finite-dimensional NLP equation (22).
% --- Original text end ---

\begin{frame}{Consistency of Discretization}
\begin{itemize}
\item Forming the NLP Lagrangian and differentiating recovers discrete NOCs.
\item These conditions mirror the continuous-time NOCs once the same forward Euler approximations are applied.
\item Therefore refining the mesh ($h_i \to 0$) pushes the discrete solution toward the original OCP optimum.
\end{itemize}
\pause
\[
\nabla_{\x_i} L = h_i \frac{\partial g}{\partial \x} + h_i \Big(\frac{\partial a}{\partial \x}\Big)^{\top} \! \bm{\lambda}_i + (\bm{\lambda}_i - \bm{\lambda}_{i-1}) = 0
\]
\end{frame}
% --- Original text start ---
% The Lagrangian for (22) ... NOCs (23) ... compare with indirect method NOCs (24) and show convergence as $h_i \to 0$.
% --- Original text end ---

\begin{frame}{Reminder: Lagrange Multipliers}
\begingroup
\setbeamercolor{block title}{bg=gray!20,fg=black}
\setbeamercolor{block body}{bg=gray!10,fg=black}
\begin{block}{Concept Recap}
\begin{itemize}
\item Multipliers attach to each constraint to form $L(\x,\bm{\lambda})$, enabling unconstrained calculus tools.
\item Stationary points of $L$ satisfy both primal feasibility and dual feasibility simultaneously.
\item This idea extends directly to Hamiltonians and costates when constraints are differential equations.
\end{itemize}
\end{block}
\endgroup
\end{frame}
% --- Original text start ---
% Necessary optimality conditions for equality-constrained problems ... Lagrangian definition ... gradients must vanish.
% --- Original text end ---

\begin{frame}{Exercises}
\begin{itemize}
\item Practice translating theory into code via the online exercise on optimal control and trajectory optimization.
\item You will derive NOCs for a unicycle and solve them numerically using a TPBVP solver.
\item Repository: \url{https://github.com/PrinciplesofRobotAutonomy/AA274A_HW1}
\end{itemize}
\end{frame}
% --- Original text start ---
% Complete Extra Problem ... URL ... compute a dynamically feasible and optimal trajectory for a unicycle robot.
% --- Original text end ---

\end{document}
