\documentclass{beamer}

% Shared Beamer configuration for Principles of Robot Autonomy decks
\usetheme{Boadilla}
%\usecolortheme{seagull}

% Use Palatino to match the book's Tufte style
\usepackage{mathpazo}
\usefonttheme{serif}

\usepackage{amsmath}
\usepackage{amssymb}
\usepackage{bm}
\usepackage{graphicx}
\usepackage{enumitem}
\usepackage{xparse}
\usepackage{array}

% Increase vertical spacing between list items for readability.
\setlist[itemize]{itemsep=0.75em}
\setlist[enumerate]{itemsep=0.75em}

% Define bold symbol macros if not already defined
\providecommand{\x}{\boldsymbol{x}}
\providecommand{\z}{\boldsymbol{z}}
\providecommand{\q}{\boldsymbol{q}}
\providecommand{\bxi}{\boldsymbol{\xi}}
\providecommand{\bu}{\boldsymbol{u}}
\providecommand{\bmu}{\boldsymbol{\mu}}
\providecommand{\bSigma}{\boldsymbol{\Sigma}}
\providecommand{\y}{\boldsymbol{y}}
\providecommand{\m}{\mathcal{M}}
\providecommand{\bc}{\boldsymbol{c}}
\providecommand{\R}{\mathbb{R}}

% Animated lists by default, with override via optional overlay argument.
\beamerdefaultoverlayspecification{<+->}

% Paragraphs reveal sequentially by default.
\AtBeginDocument{%
  \ifx\paragraph\undefined
    \newcommand{\paragraph}[1]{\textbf{#1}\par\onslide<+->{} }
  \else
    \let\origparagraph\paragraph
    \renewcommand{\paragraph}[1]{\origparagraph{#1}\par\onslide<+->{} }
  \fi
}

% Add other shared packages or theme settings above this line as needed.

\author[COMP 4766]{Adapted from ``Principles of Robot Autonomy'' by D. Gammelli, J. Lorenzetti,
K. Luo, G. Zardini, and M. Pavone}
\date{\today}


\title{Chapter 19 - Finite State Machines}

\begin{document}

\frame{\titlepage}

\begin{frame}{From Continuous to Discrete Logic}
\begin{itemize}
\item High-level tasks (e.g., pickup and delivery) need discrete progress tracking in addition to continuous pose.
\item FSMs let us capture plan stages, sensor health, or safety contexts as finite symbols.
\item They complement motion planners and estimators by supervising task flow.
\end{itemize}
\end{frame}
% --- Original text start ---
% Intro paragraphs motivating discrete abstractions on top of continuous control and perception stacks.
% --- Original text end ---

\begin{frame}{What FSMs Offer}
\begin{itemize}
\item Specify desired programs or behaviors (e.g., vending machines, ATMs).
\item Model existing controllers interacting with their environment.
\item Predict future behavior from input histories.
\end{itemize}
\end{frame}
% --- Original text start ---
% FSM definition section listing specification, modeling, and prediction uses.
% --- Original text end ---

\begin{frame}{FSM Ingredients}
\begin{itemize}
\item Finite state set $S$, input alphabet $I$, and output alphabet $O$.
\item Next-state function $n(i_t, s_t) \to s_{t+1}$ and output function $o(i_t, s_t) \to o_t$.
\item Initial state $s_0$ anchors execution; history impacts behavior through $s_t$.
\end{itemize}
\pause
\begin{equation*}
\text{FSM} = (S, I, O, n, o, s_0)
\end{equation*}
\end{frame}
% --- Original text start ---
% Mathematical description of FSM elements and notation.
% --- Original text end ---

\begin{frame}{Graphical View}
\begin{itemize}
\item Nodes depict states; directed edges encode transitions for input/output pairs.
\item Edge labels typically include triggering input and emitted output.
\item Helps sanity-check reachability and transition coverage.
\end{itemize}
\pause
\begin{center}
    \includegraphics[width=0.55\linewidth]{../tex/figs/ch19_figs/fsm.png}
\end{center}
\end{frame}
% --- Original text start ---
% Discussion of graph representations and Figure \ref{fig:fsm}.
% --- Original text end ---

\begin{frame}{Real-World Complexity}
\begin{itemize}
\item Even moderate state counts can explode into dense transition graphs.
\item PX4 flight stack FSM highlights continual evolution of autonomy logic.
\item Designers balance completeness with maintainability.
\end{itemize}
\pause
\begin{center}
    \includegraphics[width=0.72\linewidth]{../tex/figs/ch19_figs/px4_state_machine.png}
\end{center}
\end{frame}
% --- Original text start ---
% Paragraph describing PX4 FSM complexity and Figure \ref{fig:px4fsm}.
% --- Original text end ---

\begin{frame}{Parking Gate State Space}
\begin{itemize}
\item Abstract continuous gate motion into four discrete modes.
\item Sensor-driven inputs report car presence and gate pose.
\item Outputs command raise, lower, or hold actions.
\end{itemize}
\pause
\begin{equation*}
S = \{\text{down}, \text{raising}, \text{up}, \text{lowering}\}
\end{equation*}
\end{frame}
% --- Original text start ---
% Example \ref{ex:parkinggate} definition of S, I, O sets.
% --- Original text end ---

\begin{frame}{Parking Gate Behavior}
\begin{itemize}
\item Next-state logic ties sensor facts to desired actuator commands.
\item Example: $n(\text{car waiting},\text{down}) = \text{raising}$ with output ``raise''.
\item Graph view clarifies the closed-loop policy.
\end{itemize}
\pause
\begin{center}
    \includegraphics[width=0.9\linewidth]{../tex/figs/ch19_figs/parkinggate_fsm.png}
\end{center}
\end{frame}
% --- Original text start ---
% Continued parking gate discussion plus Figure \ref{fig:parkinggate}.
% --- Original text end ---

\begin{frame}{Managing FSM Complexity}
\begin{itemize}
\item Redundant states inflate transitions; seek equivalent-state reductions.
\item Hierarchical FSMs cluster related states into super-states with generalized transitions.
\item Compositions (cascade, parallel, feedback) reuse patterns without rewriting logic.
\end{itemize}
\end{frame}
% --- Original text start ---
% FSM architecture strategies: reducing states, hierarchical FSMs, compositions overview.
% --- Original text end ---

\begin{frame}{State Reduction Algorithm}
\begin{itemize}
\item Partition states by identical outputs, then iteratively refine via next-state destinations.
\item Continue until no partition can be split; each block becomes a reduced state.
\item Guarantees equivalent input/output behavior with fewer nodes.
\end{itemize}
\pause
\begin{equation*}
\{\text{Reset},0,1,00,10\} \Rightarrow \{\text{Reset}\}, \{0,1\}, \{00,10\}
\end{equation*}
\end{frame}
% --- Original text start ---
% Algorithm description and Example \ref{ex:sequence} partition steps.
% --- Original text end ---

\begin{frame}{Sequence Detector Example}
\begin{itemize}
\item Original FSM detected 010 or 110 with seven partial-sequence states.
\item Applying reduction yields four aggregate states while preserving True/False outputs.
\item Simplified transition table retains detection accuracy with fewer edges.
\end{itemize}
\end{frame}
% --- Original text start ---
% Tables \ref{tab:sequence} and \ref{tab:sequence2} showing reduction results.
% --- Original text end ---

\begin{frame}{Implementation Pattern}
\begin{itemize}
\item Treat the FSM as a class: store current state and expose an update loop.
\item Use sensor callbacks to populate inputs, then route through if/else transition logic.
\item Publish commands each cycle; ROS example uses topics \texttt{/car\_sensor}, \texttt{/gate\_sensor}, \texttt{/gate\_cmd}.
\end{itemize}
\end{frame}
% --- Original text start ---
% Python example implementing parking gate FSM with rospy subscribers and publisher.
% --- Original text end ---

\begin{frame}{Visualization Tools}
\begin{itemize}
\item SMACH provides a ROS-friendly GUI for building and introspecting FSMs.
\item Useful for debugging transitions and documenting autonomy stacks.
\item See the ROS Wiki (http://wiki.ros.org/smach) for tutorials.
\end{itemize}
\end{frame}
% --- Original text start ---
% Mention of SMACH as an FSM visualization tool analogous to RViz.
% --- Original text end ---

\begin{frame}{Reminder: FSM Toolbox}
\begingroup
\setbeamercolor{block title}{bg=gray!20,fg=black}
\setbeamercolor{block body}{bg=gray!10,fg=black}
\begin{block}{Key Takeaways}
\begin{itemize}
\item Discrete abstractions track task progress and interface with continuous controllers.
\item Graphical FSMs aid reasoning, while reduction, hierarchy, and composition tame complexity.
\item Implementations lean on clean state-update loops and tooling such as SMACH for visibility.
\end{itemize}
\end{block}
\endgroup
\end{frame}
% --- Original text start ---
% Chapter themes: motivation, architectural strategies, and implementation aids.
% --- Original text end ---

\end{document}
