\documentclass{beamer}

% Shared Beamer configuration for Principles of Robot Autonomy decks
\usetheme{Boadilla}
%\usecolortheme{seagull}

% Use Palatino to match the book's Tufte style
\usepackage{mathpazo}
\usefonttheme{serif}

\usepackage{amsmath}
\usepackage{amssymb}
\usepackage{bm}
\usepackage{graphicx}
\usepackage{enumitem}
\usepackage{xparse}
\usepackage{array}

% Increase vertical spacing between list items for readability.
\setlist[itemize]{itemsep=0.75em}
\setlist[enumerate]{itemsep=0.75em}

% Define bold symbol macros if not already defined
\providecommand{\x}{\boldsymbol{x}}
\providecommand{\z}{\boldsymbol{z}}
\providecommand{\q}{\boldsymbol{q}}
\providecommand{\bxi}{\boldsymbol{\xi}}
\providecommand{\bu}{\boldsymbol{u}}
\providecommand{\bmu}{\boldsymbol{\mu}}
\providecommand{\bSigma}{\boldsymbol{\Sigma}}
\providecommand{\y}{\boldsymbol{y}}
\providecommand{\m}{\mathcal{M}}
\providecommand{\bc}{\boldsymbol{c}}
\providecommand{\R}{\mathbb{R}}

% Animated lists by default, with override via optional overlay argument.
\beamerdefaultoverlayspecification{<+->}

% Paragraphs reveal sequentially by default.
\AtBeginDocument{%
  \ifx\paragraph\undefined
    \newcommand{\paragraph}[1]{\textbf{#1}\par\onslide<+->{} }
  \else
    \let\origparagraph\paragraph
    \renewcommand{\paragraph}[1]{\origparagraph{#1}\par\onslide<+->{} }
  \fi
}

% Add other shared packages or theme settings above this line as needed.

\author[COMP 4766]{Adapted from ``Principles of Robot Autonomy'' by D. Gammelli, J. Lorenzetti,
K. Luo, G. Zardini, and M. Pavone}
\date{\today}


\title{Mobile Robot Kinematics: Part 2}

\begin{document}

\begin{frame}[plain]
\titlepage
\end{frame}

\begin{frame}{Source Material}
    Covering the following sections from ``Principles of Robot Autonomy'':
    \begin{itemize}
    \item 1.2:
    \begin{itemize}
        \item 1.2.1 - 1.2.4 covered fully
        \item 1.2.5 not covered
    \end{itemize}
    \end{itemize}
\end{frame}



\begin{frame}{Kinematics and Dynamics}
    \begin{block}{Kinematics}
    Kinematics is the study of the motion of physical
systems, concerned with describing positions, velocities, and accelerations over time, without reference to the forces or torques that produce the motion.
    \end{block}
    \pause

    \begin{block}{Dynamics}
    Dynamics is the study of the motion of physical
systems as determined by the forces and torques acting upon them. It seeks to
relate a system’s motion to the underlying physical causes of that motion, such
as gravity, friction, or applied inputs.
    \end{block}
    \pause

    Both perspectives are necessary to accurately describe a robot's motion. However, we might choose to focus on only one to yield a simpler model.
\end{frame}


\begin{frame}
    \textbf{In mobile robotics we often ignore dynamics: why?}
    \begin{itemize}
        \item Most mobile robots utilize a layered control architecture. These controllers handle the ``dirty work'' of dynamics (voltage, torque, friction) so that the higher-level system can simply treat the robot as a device that accepts a velocity.

        \item Kinematic models are sufficient when physical forces are small: Many indoor robots move at speeds where centrifugal forces and inertia are small.
        
        \item Solving dynamics equations is typically much slower than kinematic equations. Dynamics equations often require more knowledge of parameters (e.g. mass, centre-of-mass, friction coefficients, ...).
    \end{itemize}
\end{frame}


\begin{frame}{When Dynamics CANNOT be Ignored}
    
    \begin{table}
    \begin{tabular}{|l|l|}
    \hline
    \rule[-1.2ex]{0pt}{4ex} \textbf{Kinematics-only if...} & \textbf{Incorporate Dynamics if...} \\ \hline
    Slow/Moderate speed & High-speed maneuvers \\ \hline
    Heavy-duty motors & Limited torque/underactuated \\ \hline
    Wheeled/Flat ground & Legged or Aerial robots \\ \hline
    Rigid environment & Soft or slippery terrain \\ \hline
    \end{tabular}
    \end{table}
\end{frame}


\begin{frame}{Generalized Coordinates}
\begin{itemize}
\item The \emph{generalized coordinates} of a robot is the set of variables that specify the position of every point on the robot.

\item Denoted $\mathbf{q}(t) \in \mathbb{R}^{n_g}$.

\item The generalized coordinates may be the same as the full system state $\mathbf{x}$ or they may represent a subset of the state (e.g. the state may include the generalized coordinates plus velocities).
\end{itemize}
\end{frame}


\begin{frame}{Example: A Wheel}
\begin{center}
    \includegraphics[width=0.6\linewidth]{../book/figs/ch01_figs/wheel_roll.png}
\end{center}
\begin{itemize}
\item $(x,y)$ are the coordinates of the wheel's contact point with the ground.
\item $\theta$ is the heading angle, relative to the $x-axis$.
\item Generalized coordinates: $\q = [x, y, \theta]^\top$.
\item Note that other choices of generalized coordinates are possible. For example, rather than the Cartesian coordinates $(x, y)$ we could use polar coordinates.
\end{itemize}
\pause
\end{frame}


\begin{frame}{Kinematic Constraints}
    \begin{block}{Kinematic Constraints}
    Kinematic constraints are a set of constraints imposed on the generalized coordinates, $\q$, and generalized velocities, $\dot{\q}$. We express kinematic constraints mathematically as:
    $$a_i(\q, \dot{\q}) = 0, \qquad i = 1, ..., k < n_g,$$
    where $k$ is the number of constraints and $n_g$ is the number of generalized coordinates.
    \end{block}
    \pause\vspace{0.3cm}
    Intuitively: a kinematic constraint tells us one or both of the following:
    \begin{itemize}
        \item ``you can't be here''
        \item ``you can't move in this direction''
    \end{itemize}
\end{frame}


\begin{frame}{Pfaffian Constraints}
\begin{itemize}
\item Many robotic constraints are linear in $\dot{\q}$, yielding Pfaffian form 
$$\mathbf{a}_i^\top(\q)\dot{\q} = 0, \qquad i = 1, ..., k < n_g.$$

\item Example: we are modelling a robot that lives only on the $x-$axis, but we choose to represent its generalized coordinates as $\q = [x, y]^\top$.
\item We could constrain its velocity to be along the $x-$axis through the following Pfaffian-form kinematic constraint:
\item $$\begin{bmatrix} 0 & 1\end{bmatrix} \begin{bmatrix} \dot{x} \\ \dot{y} \end{bmatrix} = 0$$
\item Meanwhile, this example is rather contrived. It would be better to just represent $\q$ as $x$.
\end{itemize}
\end{frame}
% --- Original text start ---
% Kinematic constraints in robotics applications are often linear with respect to the generalized velocities... expressed as Pfaffian constraints $a_i^\top(\bxi)\dot{\bxi}=0$ or $A^\top(\bxi)\dot{\bxi}=\mathbf{0}$.
% --- Original text end ---

\begin{frame}{Example: Pendulum Constraint}
\begin{center}
    \includegraphics[width=0.2\linewidth]{../book/figs/ch01_figs/pendulum.png}
\end{center}
\pause
\begin{itemize}
\item Consider a pendulum: a mass at the end of a rod, swinging freely from the origin.
\item Choose $\q = [x, y]^\top$ wherer $(x, y)$ are the coordinates of the mass.
\item The rod is assumed to be rigid and of length $L$. This assumption yields a kinematic constraint.
$$a_1(\q, \dot{q}) = x^2 + y^2 - L^2 = 0$$
\end{itemize}
\noindent
\begin{minipage}{0.2\linewidth}
    \centering
    \includegraphics[width=\linewidth]{../book/figs/ch01_figs/pendulum.png}
\end{minipage}%
\hfill
\begin{minipage}{0.6\linewidth}
    \begin{itemize}
    \item This is not in Pfaffian form, but we can obtain the constraint in that form by differentiating with respect to time.
    $$2x\dot{x} + 2y\dot{y} = 0$$
    \end{itemize}
\end{minipage}
\begin{itemize}
\item Once again, it might be wiser to choose $\q = \theta$, which would make this constraint redundant.
\end{itemize}
\end{frame}


\begin{frame}{Example: No-Slip Wheel}
\begin{itemize}
\item A rolling wheel with coordinates $[x,y,\theta]^\top$ maintains zero lateral velocity.
\item Using basis vectors aligned with the wheel gives the constraint $\dot{x}\sin\theta - \dot{y}\cos\theta = 0$.
\item The Pfaffian form keeps the constraint linear in $\dot{x}$ and $\dot{y}$.
\end{itemize}
\pause
\[
\dot{x}\sin\theta - \dot{y}\cos\theta = 0
\]
\end{frame}
% --- Original text start ---
% The no-slip condition means the velocity component in the lateral direction is zero, leading to $\dot{x} \sin\theta - \dot{y} \cos \theta = 0$.
% --- Original text end ---

\begin{frame}{Holonomic Constraints}
\begin{itemize}
\item Holonomic constraints depend only on configuration: $h_i(\bxi)=0$.
\item Differentiating $h_i$ reproduces the equivalent Pfaffian form.
\item Holonomic systems contain only such integrable constraints.
\end{itemize}
\pause
\[
\frac{dh_i(\bxi)}{dt} = \frac{dh_i(\bxi)}{d\bxi}\dot{\bxi} = 0
\]
\end{frame}
% --- Original text start ---
% Holonomic constraints ... can be expressed as $h_i(\bxi)=0$ and differentiating yields the Pfaffian form.
% --- Original text end ---

\begin{frame}{Holonomic Access}
\begin{itemize}
\item Holonomic constraints restrict the reachable configurations to an $(n-k)$-dimensional subset.
\item A pendulum mass remains on a circle of radius $L$, reducing the space to one dimension.
\item The no-slip wheel constraint is not integrable, so it does not limit which $(x,y)$ can be reached.
\end{itemize}
\end{frame}
% --- Original text start ---
% Holonomic constraints restrict the accessible configurations ... pendulum example vs wheel example.
% --- Original text end ---

\begin{frame}{Nonholonomic Constraints}
\begin{itemize}
\item Nonholonomic constraints are Pfaffian but cannot be integrated to $h_i(\bxi)=0$.
\item They restrict allowable motions between configurations rather than the configurations themselves.
\item Any system subject to at least one such constraint is nonholonomic.
\end{itemize}
\end{frame}
% --- Original text start ---
% Constraints that can be described in Pfaffian form, but cannot be integrated to $h_i(\bxi) = 0$ are called nonholonomic.
% --- Original text end ---

\begin{frame}{Reminder: Null Space}
\begingroup
\setbeamercolor{block title}{bg=gray!20,fg=black}
\setbeamercolor{block body}{bg=gray!10,fg=black}
\begin{block}{Concept Recap}
\begin{itemize}
\item The null space of a matrix $A$ is the set of all vectors $\mathbf{x}$ such that $A\mathbf{x} = \mathbf{0}$, where $\mathbf{0}$ denotes the zero vector.
\item Emphasizing the zero vector clarifies that every component of $A\mathbf{x}$ must vanish simultaneously.
\item Example: For $A = \begin{bmatrix} 1 & 2\\ 3 & 6 \end{bmatrix}$, any scalar multiple of $\begin{bmatrix} 2\\ -1 \end{bmatrix}$ lies in the null space because $A\begin{bmatrix} 2\\ -1 \end{bmatrix} = \mathbf{0}$.
\end{itemize}
\end{block}
\endgroup
\end{frame}
% --- Original text start ---
% The restriction of instantaneous motion that is induced by a nonholonomic constraint can be interpreted by considering the Pfaffian form $a_i(\bxi)^\top \dot{\bxi} = 0$. It is clear that for any coordinate $\bxi$, this constraint limits the motion ($\dot{\bxi}$) to lie in the null space of $a_i(\bxi)^\top $.
% --- Original text end ---

\begin{frame}{Instantaneous Motion Limits}
\begin{itemize}
\item Nonholonomic constraints confine $\dot{\bxi}$ to the null space of $a_i^\top(\bxi)$.
\item For the wheel, the null space is spanned by the rolling direction and pure spin.
\item Any feasible velocity is a combination of those motions.
\end{itemize}
\end{frame}
% --- Original text start ---
% Consider again the wheel example ... The null space of $a_i(\bxi)^\top$ is spanned by $[\cos \theta, \: \sin \theta,\: 0]$ and $[0, \: 0,\: 1]$, indicating admissible motions combine rolling and spinning.
% --- Original text end ---

\begin{frame}{Building Kinematic Models}
\begin{itemize}
\item After identifying constraints, express motion as $\dot{\bxi} = G(\bxi)\bu$.
\item Columns of $G(\bxi)$ span the null space of $A^\top(\bxi)$, guaranteeing constraint satisfaction.
\item Inputs $\bu$ parameterize allowable motions of dimension $n-k$.
\end{itemize}
\end{frame}
% --- Original text start ---
% Once an appropriate set of generalized coordinates $\bxi \in \R^n$ and all relevant kinematic constraints have been identified ... $\dot{\bxi} = G(\bxi)u$ with column space spanning the null space of $A^\top(\bxi)$.
% --- Original text end ---

\begin{frame}{Example: Wheel Kinematic Model}
\begin{itemize}
\item The wheel's constraint matrix has a one-dimensional row space, so $G(\bxi)$ has two columns.
\item Choose columns $[\cos\theta\;\sin\theta\;0]^\top$ (rolling) and $[0\;0\;1]^\top$ (spin).
\item Inputs $u_1$ and $u_2$ become forward speed and rotational rate, respectively.
\end{itemize}
\pause
\[
\begin{bmatrix}
\dot{x}\\\dot{y}\\\dot{\theta}
\end{bmatrix} =
\begin{bmatrix}
\cos\theta & 0\\
\sin\theta & 0\\
0 & 1
\end{bmatrix}
\begin{bmatrix}
 u_1\\ u_2
\end{bmatrix}
\]
\end{frame}
% --- Original text start ---
% Consider again the wheel example ... The null space is spanned by $[\cos \theta, \: \sin \theta,\: 0]$ and $[0, \: 0,\: 1]$, giving the model in \eqref{eq:wheelkinmodel}.
% --- Original text end ---



\end{document}
