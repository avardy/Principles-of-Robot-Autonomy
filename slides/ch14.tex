\documentclass{beamer}

% Shared Beamer configuration for Principles of Robot Autonomy decks
\usetheme{Boadilla}
%\usecolortheme{seagull}

% Use Palatino to match the book's Tufte style
\usepackage{mathpazo}
\usefonttheme{serif}

\usepackage{amsmath}
\usepackage{amssymb}
\usepackage{bm}
\usepackage{graphicx}
\usepackage{enumitem}
\usepackage{xparse}
\usepackage{array}

% Increase vertical spacing between list items for readability.
\setlist[itemize]{itemsep=0.75em}
\setlist[enumerate]{itemsep=0.75em}

% Define bold symbol macros if not already defined
\providecommand{\x}{\boldsymbol{x}}
\providecommand{\z}{\boldsymbol{z}}
\providecommand{\q}{\boldsymbol{q}}
\providecommand{\bxi}{\boldsymbol{\xi}}
\providecommand{\bu}{\boldsymbol{u}}
\providecommand{\bmu}{\boldsymbol{\mu}}
\providecommand{\bSigma}{\boldsymbol{\Sigma}}
\providecommand{\y}{\boldsymbol{y}}
\providecommand{\m}{\mathcal{M}}
\providecommand{\bc}{\boldsymbol{c}}
\providecommand{\R}{\mathbb{R}}

% Animated lists by default, with override via optional overlay argument.
\beamerdefaultoverlayspecification{<+->}

% Paragraphs reveal sequentially by default.
\AtBeginDocument{%
  \ifx\paragraph\undefined
    \newcommand{\paragraph}[1]{\textbf{#1}\par\onslide<+->{} }
  \else
    \let\origparagraph\paragraph
    \renewcommand{\paragraph}[1]{\origparagraph{#1}\par\onslide<+->{} }
  \fi
}

% Add other shared packages or theme settings above this line as needed.

\author[COMP 4766]{Adapted from ``Principles of Robot Autonomy'' by D. Gammelli, J. Lorenzetti,
K. Luo, G. Zardini, and M. Pavone}
\date{\today}


\title{Chapter 14 - Parametric Filters}

\begin{document}

\frame{\titlepage}

\begin{frame}{Why Parametric Filters}
\begin{itemize}
\item Bayes filter is exact but rarely tractable; approximations need structure.
\item Discrete grids handle finite states, but continuous robots need compact beliefs.
\item Parametric filters summarize beliefs with a few parameters (e.g., Gaussian moments).
\end{itemize}
\end{frame}
% --- Original text start ---
% Motivation for parametric filters as tractable Bayes filter implementations.
% --- Original text end ---

\begin{frame}{Gaussian Primer}
\begin{itemize}
\item Univariate pdf: $p(x)=\frac{1}{\sqrt{2\pi\sigma^2}}e^{-\frac{1}{2}\frac{(x-\mu)^2}{\sigma^2}}$.
\item Multivariate pdf uses mean $\bmu$ and covariance $\bSigma$ in $\R^n$.
\item Notation: $X\sim\mathcal{N}(\bmu,\bSigma)$ encodes the entire belief.
\end{itemize}
\pause
\begin{center}
    \includegraphics[width=0.8\linewidth]{../tex/figs/ch14_figs/gaussians.png}
\end{center}
\end{frame}
% --- Original text start ---
% Section introducing Gaussian distributions and Figure gaussians.
% --- Original text end ---

\begin{frame}{Useful Gaussian Properties}
\begin{itemize}
\item Affine transforms preserve Gaussian form: $Y=AX+b$ stays normal.
\item Sum of independent Gaussians is Gaussian with summed means/covariances.
\item Product of Gaussians is Gaussian, enabling analytic measurement updates.
\end{itemize}
\end{frame}
% --- Original text start ---
% Enumerated properties exploited by filters.
% --- Original text end ---

\begin{frame}{Kalman Filter Setup}
\begin{itemize}
\item Model belief as $bel(\x_t)=\mathcal{N}(\bmu_t,\bSigma_t)$.
\item Requires Gaussian prior $\x_0\sim\mathcal{N}(\bmu_0,\bSigma_0)$.
\item Prediction and correction must keep the belief Gaussian.
\end{itemize}
\end{frame}
% --- Original text start ---
% Intro paragraph defining Gaussian belief.
% --- Original text end ---

\begin{frame}{Linear-Gaussian Assumptions}
\begin{itemize}
\item Dynamics: $\x_t=A_t\x_{t-1}+B_t\bu_t+\bm\epsilon_t$, $\bm\epsilon_t\sim\mathcal{N}(0,\bm R_t)$.
\item Measurement: $\z_t=C_t\x_t+\bm\delta_t$, $\bm\delta_t\sim\mathcal{N}(0,\bm Q_t)$.
\item These affine models guarantee Gaussian beliefs after each update.
\end{itemize}
\end{frame}
% --- Original text start ---
% Section on KF assumptions and noise models.
% --- Original text end ---

\begin{frame}{Kalman Filter Recursion}
\begin{itemize}
\item Prediction: $\bar{\bmu}_t=A_t\bmu_{t-1}+B_t\bu_t$, $\bar{\bSigma}_t=A_t\bSigma_{t-1}A_t^\top+\bm R_t$.
\item Kalman gain: $K_t=\bar{\bSigma}_t C_t^\top(C_t\bar{\bSigma}_t C_t^\top+\bm Q_t)^{-1}$.
\item Correction: $\bmu_t=\bar{\bmu}_t+K_t(\z_t-C_t\bar{\bmu}_t)$, $\bSigma_t=(I-K_tC_t)\bar{\bSigma}_t$.
\end{itemize}
\end{frame}
% --- Original text start ---
% Algorithm  Kalman filter steps.
% --- Original text end ---

\begin{frame}{Kalman Filter Insights}
\begin{itemize}
\item Prediction uses affine/Gaussian closure: no integrals required.
\item Correction equals product of Gaussians; matrix inversion lemma yields standard form.
\item Efficient for continuous states but inherently unimodal and linearized.
\end{itemize}
\end{frame}
% --- Original text start ---
% Practical considerations and derivation commentary.
% --- Original text end ---

\begin{frame}{Toward Nonlinear Models}
\begin{itemize}
\item Many robots have nonlinear motion and sensing ($g,h$ functions).
\item Extended Kalman Filter (EKF) keeps Gaussian beliefs but linearizes models.
\item Unscented Kalman Filter (UKF) avoids explicit Jacobians via sigma points.
\end{itemize}
\end{frame}
% --- Original text start ---
% Transition into EKF/UKF sections.
% --- Original text end ---

\begin{frame}{EKF Prediction}
\begin{itemize}
\item Nonlinear dynamics: $\x_t=g(\bu_t,\x_{t-1})+\bm\epsilon_t$.
\item Linearize about $\bmu_{t-1}$: Jacobian $G_t=\nabla_{\x}g(\bu_t,\bmu_{t-1})$.
\item Updates: $\bar{\bmu}_t=g(\bu_t,\bmu_{t-1})$, $\bar{\bSigma}_t=G_t\bSigma_{t-1}G_t^\top+\bm R_t$.
\end{itemize}
\end{frame}
% --- Original text start ---
% EKF prediction derivation using Taylor expansion.
% --- Original text end ---

\begin{frame}{EKF Correction}
\begin{itemize}
\item Measurement linearized at $\bar{\bmu}_t$: $H_t=\nabla_{\x}h(\bar{\bmu}_t)$.
\item Gain: $K_t=\bar{\bSigma}_t H_t^\top(H_t\bar{\bSigma}_t H_t^\top+\bm Q_t)^{-1}$.
\item Update: $\bmu_t=\bar{\bmu}_t+K_t(\z_t-h(\bar{\bmu}_t))$, $\bSigma_t=(I-K_tH_t)\bar{\bSigma}_t$.
\end{itemize}
\end{frame}
% --- Original text start ---
% EKF measurement update equations.
% --- Original text end ---

\begin{frame}{EKF Caveats}
\begin{itemize}
\item Accuracy hinges on first-order Taylor approximation; large nonlinearities can diverge.
\item Still enforces a single Gaussian, so multi-modal beliefs are missed.
\item Requires Jacobians, which may be tedious or brittle for complex sensors.
\end{itemize}
\end{frame}
% --- Original text start ---
% EKF practical considerations.
% --- Original text end ---

\begin{frame}{Unscented Kalman Filter}
\begin{itemize}
\item Represent Gaussian via sigma points that capture mean/covariance statistics.
\item Propagate each sigma point through nonlinear $g$ and $h$ exactly.
\item Recombine weighted points to recover updated Gaussian without Jacobians.
\end{itemize}
\end{frame}
% --- Original text start ---
% UKF overview section.
% --- Original text end ---

\begin{frame}{Exercises}
\begin{itemize}
\item HW4 Problem 1: implement EKF localization with known landmark map.
\item Practice linear feature extraction plus EKF measurement models.
\item Repository: \url{https://github.com/PrinciplesofRobotAutonomy/AA274A_HW4}.
\end{itemize}
\end{frame}
% --- Original text start ---
% Exercise description for EKF localization.
% --- Original text end ---

\begin{frame}{Reminder: Parametric Filtering}
\begingroup
\setbeamercolor{block title}{bg=gray!20,fg=black}
\setbeamercolor{block body}{bg=gray!10,fg=black}
\begin{block}{Key Takeaways}
\begin{itemize}
\item Linear-Gaussian models yield the classic Kalman filter with exact Gaussian updates.
\item EKF and UKF extend the idea to nonlinear robots while retaining Gaussian state beliefs.
\item Watch for model mismatch and unimodal limits when deploying in the field.
\end{itemize}
\end{block}
\endgroup
\end{frame}
% --- Original text start ---
% Summary tying KF, EKF, and UKF guidance.
% --- Original text end ---

\end{document}
