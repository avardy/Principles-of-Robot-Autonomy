\documentclass{beamer}

% Shared Beamer configuration for Principles of Robot Autonomy decks
\usetheme{Boadilla}
%\usecolortheme{seagull}

% Use Palatino to match the book's Tufte style
\usepackage{mathpazo}
\usefonttheme{serif}

\usepackage{amsmath}
\usepackage{amssymb}
\usepackage{bm}
\usepackage{graphicx}
\usepackage{enumitem}
\usepackage{xparse}
\usepackage{array}

% Increase vertical spacing between list items for readability.
\setlist[itemize]{itemsep=0.75em}
\setlist[enumerate]{itemsep=0.75em}

% Define bold symbol macros if not already defined
\providecommand{\x}{\boldsymbol{x}}
\providecommand{\z}{\boldsymbol{z}}
\providecommand{\q}{\boldsymbol{q}}
\providecommand{\bxi}{\boldsymbol{\xi}}
\providecommand{\bu}{\boldsymbol{u}}
\providecommand{\bmu}{\boldsymbol{\mu}}
\providecommand{\bSigma}{\boldsymbol{\Sigma}}
\providecommand{\y}{\boldsymbol{y}}
\providecommand{\m}{\mathcal{M}}
\providecommand{\bc}{\boldsymbol{c}}
\providecommand{\R}{\mathbb{R}}

% Animated lists by default, with override via optional overlay argument.
\beamerdefaultoverlayspecification{<+->}

% Paragraphs reveal sequentially by default.
\AtBeginDocument{%
  \ifx\paragraph\undefined
    \newcommand{\paragraph}[1]{\textbf{#1}\par\onslide<+->{} }
  \else
    \let\origparagraph\paragraph
    \renewcommand{\paragraph}[1]{\origparagraph{#1}\par\onslide<+->{} }
  \fi
}

% Add other shared packages or theme settings above this line as needed.

\author[COMP 4766]{Adapted from ``Principles of Robot Autonomy'' by D. Gammelli, J. Lorenzetti,
K. Luo, G. Zardini, and M. Pavone}
\date{\today}


\title{Chapter 18 - Sensor Fusion}

\begin{document}

\begin{frame}[plain]
\titlepage
\end{frame}

\begin{frame}{Why Sensor Fusion?}
\begin{itemize}
\item Combining heterogeneous sensors reduces the uncertainty that any single modality leaves behind.
\item Redundancy and diversity guard against range, FOV, or weather-related failures.
\item Fusion enables more reliable localization and perception pipelines for autonomy stacks.
\end{itemize}
\pause
\begin{center}
    \includegraphics[width=0.75\linewidth]{../book/figs/ch18_figs/carsensors.png}
\end{center}
\end{frame}
% --- Original text start ---
% Intro paragraphs plus Figure \ref{fig:reduceuncertainty} on sensor strengths/weaknesses and motivation for multi-sensor fusion.
% --- Original text end ---

\begin{frame}{Data Imperfections}
\begin{itemize}
\item Noise, bias, and coarse quantization all degrade raw measurements.
\item Correlations, disparities, and inconsistencies appear simultaneously in practice.
\item Robust fusion must withstand multiple imperfection types acting at once.
\end{itemize}
\end{frame}
% --- Original text start ---
% Data-related taxonomy section discussing uncertainty, imprecision, granularity, correlation, disparity, inconsistency.
% --- Original text end ---

\begin{frame}{Fusion Relationships}
\begin{itemize}
\item Low-, intermediate-, and high-level fusion respectively mix signals, features, and decisions.
\item Competitive sensors provide redundant measurements, complementary ones fill gaps, and cooperative sensors unlock new observables.
\item Reliability grows with competition, completeness with complementarity, and scope with cooperation.
\end{itemize}
\end{frame}
% --- Original text start ---
% Fusion-related taxonomy paragraphs defining low/intermediate/high level fusion and competitive/complementary/cooperative roles.
% --- Original text end ---

\begin{frame}{Architectural Choices}
\begin{itemize}
\item Centralized fusion aggregates every measurement before computing an optimal estimate but demands heavy comms.
\item Decentralized setups split the system into several centralized islands, inheriting similar bandwidth costs.
\item Distributed approaches process locally first, scaling better at the expense of globally optimal performance.
\end{itemize}
\end{frame}
% --- Original text start ---
% Architectural taxonomy section on centralized, decentralized, and distributed fusion architectures.
% --- Original text end ---

\begin{frame}{Bayesian Sensor Fusion}
\begin{itemize}
\item Model unknown quantities as random variables and capture knowledge in probability distributions.
\item Bayes' rule blends heterogeneous measurements while propagating uncertainty.
\item Probabilistic framing naturally handles missing data and classification decisions.
\end{itemize}
\end{frame}
% --- Original text start ---
% Bayesian approach to sensor fusion section describing reasons for probabilistic modeling.
% --- Original text end ---

\begin{frame}{Competitive Fusion Example}
\begin{itemize}
\item Two conditionally independent Gaussian sensors yield a joint Gaussian measurement.
\item The fused estimate weights each sensor by its variance and lowers total uncertainty.
\end{itemize}
\pause
\begin{equation*}
\mu = \frac{y_1\sigma_2^2 + y_2\sigma_1^2}{\sigma_1^2 + \sigma_2^2}, \quad \sigma = \frac{\sigma_1^2 \sigma_2^2}{\sigma_1^2 + \sigma_2^2}
\end{equation*}
\end{frame}
% --- Original text start ---
% Example \ref{ex:competitive} showing fused Gaussian measurement and resulting mean/variance.
% --- Original text end ---

\begin{frame}{Kalman Filter for Fusion}
\begin{itemize}
\item Linear dynamics $\x_t = A_t \x_{t-1} + B_t \bu_t + \bm{\epsilon}_t$ and linear measurements $\z_t = C_t \x_t + \bm{\delta}_t$ fit multi-sensor data.
\item The filter consumes every measurement at time $t$ simultaneously via the correction step.
\item Covariance matrices $\bm{R}_t$ and $\bm{Q}_t$ bias updates toward higher-precision sensors through the Kalman gain.
\end{itemize}
\end{frame}
% --- Original text start ---
% Kalman filter sensor fusion subsection with dynamics/measurement equations and discussion of covariance weighting.
% --- Original text end ---

\begin{frame}{State Augmentation Tricks}
\begin{itemize}
\item The state vector can incorporate biases, offsets, or health indicators alongside kinematics.
\item Augmenting the state lets the EKF track auxiliary quantities with the same predict--correct loop.
\item Capturing these latents improves downstream fusion accuracy and resiliency.
\end{itemize}
\end{frame}
% --- Original text start ---
% Paragraph noting that \x may include auxiliary states such as sensor bias or health information.
% --- Original text end ---

\begin{frame}{Self-Driving Car Example}
\begin{itemize}
\item State $[p, v, a]^\top$ follows a discretized double-integrator with process noise $\bm{\epsilon}_t$.
\item Lidar and GNSS both observe position while the IMU senses acceleration, forming the measurement matrix.
\item Sensor-specific variances populate $\bm{Q}_t$, letting the Kalman gain favor GNSS when it is more precise.
\end{itemize}
\pause
\begin{equation*}
\begin{bmatrix}
z_{\text{lidar}} \\
z_{\text{gnss}} \\
z_{\text{imu}}
\end{bmatrix}
=
\begin{bmatrix}
1 & 0 & 0 \\
1 & 0 & 0 \\
0 & 0 & 1
\end{bmatrix}
\begin{bmatrix}
p \\
v \\
a
\end{bmatrix}
+ \bm{\delta}_t
\end{equation*}
\end{frame}
% --- Original text start ---
% Example \ref{ex:car} describing longitudinal vehicle model, measurement matrix, and covariance choices.
% --- Original text end ---

\begin{frame}{Fusion Benefit Visualization}
\begin{itemize}
\item Adding the lower-variance GNSS signal sharpens the fused position estimate.
\item Even noisier sensors can help, though their impact is smaller.
\item Simulation underscores how the Kalman filter mixes temporal dynamics with cross-sensor data.
\end{itemize}
\pause
\begin{center}
    \includegraphics[width=0.8\linewidth]{../book/figs/ch18_figs/carKFfusion.png}
\end{center}
\end{frame}
% --- Original text start ---
% Figure \ref{fig:fusionexample} comparing lidar-only vs. lidar+GNSS+IMU position estimates.
% --- Original text end ---

\begin{frame}{Fusion Challenges}
\begin{itemize}
\item Registration mismatches in time or space must be corrected before fusing.
\item Biases creep in during coordinate transforms and require calibration or estimation.
\item Sensor correlations, data association, and out-of-sequence packets complicate inference.
\item Multi-agent settings intensify communication constraints that drive these issues.
\end{itemize}
\end{frame}
% --- Original text start ---
% Challenges section covering registration, bias, correlation, data association, and out-of-sequence measurements.
% --- Original text end ---

\begin{frame}{Reminder: Sensor Fusion Toolbox}
\begingroup
\setbeamercolor{block title}{bg=gray!20,fg=black}
\setbeamercolor{block body}{bg=gray!10,fg=black}
\begin{block}{Key Takeaways}
\begin{itemize}
\item Use probabilistic models (Kalman or beyond) to combine heterogeneous sensors coherently.
\item Competitive/complementary/cooperative sensing patterns dictate reliability vs. coverage trade-offs.
\item Address registration, bias, and association early to unlock the promised uncertainty reduction.
\end{itemize}
\end{block}
\endgroup
\end{frame}
% --- Original text start ---
% Chapter summary themes on Bayesian framing, taxonomy of fusion strategies, and practical challenges.
% --- Original text end ---

\end{document}
