\documentclass{beamer}

% Shared Beamer configuration for Principles of Robot Autonomy decks
\usetheme{Boadilla}
%\usecolortheme{seagull}

% Use Palatino to match the book's Tufte style
\usepackage{mathpazo}
\usefonttheme{serif}

\usepackage{amsmath}
\usepackage{amssymb}
\usepackage{bm}
\usepackage{graphicx}
\usepackage{enumitem}
\usepackage{xparse}
\usepackage{array}

% Increase vertical spacing between list items for readability.
\setlist[itemize]{itemsep=0.75em}
\setlist[enumerate]{itemsep=0.75em}

% Define bold symbol macros if not already defined
\providecommand{\x}{\boldsymbol{x}}
\providecommand{\z}{\boldsymbol{z}}
\providecommand{\q}{\boldsymbol{q}}
\providecommand{\bxi}{\boldsymbol{\xi}}
\providecommand{\bu}{\boldsymbol{u}}
\providecommand{\bmu}{\boldsymbol{\mu}}
\providecommand{\bSigma}{\boldsymbol{\Sigma}}
\providecommand{\y}{\boldsymbol{y}}
\providecommand{\m}{\mathcal{M}}
\providecommand{\bc}{\boldsymbol{c}}
\providecommand{\R}{\mathbb{R}}

% Animated lists by default, with override via optional overlay argument.
\beamerdefaultoverlayspecification{<+->}

% Paragraphs reveal sequentially by default.
\AtBeginDocument{%
  \ifx\paragraph\undefined
    \newcommand{\paragraph}[1]{\textbf{#1}\par\onslide<+->{} }
  \else
    \let\origparagraph\paragraph
    \renewcommand{\paragraph}[1]{\origparagraph{#1}\par\onslide<+->{} }
  \fi
}

% Add other shared packages or theme settings above this line as needed.

\author[COMP 4766]{Adapted from ``Principles of Robot Autonomy'' by D. Gammelli, J. Lorenzetti,
K. Luo, G. Zardini, and M. Pavone}
\date{\today}


\title{Chapter 25 - Formal Methods}

\begin{document}

\frame{\titlepage}

\begin{frame}{Why Formal Guarantees?}
\begin{itemize}
\item Robots now act in open, interactive settings where safety and explainability are non-negotiable.
\item Testing alone struggles to capture every scenario or success criterion the robot must respect.
\item Formal methods target provably correct autonomy: requirements, proofs, and even correct-by-construction synthesis.
\end{itemize}
\end{frame}
% --- Original text start ---
% Intro paragraphs motivating safe robotics, need for provable correctness, definition of formal methods.
% --- Original text end ---

\begin{frame}{What Are Formal Methods?}
\begin{itemize}
\item Mathematically based techniques for specification, development, and verification of hardware/software.
\item Encompass languages for requirements, tools for verification/model checking, and synthesis procedures.
\item This chapter focuses on LTL specifications, reactive-system verification, and reactive synthesis.
\end{itemize}
\end{frame}
% --- Original text start ---
% Definition and scope of formal methods, chapter roadmap toward LTL, verification, synthesis.
% --- Original text end ---

\begin{frame}{LTL Language Basics}
\begin{itemize}
\item Extends propositional logic (Booleans, symbols, \(\lnot, \land, \lor, \to\)) with temporal context over discrete time.
\item Propositions capture atomic facts (e.g., region visited, button pressed) evaluated at each time step.
\item Specifications become formulas \(\psi\) built by composing these atoms to describe desired sequences.
\end{itemize}
\end{frame}
% --- Original text start ---
% Section on LTL operators inherited from propositional logic and their meaning.
% --- Original text end ---

\begin{frame}{Temporal Operators}
\begin{itemize}
\item \(X \psi\): $\psi$ must hold on the next time step; \(\psi_1 U \psi_2\): keep \(\psi_1\) true until \(\psi_2\) occurs.
\item Eventually and always derived as \(F \psi = \text{true} \: U \: \psi\) and \(G \psi = \lnot F\, \lnot \psi\).
\item Rich behaviors emerge by composing operators to encode timing, ordering, and persistence constraints.
\end{itemize}
\begin{center}
    \includegraphics[width=0.55\linewidth]{../tex/figs/ch25_figs/next.png}
\end{center}
\end{frame}
% --- Original text start ---
% Temporal operators (next, until, eventually, always) and associated figures.
% --- Original text end ---

\begin{frame}{Composing Specifications}
\begin{itemize}
\item Infinitely-often: \(GF \psi\) ensures $\psi$ recurs forever; stability: \(FG \psi\) locks $\psi$ after some time.
\item Response templates \(G(\psi_1 \to F \psi_2)\) trigger follow-up actions after events.
\item LTL is a problem-formulation language—precise specs precede any algorithmic solution.
\end{itemize}
\end{frame}
% --- Original text start ---
% Discussion of GF, FG, response compositions and LTL as specification language.
% --- Original text end ---

\begin{frame}{Example: Coffee Robot}
\begin{itemize}
\item Button press should start two grind cycles followed by two brew cycles.
\item Wrap the requirement in \(G\) so it applies whenever the button is pressed.
\item Highlights how discrete-time sequencing maps naturally into LTL syntax.
\end{itemize}
\[
\phi = G\big( \text{button} \to \text{grind} \land X\text{grind} \land XX(\lnot\text{grind} \land \text{brew}) \land XXX(\lnot\text{grind} \land \text{brew}) \big)
\]
\end{frame}
% --- Original text start ---
% Coffee machine example translating informal requirement to LTL.
% --- Original text end ---

\begin{frame}{Verification Workflow}
\begin{itemize}
\item Treat the robot (or component) as model \(P\) with inputs \(\hat{i}\) and outputs \(\hat{o}\).
\item Model checking answers whether \(P \models \phi\) for all admissible input sequences.
\item Success yields provable guarantees; failure surfaces counterexamples to refine design/spec.
\end{itemize}
\begin{center}
    \includegraphics[width=0.6\linewidth]{../tex/figs/ch25_figs/verification.png}
\end{center}
\end{frame}
% --- Original text start ---
% Verification section introducing P models phi notation and Figure verification.
% --- Original text end ---

\begin{frame}{Reactive Models}
\begin{itemize}
\item Inputs represent environment signals; outputs encode the robot’s decisions or actuator commands.
\item Specifications are written over the combined input/output traces \(\hat{i} \cup \hat{o}\).
\item Goal: guarantee desired behavior under \emph{all} admissible input sequences.
\end{itemize}
\begin{center}
    \includegraphics[width=0.55\linewidth]{../tex/figs/ch25_figs/model.png}
\end{center}
\end{frame}
% --- Original text start ---
% Reactive model explanation with margin figure depicting inputs and outputs.
% --- Original text end ---

\begin{frame}{Reactive Synthesis}
\begin{itemize}
\item Instead of checking an existing design, build a model that is correct-by-construction.
\item Synthesis consumes \(\phi\) and searches for a finite-state strategy that satisfies it under all inputs.
\item Requires specification realizability; otherwise no controller can enforce the behavior.
\end{itemize}
\begin{center}
    \includegraphics[width=0.58\linewidth]{../tex/figs/ch25_figs/synthesis.png}
\end{center}
\end{frame}
% --- Original text start ---
% Reactive synthesis section and Figure synthesis overview.
% --- Original text end ---

\begin{frame}{Satisfiable vs Realizable}
\begin{itemize}
\item Satisfiable: there exists some input/output pair that meets \(\phi\); useful for feasibility checks.
\item Realizable: for all inputs there exists an output trace satisfying \(\phi\); crucial for safety guarantees.
\item Coffee robot spec is satisfiable but not realizable—two rapid button presses create conflicting grind demands.
\end{itemize}
\end{frame}
% --- Original text start ---
% Definitions of satisfiability and realizability plus coffee machine realization example.
% --- Original text end ---

\begin{frame}{Two-Player Game Formulation}
\begin{itemize}
\item Environment chooses among \(2^{|I|}\) inputs; system chooses among \(2^{|O|}\) outputs each step.
\item Strategy \(f:(2^{|I|})^* \to 2^{|O|}\) maps input histories to actions; winning if \(\hat{i} \cup \hat{o} \models \phi\).
\item Construct game by translating \(\phi\) into a B"{u}chi automaton, then solving for a winning strategy.
\end{itemize}
\begin{center}
    \includegraphics[width=0.5\linewidth]{../tex/figs/ch25_figs/game.png}
\end{center}
\end{frame}
% --- Original text start ---
% Description of synthesis game components and Figure game derived from automaton.
% --- Original text end ---

\begin{frame}{Example Strategy: Always Grant}
\begin{itemize}
\item Spec \(G(r \to X g)\) demands each request be granted next step.
\item Derived game shows contradiction state if a request goes unfulfilled.
\item A simple winning strategy is to grant every cycle, avoiding contradiction entirely.
\end{itemize}
\begin{center}
    \includegraphics[width=0.45\linewidth]{../tex/figs/ch25_figs/game_solution.png}
\end{center}
\end{frame}
% --- Original text start ---
% Reactive synthesis example with automaton/game depiction and discussion of always-grant strategy.
% --- Original text end ---

\begin{frame}{Reminder: Formal Methods Habits}
\begingroup
\setbeamercolor{block title}{bg=gray!20,fg=black}
\setbeamercolor{block body}{bg=gray!10,fg=black}
\begin{block}{Key Takeaways}
\begin{itemize}
\item Capture requirements precisely with LTL before coding controllers or planners.
\item Use verification to prove \(P \models \phi\) and diagnose gaps through counterexamples.
\item When feasible, leverage synthesis to build controllers that are correct-by-construction.
\end{itemize}
\end{block}
\endgroup
\end{frame}
% --- Original text start ---
% Chapter summary emphasizing LTL specs, verification, and synthesis-driven design.
% --- Original text end ---

\end{document}
