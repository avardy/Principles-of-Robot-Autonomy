\documentclass{beamer}

% Shared Beamer configuration for Principles of Robot Autonomy decks
\usetheme{Boadilla}
%\usecolortheme{seagull}

% Use Palatino to match the book's Tufte style
\usepackage{mathpazo}
\usefonttheme{serif}

\usepackage{amsmath}
\usepackage{amssymb}
\usepackage{bm}
\usepackage{graphicx}
\usepackage{enumitem}
\usepackage{xparse}
\usepackage{array}

% Increase vertical spacing between list items for readability.
\setlist[itemize]{itemsep=0.75em}
\setlist[enumerate]{itemsep=0.75em}

% Define bold symbol macros if not already defined
\providecommand{\x}{\boldsymbol{x}}
\providecommand{\z}{\boldsymbol{z}}
\providecommand{\q}{\boldsymbol{q}}
\providecommand{\bxi}{\boldsymbol{\xi}}
\providecommand{\bu}{\boldsymbol{u}}
\providecommand{\bmu}{\boldsymbol{\mu}}
\providecommand{\bSigma}{\boldsymbol{\Sigma}}
\providecommand{\y}{\boldsymbol{y}}
\providecommand{\m}{\mathcal{M}}
\providecommand{\bc}{\boldsymbol{c}}
\providecommand{\R}{\mathbb{R}}

% Animated lists by default, with override via optional overlay argument.
\beamerdefaultoverlayspecification{<+->}

% Paragraphs reveal sequentially by default.
\AtBeginDocument{%
  \ifx\paragraph\undefined
    \newcommand{\paragraph}[1]{\textbf{#1}\par\onslide<+->{} }
  \else
    \let\origparagraph\paragraph
    \renewcommand{\paragraph}[1]{\origparagraph{#1}\par\onslide<+->{} }
  \fi
}

% Add other shared packages or theme settings above this line as needed.

\author[COMP 4766]{Adapted from ``Principles of Robot Autonomy'' by D. Gammelli, J. Lorenzetti,
K. Luo, G. Zardini, and M. Pavone}
\date{\today}


\title{Chapter 11 - Information Extraction}

\begin{document}

\frame{\titlepage}

\begin{frame}{From Features to Semantics}
\begin{itemize}
\item Chapter 10 handled local filters; now we extract higher-level structure and object labels.
\item Geometric primitives (lines, planes) support mapping and localization across many sensors.
\item Object recognition pipelines classify discrete entities for planning and decision making.
\end{itemize}
\end{frame}
% --- Original text start ---
% Chapter introduction on geometric feature extraction and object recognition motivation.
% --- Original text end ---

\begin{frame}{Geometric Feature Extraction}
\begin{itemize}
\item Represent environments via simple primitives fit to range or vision data.
\item Two subproblems: segmentation (which points) and fitting (parameter estimation).
\item Focus here on 2D line extraction; other primitives follow similar recipes.
\end{itemize}
\end{frame}
% --- Original text start ---
% Section introduction: segmentation vs fitting for lines.
% --- Original text end ---

\begin{frame}{Split-and-Merge Segmentation}
\begin{itemize}
\item Fit a line to a set; if any point exceeds threshold $d$, split at that point.
\item Repeat until all subsets satisfy the distance bound, then merge colinear neighbors.
\item Iterative-end-point-fit variant simply connects subset endpoints for speed.
\end{itemize}
\pause
\begin{center}
    \includegraphics[width=0.75\linewidth]{../tex/figs/ch11_figs/iterative_end_point_fit.png}
\end{center}
\end{frame}
% --- Original text start ---
% Algorithm \ref{alg:splitmerge} and Figure \ref{fig:splitmerge}.
% --- Original text end ---

\begin{frame}{RANSAC for Lines}
\begin{itemize}
\item Randomly sample minimal sets, fit candidate lines, score inliers within distance $d$.
\item Keep the hypothesis with the most inliers; repeat to boost success probability.
\item Robust to outliers, but run sequentially when multiple lines are needed.
\end{itemize}
\pause
\begin{center}
    \includegraphics[width=0.75\linewidth]{../tex/figs/ch11_figs/ransac.png}
\end{center}
\end{frame}
% --- Original text start ---
% Algorithm \ref{alg:ransac} description and Figure \ref{fig:ransac-working}.
% --- Original text end ---

\begin{frame}{RANSAC Iteration Budget}
\begin{itemize}
\item Let $w$ be probability of drawing an inlier; two-inlier sample occurs with $w^2$.
\item Probability RANSAC never sees a clean sample in $k$ trials is $(1-w^2)^k$.
\item Solve for $k$ to reach success probability $p$ instead of brute-force $\frac{N(N-1)}{2}$.
\end{itemize}
\pause
\begin{equation*}
\bar{k} = \frac{\log(1-p)}{\log(1-w^2)}
\end{equation*}
\end{frame}
% --- Original text start ---
% Statistical analysis leading to Eq. \eqref{eq:magic-k}.
% --- Original text end ---

\begin{frame}{Hough Transform Intuition}
\begin{itemize}
\item Each point $(x_i,y_i)$ votes for all line parameters passing through it.
\item In $(m,b)$ space, original points map to lines; intersections reveal candidate lines.
\item Polar form $(\alpha,r)$ avoids infinite slopes and offers uniform grids.
\end{itemize}
\pause
\begin{center}
    \includegraphics[width=0.7\linewidth]{../tex/figs/ch11_figs/hough.png}
\end{center}
\end{frame}
% --- Original text start ---
% Figures \ref{fig:hough1}-\ref{fig:hough4} on Hough transform mappings.
% --- Original text end ---

\begin{frame}{Polar Hough Space}
\begin{itemize}
\item Line representation: $x\cos \alpha + y\sin \alpha = r$.
\item Single data point maps to sinusoid in $(\alpha,r)$; intersections mark consistent lines.
\item Voting via discretized accumulator; pick local maxima as detected lines.
\end{itemize}
\pause
\begin{center}
    \includegraphics[width=0.7\linewidth]{../tex/figs/ch11_figs/hough4.png}
\end{center}
\end{frame}
% --- Original text start ---
% Polar mapping explanation with Figures \ref{fig:hough3}, \ref{fig:hough4}.
% --- Original text end ---

\begin{frame}{Line Fitting via Least Squares}
\begin{itemize}
\item Once segments are known, estimate line parameters $(\alpha,r)$ minimizing perpendicular error.
\item Work in polar coords: $d_i = \rho_i \cos(\theta_i - \alpha) - r$.
\item Minimize $S(r,\alpha) = \sum d_i^2$ to get closed-form solutions.
\end{itemize}
\pause
\begin{center}
    \includegraphics[width=0.5\linewidth]{../tex/figs/ch11_figs/polarlinefit.png}
\end{center}
\end{frame}
% --- Original text start ---
% Equations \eqref{eq:polarline}, \eqref{eq:lineerror}, \eqref{eq:MSL} and Figure \ref{fig:polarline}.
% --- Original text end ---

\begin{frame}{Weighted Least Squares}
\begin{itemize}
\item Handle heterogeneous noise with weights $w_i = 1/\sigma_i^2$.
\item Cost: $S_w = \sum w_i (\rho_i \cos(\theta_i - \alpha) - r)^2$.
\item Closed-form expressions adjust $r$ and $\alpha$ to favor precise measurements.
\end{itemize}
\end{frame}
% --- Original text start ---
% Weighted formulation \eqref{eq:WMSL} and resulting expressions.
% --- Original text end ---

\begin{frame}{Object Recognition Tasks}
\begin{itemize}
\item Move from geometric primitives to semantic labels (people, cars, tools).
\item Challenges: pose variation, intra-class diversity, occlusion, clutter.
\item Common families: template matching, bag-of-visual-words, convolutional nets.
\end{itemize}
\end{frame}
% --- Original text start ---
% Object recognition overview and methods list.
% --- Original text end ---

\begin{frame}{Template Matching}
\begin{itemize}
\item Cross-correlate template $T$ across image $I$; high responses indicate matches.
\item SAD/SSD windows from Chapter 10 provide alternative similarity measures.
\item Image pyramids mitigate scale changes: detect coarsely, refine at higher resolution.
\end{itemize}
\pause
\begin{center}
    \includegraphics[width=0.55\linewidth]{../tex/figs/ch11_figs/image_pyramid.png}
\end{center}
\end{frame}
% --- Original text start ---
% Template matching, SAD, and pyramid discussion with Figure \ref{image-pyramid}.
% --- Original text end ---

\begin{frame}{Bag of Visual Words}
\begin{itemize}
\item Treat objects as unordered collections of part-level "visual words".
\item Build histograms over detected words; classify via similarity to training distributions.
\item Works across categories where layout varies but part presence matters.
\end{itemize}
\end{frame}
% --- Original text start ---
% Bag of visual words section describing histogram-based recognition.
% --- Original text end ---

\begin{frame}{Convolutional Neural Networks}
\begin{itemize}
\item CNNs learn hierarchical filters end-to-end for recognition and detection tasks.
\item Convolutions + pooling handle spatial locality and translation variance.
\item Modern robotics stacks rely on CNN backbones for perception modules.
\end{itemize}
\end{frame}
% --- Original text start ---
% CNN overview and historical context.
% --- Original text end ---

\begin{frame}{Exercises}
\begin{itemize}
\item HW3 Problem 2: implement Split-and-Merge for Lidar line extraction.
\item HW3 Problem 4: practice template matching with OpenCV.
\item Extra Problem: study Gaussian pyramids for multiscale matching.
\end{itemize}
\end{frame}
% --- Original text start ---
% Exercises list referencing AA274A_HW3 repository problems.
% --- Original text end ---

\begin{frame}{Reminder: Extraction Checklist}
\begingroup
\setbeamercolor{block title}{bg=gray!20,fg=black}
\setbeamercolor{block body}{bg=gray!10,fg=black}
\begin{block}{Key Takeaways}
\begin{itemize}
\item Combine segmentation (Split-Merge, RANSAC, Hough) with least-squares fitting for clean maps.
\item Tune thresholds $d$, iteration counts $k$, and accumulator grids to balance speed/robustness.
\item Template/CNN pipelines need multiscale reasoning; reuse pyramids and descriptors from prior chapters.
\end{itemize}
\end{block}
\endgroup
\end{frame}
% --- Original text start ---
% Summary reminder synthesizing segmentation, fitting, and recognition lessons.
% --- Original text end ---

\end{document}
