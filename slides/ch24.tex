\documentclass{beamer}

% Shared Beamer configuration for Principles of Robot Autonomy decks
\usetheme{Boadilla}
%\usecolortheme{seagull}

% Use Palatino to match the book's Tufte style
\usepackage{mathpazo}
\usefonttheme{serif}

\usepackage{amsmath}
\usepackage{amssymb}
\usepackage{bm}
\usepackage{graphicx}
\usepackage{enumitem}
\usepackage{xparse}
\usepackage{array}

% Increase vertical spacing between list items for readability.
\setlist[itemize]{itemsep=0.75em}
\setlist[enumerate]{itemsep=0.75em}

% Define bold symbol macros if not already defined
\providecommand{\x}{\boldsymbol{x}}
\providecommand{\z}{\boldsymbol{z}}
\providecommand{\q}{\boldsymbol{q}}
\providecommand{\bxi}{\boldsymbol{\xi}}
\providecommand{\bu}{\boldsymbol{u}}
\providecommand{\bmu}{\boldsymbol{\mu}}
\providecommand{\bSigma}{\boldsymbol{\Sigma}}
\providecommand{\y}{\boldsymbol{y}}
\providecommand{\m}{\mathcal{M}}
\providecommand{\bc}{\boldsymbol{c}}
\providecommand{\R}{\mathbb{R}}

% Animated lists by default, with override via optional overlay argument.
\beamerdefaultoverlayspecification{<+->}

% Paragraphs reveal sequentially by default.
\AtBeginDocument{%
  \ifx\paragraph\undefined
    \newcommand{\paragraph}[1]{\textbf{#1}\par\onslide<+->{} }
  \else
    \let\origparagraph\paragraph
    \renewcommand{\paragraph}[1]{\origparagraph{#1}\par\onslide<+->{} }
  \fi
}

% Add other shared packages or theme settings above this line as needed.

\author[COMP 4766]{Adapted from ``Principles of Robot Autonomy'' by D. Gammelli, J. Lorenzetti,
K. Luo, G. Zardini, and M. Pavone}
\date{\today}


\title{Chapter 24 - Robot Operating System}

\begin{document}

\frame{\titlepage}

\begin{frame}{Why ROS Middleware?}
\begin{itemize}
\item Robot software must coordinate sensors, actuators, and decision loops across heterogeneous hardware.
\item ROS supplies shared tooling, libraries, and conventions so teams can reuse work instead of rebuilding infrastructure.
\item Treat ROS as a robotics-focused middleware layer, not a replacement operating system.
\end{itemize}
\end{frame}
% --- Original text start ---
% This chapter introduces the fundamentals of the Robot Operating System (ROS)... ROS is a "middleware" ... simplified and consistent manner across a wide variety of robotic platforms.
% --- Original text end ---

\begin{frame}{Robot Programming Challenges}
\begin{itemize}
\item Multitask across many sensor/actuator threads that must exchange data safely.
\item Provide low-level device access (GPIO, USB, SPI) while supporting C/C++/Python APIs.
\item Encourage modular OOP design and leverage rich third-party ecosystems plus community support.
\end{itemize}
\end{frame}
% --- Original text start ---
% Robot programming ... manage many different individual hardware components ... Therefore, the following features are needed ... multitasking, low level device control, high level OOP, third-party libraries.
% --- Original text end ---

\begin{frame}{Origins of ROS}
\begin{itemize}
\item 2007 STAIR project envisioned an open, sharable robotics stack spanning hardware to libraries.
\item Willow Garage scaled the effort, releasing PR2 platforms and ROS 0.4--1.0 between 2009 and 2010.
\item OSRF now stewards ROS with LTS distributions (e.g., Indigo) and industry-wide adoption.
\end{itemize}
\end{frame}
% --- Original text start ---
% Brief history of ROS ... STAIR project ... Willow Garage ... PR2 ... OSRF ... LTS releases ...
% --- Original text end ---

\begin{frame}{Guiding Principles}
\begin{itemize}
\item Keep the platform free/open-source to maximize collaboration.
\item Ship reusable hardware interfaces and software packages ready for new projects.
\item Integrate smoothly with existing frameworks such as OpenCV, SLAM stacks, and Gazebo.
\end{itemize}
\end{frame}
% --- Original text start ---
% STAIR vision bullet list describing free/open, core components available, and seamless integration.
% --- Original text end ---

\begin{frame}{Core ROS Capabilities}
\begin{itemize}
\item Modularity: encapsulate functionality into nodes and deployable packages.
\item Message passing: standard pub/sub plumbing connects perception, planning, and control.
\item Built-in algorithms and thriving community shorten prototyping time for complex robots.
\end{itemize}
\end{frame}
% --- Original text start ---
% Characteristics of ROS ... modularity, message passing, built-in algorithms, third-party libraries/community.
% --- Original text end ---

\begin{frame}{Node and Package Modularity}
\begin{itemize}
\item Nodes implement focused behaviors that can be developed, tested, and reused independently.
\item Packages bundle nodes plus config/data so capabilities ship as installable units.
\item Modular decomposition keeps sprawling robots manageable.
\end{itemize}
\begin{center}
    \includegraphics[width=0.68\linewidth]{../tex/figs/ch24_figs/ROS_modular.png}
\end{center}
\end{frame}
% --- Original text start ---
% Modularity description referencing Figure modularity and nodes/packages discussion.
% --- Original text end ---

\begin{frame}{ROS Communication Model}
\begin{itemize}
\item Nodes publish or subscribe to typed messages that flow through named topics.
\item The master process keeps track of publishers/subscribers so peer-to-peer links can be formed.
\item Pub/sub decouples producers from consumers, improving scalability across machines.
\end{itemize}
\end{frame}
% --- Original text start ---
% Robot Programming with ROS ... pub/sub structure ... nodes, messages, topics, master definitions.
% --- Original text end ---

\begin{frame}{Publish/Subscribe Graph}
\begin{itemize}
\item Visualize ROS as chat rooms where sensing, perception, and planning modules exchange data.
\item Each topic enforces a single message type, so interfaces stay explicit.
\item Adding nodes simply means joining the right chat rooms.
\end{itemize}
\begin{center}
    \includegraphics[width=0.6\linewidth]{../tex/figs/ch24_figs/pub_sub.png}
\end{center}
\end{frame}
% --- Original text start ---
% Figure showing pub/sub model and description of chat room analogy.
% --- Original text end ---

\begin{frame}{Nodes in Practice}
\begin{itemize}
\item Treat every capability (camera, edge detection, path planning) as a separate node.
\item Nodes can publish sensor products, consume higher-level cues, or both.
\item Reuse nodes across robots by exposing clean topic interfaces.
\end{itemize}
\end{frame}
% --- Original text start ---
% Node definition plus example with camera, edge_detection, path_planning nodes.
% --- Original text end ---

\begin{frame}{Messages and Types}
\begin{itemize}
\item Messages are simple data structures with typed fields (primitive, arrays, or custom types).
\item Chain complex data by nesting types (e.g., SensorPacket with arrays of SensorData).
\item ROS ships standard primitives (int32, float64, time, etc.) for cross-language safety.
\end{itemize}
\end{frame}
% --- Original text start ---
% Definition of messages, example definitions, and Table of built-in message types.
% --- Original text end ---

\begin{frame}{Topics and rostopic Tools}
\begin{itemize}
\item Topics are named buses for one-way streaming; publishers and subscribers must match message types.
\item Use \texttt{rostopic list/echo/hz} to inspect active traffic and diagnose timing issues.
\item Reserve services or actions for request-response interactions that topics cannot express cleanly.
\end{itemize}
\end{frame}
% --- Original text start ---
% Topics definition plus rostopic commands list and discussion of unidirectional streaming.
% --- Original text end ---

\begin{frame}{ROS Master Role}
\begin{itemize}
\item Single master assigns network addresses and helps nodes locate each other.
\item Runs anywhere on the network; after discovery, nodes communicate peer-to-peer.
\item Keep only one master alive at a time to avoid conflicting registries.
\end{itemize}
\end{frame}
% --- Original text start ---
% Master definition, responsibilities, remote hosting, single master per system.
% --- Original text end ---

\begin{frame}{Publisher Node Pattern}
\begin{itemize}
\item Initialize the node, advertise a topic with \texttt{rospy.Publisher}, and honor queue limits.
\item Enforce loop timing with \texttt{rospy.Rate} and guard with \texttt{rospy.is\_shutdown()}.
\item Log to \texttt{rosout} (\texttt{rospy.loginfo}) so console output is aggregated for debugging.
\end{itemize}
\end{frame}
% --- Original text start ---
% talker.py walkthrough covering init_node, Publisher, Rate, loginfo, loop publishing strings.
% --- Original text end ---

\begin{frame}{Subscriber Node Pattern}
\begin{itemize}
\item Register callbacks via \texttt{rospy.Subscriber} to process messages asynchronously.
\item Use \texttt{rospy.spin()} to keep the process alive while callbacks handle incoming data.
\item Mirror talker/listener structure to test pub/sub wiring quickly.
\end{itemize}
\end{frame}
% --- Original text start ---
% listener.py example describing callback, Subscriber, spin.
% --- Original text end ---

\begin{frame}{Launch Files}
\begin{itemize}
\item \texttt{.launch} XML documents start masters, nodes, and parameter settings in one shot.
\item Use the \texttt{<node name=... pkg=... type=...>} tag to override executable names and supply args.
\item Organize launch files inside each package's \texttt{launch/} folder for reuse.
\end{itemize}
\end{frame}
% --- Original text start ---
% Launch file definition, syntax examples with attributes like args and respawn.
% --- Original text end ---

\begin{frame}{Catkin Workspace}
\begin{itemize}
\item \texttt{catkin} builds interdependent ROS packages via \texttt{catkin\_make}.
\item Standard workspace structure (src, build, devel) plus auto-generated \texttt{CMakeLists.txt} and \texttt{package.xml}.
\item Rebuild whenever you add packages or change message/service definitions.
\end{itemize}
\begin{center}
    \includegraphics[width=0.7\linewidth]{../tex/figs/ch24_figs/catkin_ws.png}
\end{center}
\end{frame}
% --- Original text start ---
% catkin workspace description and Figure catkin.
% --- Original text end ---

\begin{frame}{Debugging Toolkit}
\begin{itemize}
\item \texttt{rostopic} introspects live streams; \texttt{rospy.loginfo} feeds \texttt{rqt\_console}.
\item \texttt{rosbag} records/replays topics for deterministic regression testing.
\item Fall back to Python's \texttt{pdb} when you need step-by-step inspection.
\end{itemize}
\end{frame}
% --- Original text start ---
% Debugging section listing rostopic, rospy.loginfo, rosbag, pdb.
% --- Original text end ---

\begin{frame}{Gazebo Simulation}
\begin{itemize}
\item Gazebo delivers high-fidelity physics, sensor models, and realistic scenes for testing.
\item Use \texttt{gazebo\_ros\_pkgs} to bridge simulated robots with ROS nodes seamlessly.
\item Deploy simulation throughout development to validate algorithms before hardware trials.
\end{itemize}
\begin{center}
    \includegraphics[width=0.65\linewidth]{../tex/figs/ch24_figs/Gazebo.PNG}
\end{center}
\end{frame}
% --- Original text start ---
% Gazebo description and figure reference.
% --- Original text end ---

\begin{frame}{Reminder: ROS Habits}
\begingroup
\setbeamercolor{block title}{bg=gray!20,fg=black}
\setbeamercolor{block body}{bg=gray!10,fg=black}
\begin{block}{Key Takeaways}
\begin{itemize}
\item Modularize functionality into nodes/packages and connect them through typed topics.
\item Leverage launch files, catkin, and tooling to manage growing deployments.
\item Prototype and debug in simulation (Gazebo, rosbag) before rolling onto hardware.
\end{itemize}
\end{block}
\endgroup
\end{frame}
% --- Original text start ---
% Chapter summary emphasizing ROS modularity, tooling, and simulation/development practices.
% --- Original text end ---

\end{document}
