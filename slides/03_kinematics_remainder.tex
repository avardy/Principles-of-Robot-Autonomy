\documentclass{beamer}

% Shared Beamer configuration for Principles of Robot Autonomy decks
\usetheme{Boadilla}
%\usecolortheme{seagull}

% Use Palatino to match the book's Tufte style
\usepackage{mathpazo}
\usefonttheme{serif}

\usepackage{amsmath}
\usepackage{amssymb}
\usepackage{bm}
\usepackage{graphicx}
\usepackage{enumitem}
\usepackage{xparse}
\usepackage{array}

% Increase vertical spacing between list items for readability.
\setlist[itemize]{itemsep=0.75em}
\setlist[enumerate]{itemsep=0.75em}

% Define bold symbol macros if not already defined
\providecommand{\x}{\boldsymbol{x}}
\providecommand{\z}{\boldsymbol{z}}
\providecommand{\q}{\boldsymbol{q}}
\providecommand{\bxi}{\boldsymbol{\xi}}
\providecommand{\bu}{\boldsymbol{u}}
\providecommand{\bmu}{\boldsymbol{\mu}}
\providecommand{\bSigma}{\boldsymbol{\Sigma}}
\providecommand{\y}{\boldsymbol{y}}
\providecommand{\m}{\mathcal{M}}
\providecommand{\bc}{\boldsymbol{c}}
\providecommand{\R}{\mathbb{R}}

% Animated lists by default, with override via optional overlay argument.
\beamerdefaultoverlayspecification{<+->}

% Paragraphs reveal sequentially by default.
\AtBeginDocument{%
  \ifx\paragraph\undefined
    \newcommand{\paragraph}[1]{\textbf{#1}\par\onslide<+->{} }
  \else
    \let\origparagraph\paragraph
    \renewcommand{\paragraph}[1]{\origparagraph{#1}\par\onslide<+->{} }
  \fi
}

% Add other shared packages or theme settings above this line as needed.

\author[COMP 4766]{Adapted from ``Principles of Robot Autonomy'' by D. Gammelli, J. Lorenzetti,
K. Luo, G. Zardini, and M. Pavone}
\date{\today}


\title{Mobile Robot Kinematics: Part 1}

\begin{document}

\begin{frame}[plain]
\titlepage
\end{frame}

\begin{frame}{Source Material}
    Covering the following sections from ``Principles of Robot Autonomy'':
    \begin{itemize}
    \item 1.3
    \end{itemize}
\end{frame}


\begin{frame}{Why Wheeled Models}
\begin{itemize}
\item Wheeled robots are common thanks to mobility and mechanical simplicity.
\item Standard templates (unicycle, differential drive, etc.) cover many platforms.
\item Studying canonical cases makes it easier to adapt kinematic reasoning to new robots.
\end{itemize}
\end{frame}
% --- Original text start ---
% Robots come in all shapes, sizes, and configurations ... wheeled robots are perhaps the most widely used because of their high mobility and simple design.
% --- Original text end ---

\begin{frame}{Unicycle Model}
\begin{itemize}
\item Approximates the robot as a single wheel with state $[x,y,\theta]^\top$.
\item Shares the same no-slip constraint as the earlier wheel example.
\item Inputs $(v,\omega)$ command forward speed and body rotation.
\end{itemize}
\pause
\[
\begin{bmatrix}
\dot{x}\\\dot{y}\\\dot{\theta}
\end{bmatrix} =
\begin{bmatrix}
\cos\theta & 0\\
\sin\theta & 0\\
0 & 1
\end{bmatrix}
\begin{bmatrix}
 v\\ \omega
\end{bmatrix}
\]
\pause
\begin{center}
    \includegraphics[width=0.4\linewidth]{../book/figs/ch01_figs/unicycle.png}
\end{center}
\end{frame}
% --- Original text start ---
% The unicycle model ... assumes the robot can be approximated by a single wheel and uses inputs $v$ and $\omega$.
% --- Original text end ---

\begin{frame}{Differential Drive Geometry}
\begin{itemize}
\item Two rear wheels share an axle of width $L$; a passive front wheel carries no constraints.
\item Positions of the left and right wheels offset the body frame by $\pm \tfrac{L}{2}$.
\item Both wheels obey the same no-slip condition because they are rigidly linked.
\end{itemize}
\pause
\begin{center}
    \includegraphics[width=0.45\linewidth]{../book/figs/ch01_figs/diff_drive.png}
\end{center}
\end{frame}
% --- Original text start ---
% The differential drive model ... two wheels fixed on a rear shared axle with width $L$ (Figure \ref{fig:dd}).
% --- Original text end ---

\begin{frame}{Differential Drive Constraints}
\begin{itemize}
\item Wheel positions: $p_l = [x-\tfrac{L}{2}\sin\theta,\, y+\tfrac{L}{2}\cos\theta]$ and $p_r = [x+\tfrac{L}{2}\sin\theta,\, y-\tfrac{L}{2}\cos\theta]$.
\item Differentiating gives $\dot{p}_l$ and $\dot{p}_r$; both lead to $\dot{x}\sin\theta - \dot{y}\cos\theta = 0$.
\item The redundant constraints confirm that a single nonholonomic condition governs the chassis.
\end{itemize}
\end{frame}
% --- Original text start ---
% $p_l = [x - \frac{L}{2}\sin \theta, \: y + \frac{L}{2}\cos \theta]$, $p_r = [x + \frac{L}{2}\sin \theta, \: y - \frac{L}{2}\cos \theta]$ ... both yield $\dot{x}\sin \theta - \dot{y} \cos \theta = 0$.
% --- Original text end ---

\begin{frame}{Differential Drive Inputs}
\begin{itemize}
\item Use wheel rotation rates $(\omega_l,\omega_r)$ instead of $(v,\omega)$ for more realistic actuation.
\item Speed satisfies $v = \tfrac{r}{2}(\omega_l + \omega_r)$, while rotation obeys $\omega = \tfrac{r}{L}(\omega_r - \omega_l)$.
\item The resulting model maps wheel rates to body motion via
\pause
\[
\begin{bmatrix}
\dot{x}\\\dot{y}\\\dot{\theta}
\end{bmatrix} =
\begin{bmatrix}
\tfrac{r}{2}\cos\theta & \tfrac{r}{2}\cos\theta\\
\tfrac{r}{2}\sin\theta & \tfrac{r}{2}\sin\theta\\
\tfrac{r}{L} & -\tfrac{r}{L}
\end{bmatrix}
\begin{bmatrix}
\omega_r\\\omega_l
\end{bmatrix}.
\]
\end{itemize}
\end{frame}
% --- Original text start ---
% Relationships $v = \frac{r}{2}(\omega_l + \omega_r)$ and $\omega = \frac{r}{L}(\omega_r - \omega_l)$ lead to the differential drive model \eqref{eq:dd}.
% --- Original text end ---

\begin{frame}{Why Add Dynamics}
\begin{itemize}
\item Purely kinematic models ignore actuator limits, slip, and inertia.
\item Inputs like velocity or wheel rate may be hard to command without considering required forces.
\item Some tasks therefore augment kinematics with lightweight dynamics.
\end{itemize}
\end{frame}
% --- Original text start ---
% Mobile robot kinematic models are approximations of the true system behavior ... the choice of inputs ignores other important dynamics of the robot.
% --- Original text end ---

\begin{frame}{Extending the Unicycle}
\begin{itemize}
\item Replace velocity inputs with accelerations by adding integrator states.
\item Linear acceleration $a$ drives $\dot{v}$, and angular acceleration $\alpha$ drives $\dot{\omega}$.
\item The extended model couples kinematics with simple dynamics:
\pause
\[
\begin{bmatrix}
\dot{x}\\\dot{y}\\\dot{v}\\\dot{\theta}\\\dot{\omega}
\end{bmatrix} =
\begin{bmatrix}
 v\cos\theta\\ v\sin\theta\\ a\\ \omega\\ \alpha
\end{bmatrix}.
\]
\end{itemize}
\end{frame}
% --- Original text start ---
% One common extension to kinematic models ... replace velocity input $v$ with acceleration $a$ and add integrators, yielding \eqref{eq:extendeduni}.
% --- Original text end ---

\begin{frame}{Modeling Takeaways}
\begin{itemize}
\item Choose the simplest model that captures the geometry, actuation, and performance needs of the task.
\item Add dynamics when necessary, but leverage kinematic structure to keep planning manageable.
\item Revisit models as tasks evolve to ensure accuracy remains aligned with objectives.
\end{itemize}
\end{frame}
% --- Original text start ---
% In summary, factors to consider when deciding whether a certain kinematic model is sufficient, or if additional kinematics/dynamics are needed, include the robot's configuration/geometry and the task at hand.
% --- Original text end ---

\end{document}
