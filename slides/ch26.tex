\documentclass{beamer}

% Shared Beamer configuration for Principles of Robot Autonomy decks
\usetheme{Boadilla}
%\usecolortheme{seagull}

% Use Palatino to match the book's Tufte style
\usepackage{mathpazo}
\usefonttheme{serif}

\usepackage{amsmath}
\usepackage{amssymb}
\usepackage{bm}
\usepackage{graphicx}
\usepackage{enumitem}
\usepackage{xparse}
\usepackage{array}

% Increase vertical spacing between list items for readability.
\setlist[itemize]{itemsep=0.75em}
\setlist[enumerate]{itemsep=0.75em}

% Define bold symbol macros if not already defined
\providecommand{\x}{\boldsymbol{x}}
\providecommand{\z}{\boldsymbol{z}}
\providecommand{\q}{\boldsymbol{q}}
\providecommand{\bxi}{\boldsymbol{\xi}}
\providecommand{\bu}{\boldsymbol{u}}
\providecommand{\bmu}{\boldsymbol{\mu}}
\providecommand{\bSigma}{\boldsymbol{\Sigma}}
\providecommand{\y}{\boldsymbol{y}}
\providecommand{\m}{\mathcal{M}}
\providecommand{\bc}{\boldsymbol{c}}
\providecommand{\R}{\mathbb{R}}

% Animated lists by default, with override via optional overlay argument.
\beamerdefaultoverlayspecification{<+->}

% Paragraphs reveal sequentially by default.
\AtBeginDocument{%
  \ifx\paragraph\undefined
    \newcommand{\paragraph}[1]{\textbf{#1}\par\onslide<+->{} }
  \else
    \let\origparagraph\paragraph
    \renewcommand{\paragraph}[1]{\origparagraph{#1}\par\onslide<+->{} }
  \fi
}

% Add other shared packages or theme settings above this line as needed.

\author[COMP 4766]{Adapted from ``Principles of Robot Autonomy'' by D. Gammelli, J. Lorenzetti,
K. Luo, G. Zardini, and M. Pavone}
\date{\today}


% Bold vector shorthand for this deck
\newcommand{\f}{\mathbf{f}}
\newcommand{\w}{\mathbf{w}}
\newcommand{\btau}{\boldsymbol{\tau}}
\newcommand{\bd}{\mathbf{d}}

\title{Chapter 26 - Robot Manipulation}

\begin{document}

\begin{frame}[plain]
\titlepage
\end{frame}

\begin{frame}{Why Manipulation Is Hard}
\begin{itemize}
\item Even “simple” pick-and-place requires perception, grasp planning, collision-free motion, and force control.
\item Real scenes add clutter, uncertain object properties, and long action sequences (doors, sandwiches, tools).
\item We study grasp modeling, evaluation, planning, and learning recipes that scale beyond brittle heuristics.
\end{itemize}
\end{frame}
% --- Original text start ---
% Chapter intro motivating manipulation challenges and focus on grasping fundamentals.
% --- Original text end ---

\begin{frame}{What Is a Grasp?}
\begin{itemize}
\item Grasp: restrain an object via forces/torques applied at contact points.
\item High-DOF hands (e.g., Allegro: 12 finger DOF + 6 wrist DOF) explode the search space.
\item Must pick feasible contacts, avoid body collisions, and judge robustness post-execution.
\end{itemize}
\end{frame}
% --- Original text start ---
% Definition of grasp and bullet list of challenges (high DOF, contact choice, collisions, evaluation).
% --- Original text end ---

\begin{frame}{Contact Taxonomy}
\begin{itemize}
\item Point, line, and plane contacts; stability depends on which surfaces meet.
\item Plane contacts can be approximated by weighted point forces; line contacts by point pairs.
\item Practical analyses lean on point-on-plane abstractions for most surfaces.
\end{itemize}
\begin{center}
    \includegraphics[width=0.75\linewidth]{../book/figs/ch26_figs/contacts.png}
\end{center}
\end{frame}
% --- Original text start ---
% Contact types subsection and Figure contacts.
% --- Original text end ---

\begin{frame}{Point-on-Plane Models}
\begin{itemize}
\item Decompose contact force into normal and tangential parts: $\f = \f_{\text{normal}} + \f_{\text{tangent}}$, with $f_z \ge 0$.
\item Frictionless contact: only normal force; point contact with friction: forces inside a friction cone.
\item Soft-finger adds torsional moment $\tau_{\text{normal}}$ bounded by $|\tau_{\text{normal}}| \le \gamma f_z$.
\end{itemize}
\begin{center}
    \includegraphics[width=0.55\linewidth]{../book/figs/ch26_figs/frictioncone.png}
\end{center}
\end{frame}
% --- Original text start ---
% Point-on-plane contact models and friction cone discussion.
% --- Original text end ---

\begin{frame}{Linearizing Friction Cones}
\begin{itemize}
\item Replace the continuous cone with a pyramidal inner approximation for computation.
\item Each edge vector becomes a candidate force direction; convex combinations stay admissible.
\item Enables linear grasp maps and convex optimization.
\end{itemize}
\begin{center}
    \includegraphics[width=0.45\linewidth]{../book/figs/ch26_figs/linearfrictioncone.png}
\end{center}
\end{frame}
% --- Original text start ---
% Linear approximation figure and explanation.
% --- Original text end ---

\begin{frame}{Grasp Wrench Modeling}
\begin{itemize}
\item Wrench $\w = \begin{bmatrix}\f \\ \btau\end{bmatrix}$ stacks forces and torques at the COM.
\item Each contact contributes $\w_i = \begin{bmatrix}\f_i \\ \lambda(\bd_i \times \f_i)\end{bmatrix}$.
\item Grasp map $G = [G_1 \dots G_k]$ aggregates contacts: $\w = G[\f_1^\top \! \dots \! \f_k^\top]^\top$.
\end{itemize}
\end{frame}
% --- Original text start ---
% Wrench definition, contact wrench equation, grasp map formulation.
% --- Original text end ---

\begin{frame}{Example: 2D Grasp Wrench Space}
\begin{itemize}
\item Linearized cones at each contact map to discrete wrenches $\w_{i,j}$.
\item Grasp wrench space $\mathcal{W}$ is the Minkowski sum over contact wrench sets.
\item Larger $\mathcal{W}$ means the grasp can counter more external wrenches.
\end{itemize}
\begin{center}
    \includegraphics[width=0.72\linewidth]{../book/figs/ch26_figs/2Dexample_a.png}
\end{center}
\end{frame}
% --- Original text start ---
% Example 2D grasp with friction cones and resulting wrench space.
% --- Original text end ---

\begin{frame}{Closure Intuition}
\begin{itemize}
\item Form closure locks geometry (power grasps); force closure resists any wrench via contact forces.
\item Force closure typically needs friction and fewer contacts; form closure needs many constraints.
\item Real hardware seeks force closure plus safety margin for bounded forces.
\end{itemize}
\begin{center}
    \includegraphics[width=0.65\linewidth]{../book/figs/ch26_figs/closure.png}
\end{center}
\end{frame}
% --- Original text start ---
% Form vs force closure discussion and figure.
% --- Original text end ---

\begin{frame}{Force Closure Conditions}
\begin{itemize}
\item If $-\w \in \mathcal{W}$ for every external wrench, the grasp has force closure.
\item Theorem: in $n$-D wrench space need $n+1$ wrenches with $n$ independent and positive coefficients summing to zero.
\item Geometrically: the origin must lie strictly inside the grasp wrench space.
\end{itemize}
\end{frame}
% --- Original text start ---
% Definition of force closure and vector-space theorem.
% --- Original text end ---

\begin{frame}{Need Enough Wrenches}
\begin{itemize}
\item 2D forces-only: need 3 force directions with positive combo summing to zero (fails for pure frictionless contacts).
\item 2D wrenches (force+torque): need 4 directions; sometimes requires adding a third contact.
\item 3D wrenches: require at least 7 directions (3 contacts with friction or 7 frictionless).
\end{itemize}
\begin{center}
    \includegraphics[width=0.72\linewidth]{../book/figs/ch26_figs/2Dexample_b.png}
\end{center}
\end{frame}
% --- Original text start ---
% Examples demonstrating closure requirements in 2D/3D.
% --- Original text end ---

\begin{frame}{Grasp Wrench Hull}
\begin{itemize}
\item Exact $\mathcal{W}$ is costly; convex hull of discrete wrenches $\tilde{\mathcal{W}}$ is easier.
\item If the origin is inside $\tilde{\mathcal{W}}$, force closure holds (since $\tilde{\mathcal{W}} \subseteq \mathcal{W}$).
\item Provides a tractable basis for grasp quality metrics.
\end{itemize}
\begin{center}
    \includegraphics[width=0.58\linewidth]{../book/figs/ch26_figs/2Dexample_e.png}
\end{center}
\end{frame}
% --- Original text start ---
% Wrench hull definition and comparison figure.
% --- Original text end ---

\begin{frame}{Grasp Quality Metrics}
\begin{itemize}
\item Worst-case metric: radius $\epsilon$ of largest ball inside $\tilde{\mathcal{W}}$ signals minimum rejectable wrench.
\item Direction where the ball touches boundary = weakest disturbance direction.
\item Average-case metric: volume of $\tilde{\mathcal{W}}$ distinguishes grasps with same worst-case bound.
\end{itemize}
\begin{center}
    \includegraphics[width=0.55\linewidth]{../book/figs/ch26_figs/2Dexample_c.png}
\end{center}
\end{frame}
% --- Original text start ---
% Grasp quality metrics and illustrative figures.
% --- Original text end ---

\begin{frame}{Grasp Force Optimization}
\begin{itemize}
\item Solve for contact forces $\{\f_i\}$ that realize desired wrench $\w$ while respecting friction cones.
\item Second-order cone program: minimize convex cost (e.g., max contact force) subject to $\f_i \in \mathcal{F}_i$, extra limits, and $\w = G[\f_i]$.
\item Assumes known contact poses, friction, and wrench targets—key limitations in practice.
\end{itemize}
\end{frame}
% --- Original text start ---
% Grasp force optimization formulation and assumptions.
% --- Original text end ---

\begin{frame}{Why Learn Grasps?}
\begin{itemize}
\item Model-based pipelines need accurate friction, geometry, stiffness, and force execution.
\item Real-world uncertainty and perception noise often violate these assumptions.
\item Learning taps data (simulation or hardware) to sidestep explicit modeling and couple perception with grasp choice.
\end{itemize}
\end{frame}
% --- Original text start ---
% Motivation for learning-based approaches enumerating assumptions.
% --- Original text end ---

\begin{frame}{Synthetic Data for RGB Grasp Points}
\begin{itemize}
\item Saxena et al. rendered thousands of textured, lit scenes with labeled grasp points.
\item Supervised model predicts 2D grasp location, then triangulates 3D via structure-from-motion.
\item Generalizes to novel household objects despite purely synthetic training data.
\end{itemize}
\begin{center}
    \includegraphics[width=0.38\linewidth]{../book/figs/ch26_figs/saxenacup.png}
\end{center}
\end{frame}
% --- Original text start ---
% Saxena RGB image learning approach description and figure.
% --- Original text end ---

\begin{frame}{Dex-Net: Simulation at Scale}
\begin{itemize}
\item Build database of 3D objects, simulate millions of parallel-jaw grasps with uncertainty.
\item Learn to predict probability of force closure for candidate grasps $(x, v)$.
\item Use bandit selection online and update with outcomes for continual improvement.
\end{itemize}
\begin{center}
    \includegraphics[width=0.45\linewidth]{../book/figs/ch26_figs/dexnet.png}
\end{center}
\end{frame}
% --- Original text start ---
% Dex-Net supervised grasp success prediction description.
% --- Original text end ---

\begin{frame}{Trial-and-Error on Hardware}
\begin{itemize}
\item Levine et al. ran robots for months to gather grasp attempts in clutter.
\item End-to-end network maps RGB (and proprioception) to arm/gripper commands without explicit models.
\item Data-hungry but eliminates reliance on object models, friction estimates, or human-designed grasp sets.
\end{itemize}
\end{frame}
% --- Original text start ---
% Real-world trial-and-error learning approach description.
% --- Original text end ---

\begin{frame}{Planar Pushing Beyond Grasps}
\begin{itemize}
\item Physics-based limit-surface models characterize when friction resists vs allows sliding.
\item Controllers (e.g., MPC) need object properties, friction coefficients, and rigid-body assumptions.
\item Hybrid learning augments physics with learned perception-to-parameter maps; pure learning may overfit.
\end{itemize}
\end{frame}
% --- Original text start ---
% Planar pushing section including friction limit surface definition and hybrid vs learning comparison.
% --- Original text end ---

\begin{frame}{Contact-Rich Tasks with Multimodal Sensing}
\begin{itemize}
\item Tasks like key insertion require vision plus haptics; integrating modalities manually is hard.
\item Structured end-to-end pipelines encode raw sensor streams into compact features (autoencoders).
\item Policies trained on features act more data-efficiently and generalize across related tasks.
\end{itemize}
\end{frame}
% --- Original text start ---
% Contact-rich manipulation section describing Lee et al. approach.
% --- Original text end ---

\begin{frame}{Reminder: Grasping Principles}
\begingroup
\setbeamercolor{block title}{bg=gray!20,fg=black}
\setbeamercolor{block body}{bg=gray!10,fg=black}
\begin{block}{Key Takeaways}
\begin{itemize}
\item Model contacts to build grasp wrench spaces, check force closure, and rate quality.
\item Use optimization to realize desired wrenches but remember the modeling assumptions.
\item Lean on data-driven methods when perception, friction, or contact physics are uncertain or too complex.
\end{itemize}
\end{block}
\endgroup
\end{frame}
% --- Original text start ---
% Chapter summary tying modeling, optimization, and learning insights together.
% --- Original text end ---

\end{document}
