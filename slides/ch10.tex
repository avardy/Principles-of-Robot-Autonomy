\documentclass{beamer}

% Shared Beamer configuration for Principles of Robot Autonomy decks
\usetheme{Boadilla}
%\usecolortheme{seagull}

% Use Palatino to match the book's Tufte style
\usepackage{mathpazo}
\usefonttheme{serif}

\usepackage{amsmath}
\usepackage{amssymb}
\usepackage{bm}
\usepackage{graphicx}
\usepackage{enumitem}
\usepackage{xparse}
\usepackage{array}

% Increase vertical spacing between list items for readability.
\setlist[itemize]{itemsep=0.75em}
\setlist[enumerate]{itemsep=0.75em}

% Define bold symbol macros if not already defined
\providecommand{\x}{\boldsymbol{x}}
\providecommand{\z}{\boldsymbol{z}}
\providecommand{\q}{\boldsymbol{q}}
\providecommand{\bxi}{\boldsymbol{\xi}}
\providecommand{\bu}{\boldsymbol{u}}
\providecommand{\bmu}{\boldsymbol{\mu}}
\providecommand{\bSigma}{\boldsymbol{\Sigma}}
\providecommand{\y}{\boldsymbol{y}}
\providecommand{\m}{\mathcal{M}}
\providecommand{\bc}{\boldsymbol{c}}
\providecommand{\R}{\mathbb{R}}

% Animated lists by default, with override via optional overlay argument.
\beamerdefaultoverlayspecification{<+->}

% Paragraphs reveal sequentially by default.
\AtBeginDocument{%
  \ifx\paragraph\undefined
    \newcommand{\paragraph}[1]{\textbf{#1}\par\onslide<+->{} }
  \else
    \let\origparagraph\paragraph
    \renewcommand{\paragraph}[1]{\origparagraph{#1}\par\onslide<+->{} }
  \fi
}

% Add other shared packages or theme settings above this line as needed.

\author[COMP 4766]{Adapted from ``Principles of Robot Autonomy'' by D. Gammelli, J. Lorenzetti,
K. Luo, G. Zardini, and M. Pavone}
\date{\today}


\title{Chapter 10 - Image Processing}

\begin{document}

\frame{\titlepage}

\begin{frame}{Why Image Processing?}
\begin{itemize}
\item Earlier chapters mapped 3D structure; here we decode semantic cues like "what object am I seeing?"
\item Filtering, feature detection, and descriptors let robots interpret scenes for localization, mapping, and planning.
\item Treat grayscale images as functions $I(x,y)$; color data extends $I$ to RGB vectors.
\end{itemize}
\end{frame}
% --- Original text start ---
% Chapter introduction on image processing goals and image-as-function viewpoint.
% --- Original text end ---

\begin{frame}{Spatial Filtering Basics}
\begin{itemize}
\item Spatial filters modify a pixel using neighbors $S_{xy}$ via an operation $F$.
\item Kernel size $(2N+1)\times(2M+1)$ controls receptive field and smoothing strength.
\item Linear correlation filters: $I'(x,y) = \sum F(i,j)I(x+i,y+j)$.
\end{itemize}
\pause
\begin{center}
    \includegraphics[width=0.7\linewidth]{../tex/figs/ch10_figs/spatialfiltering.png}
\end{center}
\end{frame}
% --- Original text start ---
% Spatial filtering definition and Figure \ref{fig:spatial_filter_concept_fig}.
% --- Original text end ---

\begin{frame}{Correlation vs. Convolution}
\begin{itemize}
\item Convolution flips indices: $I'(x,y) = \sum F(i,j)I(x-i,y-j)$.
\item Associativity lets us combine smoothing + derivative masks before applying to $I$.
\item Boundary handling (padding, cropping, wrapping) keeps shapes consistent.
\end{itemize}
\end{frame}
% --- Original text start ---
% Correlation \eqref{eq:correlation}, convolution \eqref{eq:convolution}, and boundary discussion.
% --- Original text end ---

\begin{frame}{Indexing and Padding Tricks}
\begin{itemize}
\item Implementation often reindexes so the window's top-left aligns with $(x,y)$.
\item Zero ("same") padding keeps $I'$ the same size as $I$ for varied kernel shapes.
\item Beware apparent shifts when mixing center vs. top-left indexing schemes.
\end{itemize}
\pause
\begin{center}
    \includegraphics[width=0.58\linewidth]{../tex/figs/ch10_figs/topleftfilter_padding.png}
\end{center}
\end{frame}
% --- Original text start ---
% Example \ref{ex:padding} on top-left indexing and padding (Figures \ref{fig:topleftfilter}, \ref{fig:paddingfilter}).
% --- Original text end ---

\begin{frame}{Smoothing Filters}
\begin{itemize}
\item Moving average masks blur evenly while preserving brightness via normalization.
\item Gaussian smoothing weights neighbors by distance: $G_\sigma(x,y) = \frac{1}{2\pi\sigma^2} e^{-(x^2+y^2)/(2\sigma^2)}$.
\item Use larger $\sigma$ or kernels to suppress noise before differentiation.
\end{itemize}
\pause
\begin{center}
    \includegraphics[width=0.72\linewidth]{../tex/figs/ch10_figs/gaussianfilter.png}
\end{center}
\end{frame}
% --- Original text start ---
% Moving average, Gaussian filter definition, and Figure \ref{fig:gaussianfilter}.
% --- Original text end ---

\begin{frame}{Separable Masks}
\begin{itemize}
\item Mask $F$ is separable if $F = F_1 * F_2$, reducing complexity from $O(M^2wh)$ to $O(2Mwh)$.
\item Moving average mask equals an outer product of $[1,1,1]^\top$ and $[1,1,1]$.
\item Gaussian kernels decompose via $G_\sigma(x,y) = g_\sigma(x) g_\sigma(y)$.
\end{itemize}
\end{frame}
% --- Original text start ---
% Discussion of separable masks and computational savings.
% --- Original text end ---

\begin{frame}{Image Differentiation Filters}
\begin{itemize}
\item Finite differences approximate $\partial I/\partial x$ and $\partial I/\partial y$ in discrete grids.
\item Sobel masks aggregate neighbors for smoother gradient estimates.
\item Often smooth $I$ before differentiating to reduce noise amplification.
\end{itemize}
\pause
\begin{center}
    \includegraphics[width=0.7\linewidth]{../tex/figs/ch10_figs/track_sobel.png}
\end{center}
\end{frame}
% --- Original text start ---
% Central differences (Eq. \eqref{eq:cendiff}) and Sobel operator matrices.
% --- Original text end ---

\begin{frame}{Similarity Metrics}
\begin{itemize}
\item Correlate windows across views to support stereo/SfM correspondence.
\item Sum of absolute differences (SAD) is robust to small outliers.
\item Sum of squared differences (SSD) penalizes large mismatches heavily.
\end{itemize}
\pause
\begin{equation*}
\begin{aligned}
SAD &= \sum_{i=-N}^N \sum_{j=-M}^M \lvert I_1(x+i,y+j)-I_2(x'+i,y'+j)\rvert, \\
SSD &= \sum_{i=-N}^N \sum_{j=-M}^M [I_1(x+i,y+j)-I_2(x'+i,y'+j)]^2
\end{aligned}
\end{equation*}
\end{frame}
% --- Original text start ---
% Similarity measures \eqref{eq:similarity} for correspondence windows.
% --- Original text end ---

\begin{frame}{Edge Detection: 1D Intuition}
\begin{itemize}
\item Noisy differentiation hides true intensity jumps.
\item Smooth with Gaussian $g_\sigma$ before applying derivative filters.
\item Associativity lets $(\frac{d}{dx} * g_\sigma)$ act as a single edge kernel.
\end{itemize}
\pause
\begin{center}
    \includegraphics[width=0.55\linewidth]{../tex/figs/ch10_figs/edge_detection.png}
\end{center}
\end{frame}
% --- Original text start ---
% Figures \ref{fig:noisy} and \ref{fig:gauss} showing smoothing prior to differentiation.
% --- Original text end ---

\begin{frame}{Edge Detection Pipeline}
\begin{itemize}
\item Gaussian smoothing suppresses noise while preserving broad gradients.
\item Apply derivative filters (e.g., Sobel) in $x$ and $y$.
\item Enforce thresholds or non-maximum suppression for crisp responses.
\end{itemize}
\pause
\begin{center}
    \includegraphics[width=0.62\linewidth]{../tex/figs/ch10_figs/edge_detection_smooth.png}
\end{center}
\end{frame}
% --- Original text start ---
% Smoothing + differentiation workflow and Figure \ref{fig:gauss}.
% --- Original text end ---

\begin{frame}{2D Gradient Magnitude}
\begin{itemize}
\item Combined filters $G_{\sigma,x}$ and $G_{\sigma,y}$ yield smoothed derivatives.
\item Magnitude $\lvert\nabla S\rvert = \sqrt{S_x^2 + S_y^2}$ flags edge strength.
\item Practical detectors: Sobel, Canny, etc., tuned via thresholds.
\end{itemize}
\pause
\begin{center}
    \includegraphics[width=0.8\linewidth]{../tex/figs/ch10_figs/sobel.png}
\end{center}
\end{frame}
% --- Original text start ---
% Gradient computation, thresholding, and Figure \ref{fig:sobel}.
% --- Original text end ---

\begin{frame}{Canny Edge Detector}
\begin{itemize}
\item Adds hysteresis thresholding and non-maximum suppression.
\item Produces thin, well-localized edges with fewer false positives.
\item Widely used baseline in robotics perception stacks.
\end{itemize}
\pause
\begin{center}
    \includegraphics[width=0.78\linewidth]{../tex/figs/ch10_figs/canny.png}
\end{center}
\end{frame}
% --- Original text start ---
% Figure \ref{fig:canny} and discussion of accuracy/localization criteria.
% --- Original text end ---

\begin{frame}{Corner Detection}
\begin{itemize}
\item Corners exhibit high gradient variation in all directions.
\item Harris detector offers rotation and photometric invariance, but not scale invariance.
\item Scale-aware variants (Harris-Laplacian, SIFT) improve repeatability under zoom/affine changes.
\end{itemize}
\end{frame}
% --- Original text start ---
% Corner definition, Harris detector properties, motivation for scale-invariant methods.
% --- Original text end ---

\begin{frame}{Feature Descriptors}
\begin{itemize}
\item Describe patches so matches are distinct and repeatable across viewpoint/lighting.
\item Raw intensity windows lack invariance; modern descriptors include SIFT, SURF, FAST, BRIEF, ORB, BRISK.
\item Robust descriptors feed tasks like tracking, SLAM, retrieval, and classification.
\end{itemize}
\end{frame}
% --- Original text start ---
% Descriptor goals (repeatability, distinctiveness) and popular families.
% --- Original text end ---

\begin{frame}{Exercises}
\begin{itemize}
\item Work through HW3 Problem 3 \textit{Linear Filtering} to practice kernel design.
\item Repository: \url{https://github.com/PrinciplesofRobotAutonomy/AA274A_HW3}.
\item Explore smoothing vs. sharpening trade-offs on real imagery.
\end{itemize}
\end{frame}
% --- Original text start ---
% Exercises section referencing HW3 repository and topic.
% --- Original text end ---

\begin{frame}{Reminder: Image Processing Checklist}
\begingroup
\setbeamercolor{block title}{bg=gray!20,fg=black}
\setbeamercolor{block body}{bg=gray!10,fg=black}
\begin{block}{Key Takeaways}
\begin{itemize}
\item Choose kernels (mean, Gaussian, Sobel) that balance smoothing vs. detail preservation.
\item Keep implementation details straight: padding rules, separability, combined filters.
\item Edges, corners, and descriptors must be both distinctive and repeatable for downstream robotics tasks.
\end{itemize}
\end{block}
\endgroup
\end{frame}
% --- Original text start ---
% Summary of filtering, detection, and descriptor guidance.
% --- Original text end ---

\end{document}
