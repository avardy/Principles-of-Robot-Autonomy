\documentclass{beamer}

% Shared Beamer configuration for Principles of Robot Autonomy decks
\usetheme{Boadilla}
%\usecolortheme{seagull}

% Use Palatino to match the book's Tufte style
\usepackage{mathpazo}
\usefonttheme{serif}

\usepackage{amsmath}
\usepackage{amssymb}
\usepackage{bm}
\usepackage{graphicx}
\usepackage{enumitem}
\usepackage{xparse}
\usepackage{array}

% Increase vertical spacing between list items for readability.
\setlist[itemize]{itemsep=0.75em}
\setlist[enumerate]{itemsep=0.75em}

% Define bold symbol macros if not already defined
\providecommand{\x}{\boldsymbol{x}}
\providecommand{\z}{\boldsymbol{z}}
\providecommand{\q}{\boldsymbol{q}}
\providecommand{\bxi}{\boldsymbol{\xi}}
\providecommand{\bu}{\boldsymbol{u}}
\providecommand{\bmu}{\boldsymbol{\mu}}
\providecommand{\bSigma}{\boldsymbol{\Sigma}}
\providecommand{\y}{\boldsymbol{y}}
\providecommand{\m}{\mathcal{M}}
\providecommand{\bc}{\boldsymbol{c}}
\providecommand{\R}{\mathbb{R}}

% Animated lists by default, with override via optional overlay argument.
\beamerdefaultoverlayspecification{<+->}

% Paragraphs reveal sequentially by default.
\AtBeginDocument{%
  \ifx\paragraph\undefined
    \newcommand{\paragraph}[1]{\textbf{#1}\par\onslide<+->{} }
  \else
    \let\origparagraph\paragraph
    \renewcommand{\paragraph}[1]{\origparagraph{#1}\par\onslide<+->{} }
  \fi
}

% Add other shared packages or theme settings above this line as needed.

\author[COMP 4766]{Adapted from ``Principles of Robot Autonomy'' by D. Gammelli, J. Lorenzetti,
K. Luo, G. Zardini, and M. Pavone}
\date{\today}


\title{Chapter 20 - Decision Making}

\begin{document}

\begin{frame}[plain]
\titlepage
\end{frame}

\begin{frame}{Why Sequential Decisions?}
\begin{itemize}
\item Higher-level autonomy must reason over action sequences under uncertainty, not single commands.
\item Examples include intersection negotiation or choosing task order (finish A before B?).
\item Dynamic programming provides a structured way to plan ahead despite uncertainty.
\end{itemize}
\end{frame}
% --- Original text start ---
% Intro paragraphs motivating sequential decision making and mention of dynamic programming.
% --- Original text end ---

\begin{frame}{Deterministic Model Ingredients}
\begin{itemize}
\item Discrete-time model $\x_{k+1} = f_k(\x_k, \bu_k)$ for $k = 0,\dots,N-1$.
\item State $\x$ can be physical pose or abstract FSM-style modes.
\item Horizon $N$ limits how far ahead the plan must look.
\end{itemize}
\pause
\begin{equation*}
\x_{k+1}= f_k(\x_k, \bu_k)
\end{equation*}
\end{frame}
% --- Original text start ---
% Deterministic decision making model equation and discussion of discrete-time setup.
% --- Original text end ---

\begin{frame}{Controls and Costs}
\begin{itemize}
\item Control feasibility encoded via admissible set $\mathcal{U}(\x_k)$.
\item Additive stage and terminal costs capture objectives.
\item Structure mirrors trajectory optimization but in discrete time.
\end{itemize}
\pause
\begin{equation*}
J = g_N(\x_N) + \sum_{k=0}^{N-1} g_k(\x_k, \bu_k)
\end{equation*}
\end{frame}
% --- Original text start ---
% Discussion of admissible controls (Eq. 20.2) and additive cost (Eq. 20.3).
% --- Original text end ---

\begin{frame}{Deterministic Problem Statement}
\begin{itemize}
\item Optimize open-loop sequence $\{\bu_0,\dots,\bu_{N-1}\}$ from initial state $\x_0$.
\item Minimize cumulative cost subject to dynamics and constraints.
\item Brute force over all sequences quickly becomes intractable.
\end{itemize}
\pause
\begin{equation*}
J^*(\x_0) = \min_{\bu_k \in \mathcal{U}(\x_k)} J(\x_0, \bu_0, \dots, \bu_{N-1})
\end{equation*}
\end{frame}
% --- Original text start ---
% Definition of deterministic decision making problem (Eq. 20.4).
% --- Original text end ---

\begin{frame}{Principle of Optimality}
\begin{itemize}
\item Any suffix of an optimal control sequence is itself optimal for the corresponding tail problem.
\item Enables reuse of downstream solutions rather than re-solving full sequences.
\item Proof by contradiction: replacing a subpath with a better one contradicts optimality.
\end{itemize}
\pause
\begin{center}
    \includegraphics[width=0.6\linewidth]{../book/figs/ch20_figs/princopt1.png}
\end{center}
\end{frame}
% --- Original text start ---
% Principle of optimality narrative and Figure \ref{fig:princopt1}.
% --- Original text end ---

\begin{frame}{Tail Reuse in Practice}
\begin{itemize}
\item Known optimal tails let us compare only immediate successor states.
\item Backward dynamic programming evaluates states in reverse order.
\item Reduces search from exponential paths to a handful of candidates.
\end{itemize}
\pause
\begin{center}
    \includegraphics[width=0.68\linewidth]{../book/figs/ch20_figs/princopt2.png}
\end{center}
\end{frame}
% --- Original text start ---
% Figure \ref{fig:princopt2} discussion showing reduced candidate paths.
% --- Original text end ---

\begin{frame}{Dynamic Programming (Deterministic)}
\begin{itemize}
\item Backward recursion stores optimal cost-to-go $J_k^*(\x_k)$.
\item Bellman update solves a local minimization per state and stage.
\item Forward sweep recovers the optimal control sequence from $J_k^*$.
\end{itemize}
\pause
\begin{equation*}
J_k^*(\x_k) = \min_{\bu_k \in \mathcal{U}(\x_k)} g_k(\x_k,\bu_k) + J_{k+1}^*(f_k(\x_k, \bu_k))
\end{equation*}
\end{frame}
% --- Original text start ---
% Algorithm \ref{alg:dDP} description and forward pass explanation.
% --- Original text end ---

\begin{frame}{Example: Grid DP}
\begin{itemize}
\item State is current node (a--h); controls follow allowed arrows.
\item Backward DP quickly recovers the optimal path $a \to d \to e \to f \to g \to h$ with cost 18.
\item Same data yields optimal strategies for any start/horizon.
\end{itemize}
\pause
\begin{center}
    \includegraphics[width=0.8\linewidth]{../book/figs/ch20_figs/dpexample.png}
\end{center}
\end{frame}
% --- Original text start ---
% Example \ref{ex:detDP} problem and figure showing optimal path.
% --- Original text end ---

\begin{frame}{Stochastic Extension}
\begin{itemize}
\item Inject disturbances $w_k$ via $\x_{k+1}=f_k(\x_k, \bu_k, w_k)$ with known $P_k(w_k \mid \x_k, \bu_k)$.
\item Need closed-loop policies $\pi_k(\x_k)$ to react to random outcomes.
\item Cost evaluates expectation of terminal and stage losses.
\end{itemize}
\pause
\begin{equation*}
J_\pi(\x_0) = E\big[g_N(\x_N) + \sum_{k=0}^{N-1} g_k(\x_k, \pi_k(\x_k), w_k)\big]
\end{equation*}
\end{frame}
% --- Original text start ---
% Stochastic model equation (Eq. 20.12) and expected cost (Eq. 20.13).
% --- Original text end ---

\begin{frame}{Stochastic Optimality}
\begin{itemize}
\item Principle of optimality still holds when reasoning over reachable states and distributions.
\item Tail policy sequence remains optimal for every reachable tail state.
\item Enables the same backward reasoning with expectations in place of sums.
\end{itemize}
\end{frame}
% --- Original text start ---
% Theorem on stochastic principle of optimality.
% --- Original text end ---

\begin{frame}{Dynamic Programming (Stochastic)}
\begin{itemize}
\item Bellman update now minimizes expected cost with respect to $w_k$.
\item Resulting value functions $J_k(\x_k)$ guide optimal feedback policy.
\item Policies recovered via argmin over admissible controls after value iteration.
\end{itemize}
\pause
\begin{equation*}
J_k(\x_k) = \min_{\bu_k \in \mathcal{U}(\x_k)} E_{w_k}\big[g_k + J_{k+1}(f_k(\x_k, \bu_k, w_k))\big]
\end{equation*}
\end{frame}
% --- Original text start ---
% Algorithm \ref{alg:sDP} description.
% --- Original text end ---

\begin{frame}{Inventory Control Example}
\begin{itemize}
\item State: stock level $x_k$; control: new orders $u_k$; disturbance: random demand $w_k$.
\item DP computes value function over three-step horizon with constraints $x_k + u_k \leq 2$.
\item Optimal policy: order one item only when stock is zero; otherwise hold.
\end{itemize}
\end{frame}
% --- Original text start ---
% Example \ref{ex:stoDP} summary and findings.
% --- Original text end ---

\begin{frame}{DP in Practice}
\begin{itemize}
\item Curse of dimensionality: Bellman updates explode with state dimension.
\item Curse of modeling: real systems rarely yield accurate transition probabilities.
\item Curse of time: data may change online, forcing repeated solves.
\end{itemize}
\end{frame}
% --- Original text start ---
% Challenges section describing the three curses.
% --- Original text end ---

\begin{frame}{Reinforcement Learning Motivation}
\begin{itemize}
\item Suboptimal DP variants approximate value functions or policies.
\item Value-space methods approximate $J_k$; policy-space methods fit parameterized controllers.
\item Goal: retain DP’s structure while coping with large, uncertain problems.
\end{itemize}
\end{frame}
% --- Original text start ---
% Reinforcement learning subsection describing approximate DP families.
% --- Original text end ---

\begin{frame}{Reminder: Decision-Making Toolbox}
\begingroup
\setbeamercolor{block title}{bg=gray!20,fg=black}
\setbeamercolor{block body}{bg=gray!10,fg=black}
\begin{block}{Key Takeaways}
\begin{itemize}
\item Sequential decisions hinge on additive models, admissible controls, and the principle of optimality.
\item Dynamic programming solves deterministic and stochastic problems via backward Bellman recursions.
\item Practical deployments wrestle with modeling, dimensionality, and motivate RL approximations.
\end{itemize}
\end{block}
\endgroup
\end{frame}
% --- Original text start ---
% Chapter summary: decision-making models, DP algorithms, challenges, and RL motivation.
% --- Original text end ---

\end{document}
