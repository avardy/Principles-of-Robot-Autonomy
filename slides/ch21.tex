\documentclass{beamer}

% Shared Beamer configuration for Principles of Robot Autonomy decks
\usetheme{Boadilla}
%\usecolortheme{seagull}

% Use Palatino to match the book's Tufte style
\usepackage{mathpazo}
\usefonttheme{serif}

\usepackage{amsmath}
\usepackage{amssymb}
\usepackage{bm}
\usepackage{graphicx}
\usepackage{enumitem}
\usepackage{xparse}
\usepackage{array}

% Increase vertical spacing between list items for readability.
\setlist[itemize]{itemsep=0.75em}
\setlist[enumerate]{itemsep=0.75em}

% Define bold symbol macros if not already defined
\providecommand{\x}{\boldsymbol{x}}
\providecommand{\z}{\boldsymbol{z}}
\providecommand{\q}{\boldsymbol{q}}
\providecommand{\bxi}{\boldsymbol{\xi}}
\providecommand{\bu}{\boldsymbol{u}}
\providecommand{\bmu}{\boldsymbol{\mu}}
\providecommand{\bSigma}{\boldsymbol{\Sigma}}
\providecommand{\y}{\boldsymbol{y}}
\providecommand{\m}{\mathcal{M}}
\providecommand{\bc}{\boldsymbol{c}}
\providecommand{\R}{\mathbb{R}}

% Animated lists by default, with override via optional overlay argument.
\beamerdefaultoverlayspecification{<+->}

% Paragraphs reveal sequentially by default.
\AtBeginDocument{%
  \ifx\paragraph\undefined
    \newcommand{\paragraph}[1]{\textbf{#1}\par\onslide<+->{} }
  \else
    \let\origparagraph\paragraph
    \renewcommand{\paragraph}[1]{\origparagraph{#1}\par\onslide<+->{} }
  \fi
}

% Add other shared packages or theme settings above this line as needed.

\author[COMP 4766]{Adapted from ``Principles of Robot Autonomy'' by D. Gammelli, J. Lorenzetti,
K. Luo, G. Zardini, and M. Pavone}
\date{\today}


\title{Chapter 21 - Reinforcement Learning}

\begin{document}

\begin{frame}[plain]
\titlepage
\end{frame}

\begin{frame}{From DP to RL}
\begin{itemize}
\item Reinforcement learning (RL) generalizes dynamic programming by maximizing cumulative reward instead of minimizing cost.
\item Environment can be a black box, so the policy must be learned from interaction rather than exact models.
\item Successes span helicopter aerobatics, game play, finance, and robotics, motivating a lightweight survey here.
\end{itemize}
\end{frame}
% --- Original text start ---
% Intro paragraphs comparing DP and RL and motivating broader applications.
% --- Original text end ---

\begin{frame}{Interaction Loop}
\begin{itemize}
\item Agent issues an action and receives next state plus reward feedback.
\item Learning hinges on observing how the environment responds, regardless of whether the transition law is known.
\item Same abstraction covers both model-based (known dynamics) and model-free (unknown dynamics) settings.
\end{itemize}
\pause
\begin{center}
    \includegraphics[width=0.6\linewidth]{../book/figs/ch21_figs/rl.png}
\end{center}
\end{frame}
% --- Original text start ---
% Figure 21.1 showing the agent-environment feedback diagram.
% --- Original text end ---

\begin{frame}{RL Problem Ingredients}
\begin{itemize}
\item States $\x \in \mathcal{X}$ and controls $\bu \in \mathcal{U}$ follow stochastic transitions.
\item Policies $\pi_t(\x_t)$ produce closed-loop actions even when disturbances occur.
\item Rewards $R(\x_t,\bu_t)$ quantify desirability of each state-action pair.
\end{itemize}
\pause
\begin{equation*}
p(\x_t \mid \x_{t-1}, \bu_{t-1})
\end{equation*}
\end{frame}
% --- Original text start ---
% Problem formulation subsection defining the transition model, policy notation, and reward function.
% --- Original text end ---

\begin{frame}{Value Function Objective}
\begin{itemize}
\item Value function $V_T^\pi(\x)$ measures discounted reward when executing policy $\pi$.
\item Discount factor $\gamma$ keeps infinite-horizon sums finite by down-weighting distant rewards.
\item Optimal policy $\pi^*$ maximizes $V_T^\pi(\x)$ for every admissible state.
\end{itemize}
\pause
\begin{equation*}
V_T^\pi(\x) = E\Big[\sum_{t=0}^{T-1} \gamma^t R(\x_t, \pi_t(\x_t)) \,\Big|\, \x_0 = \x\Big]
\end{equation*}
\end{frame}
% --- Original text start ---
% Definition of value function, discounting, and the reinforcement learning problem statement.
% --- Original text end ---

\begin{frame}{Value Iteration Refresher}
\begin{itemize}
\item Principle of optimality leads to Bellman recursion identical in spirit to stochastic DP.
\item Backward sweep updates value function and retrieves optimal stage policies.
\item Infinite-horizon setups iterate until a stationary value and policy emerge.
\end{itemize}
\pause
\begin{equation*}
V_{k+1}^*(\x) = \max_{\bu \in \mathcal{U}} \Big[R(\x,\bu) + \gamma \sum_{\x'} p(\x' \mid \x, \bu) V_k^*(\x')\Big]
\end{equation*}
\end{frame}
% --- Original text start ---
% Section on model-based RL and Algorithm 21.1 (dynamic programming/value iteration).
% --- Original text end ---

\begin{frame}{Inventory Control Revisited}
\begin{itemize}
\item Same stochastic inventory system from Chapter 20 fits RL notation by treating rewards as negative costs.
\item Bellman updates recover ordering policy: only restock when inventory hits zero.
\item Demonstrates equivalence between DP analysis and value-iteration phrasing.
\end{itemize}
\end{frame}
% --- Original text start ---
% Example 21.1 describing the inventory control problem solved via RL Bellman updates.
% --- Original text end ---

\begin{frame}{Policy Evaluation Building Block}
\begin{itemize}
\item Policy evaluation computes $V_k^\pi(\x)$ for a fixed policy by removing the control optimization.
\item Recursion resembles Bellman update but substitutes the chosen action $\pi(\x)$.
\item Infinite-horizon case solves a linear system or iterates until convergence.
\end{itemize}
\pause
\begin{equation*}
V_{k+1}^\pi(\x) = R(\x, \pi(\x)) + \gamma \sum_{\x'} p(\x' \mid \x, \pi(\x)) V_k^\pi(\x')
\end{equation*}
\end{frame}
% --- Original text start ---
% Policy evaluation subsection and Algorithm 21.2.
% --- Original text end ---

\begin{frame}{Q-Functions and Policy Iteration}
\begin{itemize}
\item State-action value $Q^\pi(\x,\bu)$ scores a one-step deviation followed by policy $\pi$.
\item Optimal value $V^*$ arises by maximizing $Q^*$ over admissible controls.
\item Policy iteration cycles policy evaluation and greedy improvement until convergence.
\end{itemize}
\pause
\begin{equation*}
Q_{k+1}^\pi(\x,\bu) = R(\x,\bu) + \gamma \sum_{\x'} p(\x' \mid \x, \bu) V_k^\pi(\x')
\end{equation*}
\end{frame}
% --- Original text start ---
% Q-function definition and policy iteration algorithm discussion.
% --- Original text end ---

\begin{frame}{When Models Are Unknown}
\begin{itemize}
\item Many robots only observe $(\x_t, \bu_t, r_t)$ samples without transition probabilities.
\item Learning must rely on experience data, either logged or gathered online.
\item Distinguish model-based RL (learn transitions) from model-free RL (directly learn value/policy).
\end{itemize}
\end{frame}
% --- Original text start ---
% Model-free reinforcement learning introduction describing trajectory data requirements.
% --- Original text end ---

\begin{frame}{Q-Learning Essentials}
\begin{itemize}
\item Update estimates of $Q(\x,\bu)$ toward sampled reward plus discounted best successor value.
\item Requires only tuples $(\x_t, \bu_t, r_t, \x_{t+1})$ and a learning rate $\alpha$.
\item Converges for small discrete spaces but struggles with large or continuous domains.
\end{itemize}
\pause
\begin{equation*}
Q(\x_t,\bu_t) \leftarrow Q(\x_t,\bu_t) + \alpha \Big(r_t + \gamma \max_{\bu'} Q(\x_{t+1}, \bu') - Q(\x_t,\bu_t)\Big)
\end{equation*}
\end{frame}
% --- Original text start ---
% Q-learning subsection and Algorithm 21.3.
% --- Original text end ---

\begin{frame}{Policy Gradient Intuition}
\begin{itemize}
\item Parameterize a stochastic policy $\pi_{\bm{\theta}}(\bu \mid \x)$ and optimize expected return directly.
\item Likelihood-ratio trick estimates gradients from sampled trajectories without knowing dynamics.
\item REINFORCE repeatedly samples, computes trajectory reward, and ascends along the estimated gradient.
\end{itemize}
\pause
\begin{equation*}
\nabla_{\bm{\theta}} J(\bm{\theta}) \approx r(\tau) \sum_{t=0}^{T-1} \nabla_{\bm{\theta}} \log \pi_{\bm{\theta}}(\bu_t \mid \x_t)
\end{equation*}
\end{frame}
% --- Original text start ---
% Policy gradient subsection covering trajectory-based gradient estimation and REINFORCE.
% --- Original text end ---

\begin{frame}{Actor-Critic Blend}
\begin{itemize}
\item Actor parameterizes the policy while critic estimates a value baseline $V_{\bm{\phi}}(\x)$.
\item Baseline reduces variance by comparing returns to expected performance.
\item Simultaneous updates of $\bm{\theta}$ and $\bm{\phi}$ improve sample efficiency versus pure policy gradients.
\end{itemize}
\end{frame}
% --- Original text start ---
% Actor-critic subsection explaining joint learning of policy and value function.
% --- Original text end ---

\begin{frame}{Deep RL and Function Approximation}
\begin{itemize}
\item Neural networks approximate value functions or Q-functions to scale toward high-dimensional states.
\item Policies parameterized by deep nets can map images or other rich observations to actions.
\item Brings RL closer to real-world perception-action loops, at the cost of stability challenges.
\end{itemize}
\end{frame}
% --- Original text start ---
% Deep reinforcement learning subsection highlighting neural network approximators.
% --- Original text end ---

\begin{frame}{Exploration vs. Exploitation}
\begin{itemize}
\item Exploration ensures coverage of informative states/actions; exploitation leverages current knowledge.
\item Simple $\epsilon$-greedy schemes randomize with probability $\epsilon$ to probe alternatives.
\item Balancing both is fundamental for discovering high-reward behaviors without wasting data.
\end{itemize}
\end{frame}
% --- Original text start ---
% Discussion on exploration versus exploitation and $\epsilon$-greedy strategy.
% --- Original text end ---

\begin{frame}{Reminder: RL Toolkit}
\begingroup
\setbeamercolor{block title}{bg=gray!20,fg=black}
\setbeamercolor{block body}{bg=gray!10,fg=black}
\begin{block}{Key Takeaways}
\begin{itemize}
\item RL reframes sequential decision making via reward, discounting, value functions, and policies.
\item Model-based tools (value and policy iteration) mirror dynamic programming in RL notation.
\item Model-free families—Q-learning, policy gradients, actor-critic, deep RL—trade models for sampled experience while juggling exploration.
\end{itemize}
\end{block}
\endgroup
\end{frame}
% --- Original text start ---
% Chapter summary emphasizing RL formulations, algorithms, and exploration challenges.
% --- Original text end ---

\end{document}
