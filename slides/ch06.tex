\documentclass{beamer}

% Shared Beamer configuration for Principles of Robot Autonomy decks
\usetheme{Boadilla}
%\usecolortheme{seagull}

% Use Palatino to match the book's Tufte style
\usepackage{mathpazo}
\usefonttheme{serif}

\usepackage{amsmath}
\usepackage{amssymb}
\usepackage{bm}
\usepackage{graphicx}
\usepackage{enumitem}
\usepackage{xparse}
\usepackage{array}

% Increase vertical spacing between list items for readability.
\setlist[itemize]{itemsep=0.75em}
\setlist[enumerate]{itemsep=0.75em}

% Define bold symbol macros if not already defined
\providecommand{\x}{\boldsymbol{x}}
\providecommand{\z}{\boldsymbol{z}}
\providecommand{\q}{\boldsymbol{q}}
\providecommand{\bxi}{\boldsymbol{\xi}}
\providecommand{\bu}{\boldsymbol{u}}
\providecommand{\bmu}{\boldsymbol{\mu}}
\providecommand{\bSigma}{\boldsymbol{\Sigma}}
\providecommand{\y}{\boldsymbol{y}}
\providecommand{\m}{\mathcal{M}}
\providecommand{\bc}{\boldsymbol{c}}
\providecommand{\R}{\mathbb{R}}

% Animated lists by default, with override via optional overlay argument.
\beamerdefaultoverlayspecification{<+->}

% Paragraphs reveal sequentially by default.
\AtBeginDocument{%
  \ifx\paragraph\undefined
    \newcommand{\paragraph}[1]{\textbf{#1}\par\onslide<+->{} }
  \else
    \let\origparagraph\paragraph
    \renewcommand{\paragraph}[1]{\origparagraph{#1}\par\onslide<+->{} }
  \fi
}

% Add other shared packages or theme settings above this line as needed.

\author[COMP 4766]{Adapted from ``Principles of Robot Autonomy'' by D. Gammelli, J. Lorenzetti,
K. Luo, G. Zardini, and M. Pavone}
\date{\today}


\title{Chapter 06 - Sampling-Based Motion Planning}

\begin{document}

\frame{\titlepage}

\begin{frame}{From Roadmaps to Sampling}
\begin{itemize}
\item Chapter 05 relied on precomputed maps of $C_{free}$ before running graph search.
\item Sampling-based planners probe the configuration space lazily via a collision checker.
\item They shine when modeling every obstacle cell upfront is infeasible or wasteful.
\end{itemize}
\end{frame}
% --- Original text start ---
% Chapter intro motivating sampling-based planners vs grid/combinatorial approaches.
% --- Original text end ---

\begin{frame}{Key Ingredients}
\begin{itemize}
\item Random sampler proposes configurations; admissibility decided by a collision module.
\item Incrementally build a roadmap or tree rather than exhaustively enumerating $C_{free}$.
\item Trade computation (sampling + checking) for flexibility across robot models.
\end{itemize}
\pause
\begin{itemize}
\item Cannot certify infeasibility; success hinges on enough samples and well-tuned parameters.
\end{itemize}
\end{frame}
% --- Original text start ---
% Overview of sampling-based advantages/disadvantages and collision checker role.
% --- Original text end ---

\begin{frame}{Probabilistic Roadmap (PRM)}
\begin{itemize}
\item Sample $n$ configurations $q_i$ uniformly or with heuristics.
\item Run the collision checker, discard infeasible samples.
\item Connect nearby vertices (within radius $r$) if the straight line segment is collision-free.
\end{itemize}
\pause
\begin{itemize}
\item Once $G=(V,E)$ is built, attach $q_I,q_G$ and call your favorite graph search.
\end{itemize}
\end{frame}
% --- Original text start ---
% PRM algorithm outline and roadmap description.
% --- Original text end ---

\begin{frame}{PRM Connectivity Trade-offs}
\begin{itemize}
\item Larger $r$ improves connectivity but explodes edge count and collision-check effort.
\item Smaller $r$ risks disconnected components and missed narrow passages.
\item Multi-query settings justify dense sampling because the roadmap is reused.
\item Environment changes invalidate the roadmap, forcing a rebuild from scratch.
\end{itemize}
\end{frame}
% --- Original text start ---
% Discussion on choosing radius r, sample counts, and multi-query reuse.
% --- Original text end ---

\begin{frame}{PRM Example}
\begin{center}
    \includegraphics[width=0.7\linewidth]{../tex/figs/ch06_figs/prm.png}
\end{center}
\pause
\begin{itemize}
\item Random vertices cover $C_{free}$ and pink path illustrates graph-search output.
\item Sparse, uniform sampling keeps the roadmap tractable yet expressive.
\end{itemize}
\end{frame}
% --- Original text start ---
% Figure \ref{fig:prm-graph} caption description.
% --- Original text end ---

\begin{frame}{Rapidly-Exploring Random Trees (RRT)}
\begin{itemize}
\item Initialize tree $T$ at $q_I$.
\item Sample $q$ in $C$ and find nearest vertex $q_{near}$.
\item Steer toward $q$ to obtain $q_{new}$ while staying in $C_{free}$.
\item Add $q_{new}$ and edge $(q_{near}, q_{new})$ to $T$.
\end{itemize}
\pause
\begin{itemize}
\item Occasionally bias the sampler toward $q_G$ to hasten completion.
\end{itemize}
\end{frame}
% --- Original text start ---
% RRT algorithm description including tree initialization and steps.
% --- Original text end ---

\begin{frame}{Voronoi Bias Intuition}
\begin{center}
    \includegraphics[width=0.7\linewidth]{../tex/figs/ch06_figs/rrt.png}
\end{center}
\pause
\begin{itemize}
\item Frontier nodes own large Voronoi regions, so random samples likely extend the tree outward.
\item Supports fast single-query exploration without blanketing the entire space.
\end{itemize}
\end{frame}
% --- Original text start ---
% Figure \ref{fig:rrt-graph} showing exploration tree and Voronoi bias explanation.
% --- Original text end ---

\begin{frame}{RRT Variants}
\begin{itemize}
\item RRT*: rewires locally to minimize path cost, achieving asymptotic optimality.
\item RRT-Connect: grows trees from $q_I$ and $q_G$ simultaneously until they meet.
\item Parameter choices (step sizes, goal bias) govern exploration vs exploitation.
\end{itemize}
\end{frame}
% --- Original text start ---
% Paragraph describing RRT*, RRT-Connect, and sampling tweaks.
% --- Original text end ---

\begin{frame}{Reminder: Collision Checking Budget}
\begingroup
\setbeamercolor{block title}{bg=gray!20,fg=black}
\setbeamercolor{block body}{bg=gray!10,fg=black}
\begin{block}{Concept Recap}
\begin{itemize}
\item Every edge validation usually samples multiple intermediate poses.
\item Savings from lazy roadmaps or limited radii directly reduce collision-check calls.
\item Profile collision queries early; they dominate runtime in cluttered worlds.
\end{itemize}
\end{block}
\endgroup
\end{frame}
% --- Original text start ---
% Discussion noting collision checker cost and lazy validation motivation.
% --- Original text end ---

\begin{frame}{Asymptotic Guarantees}
\begin{itemize}
\item With bounded $C$ and constant $r$, PRM and RRT are probabilistically complete.
\item PRM can converge to optimal cost if $r$ scales appropriately with $n$.
\item Standard RRT may remain suboptimal with nonzero probability.
\item Analyze large-$n$ behavior to reason about convergence rather than finite certificates.
\end{itemize}
\end{frame}
% --- Original text start ---
% Theoretical results for PRM/RRT completeness and optimality.
% --- Original text end ---

\begin{frame}{FMT$^*$ Overview}
\begin{itemize}
\item FMT$^*$ blends RRT's tree efficiency with PRM's cost optimality goals.
\item Samples are connected via dynamic-programming style cost-of-arrival updates.
\item Lazy collision handling skips invalid edges without reprocessing neighbors.
\end{itemize}
\pause
\begin{center}
    \includegraphics[width=0.7\linewidth]{../tex/figs/ch06_figs/fmt.png}
\end{center}
\end{frame}
% --- Original text start ---
% FMT* description plus figure narrative.
% --- Original text end ---

\begin{frame}{Dynamic Programming Connection Rule}
\begin{itemize}
\item Maintain arrival cost $c(u)$ for explored nodes and candidate cost for sample $v$.
\end{itemize}
\pause
\begin{equation*}
    c(v) = \min_{u: \lVert u - v \rVert < r_n} \text{Cost}(u,v) + c(u)
\end{equation*}
\pause
\begin{itemize}
\item Accept the best parent, then run a collision test; reject $v$ if the edge fails.
\item Lazy evaluation keeps collision checks proportional to promising edges only.
\end{itemize}
\end{frame}
% --- Original text start ---
% Equation (\ref{eq:bellmancost}) and accompanying explanation of lazy DP in FMT*.
% --- Original text end ---

\begin{frame}{Kinodynamic Planning Basics}
\begin{itemize}
\item Some robots cannot traverse straight lines in $C$; dynamics impose $\dot{\x} = f(\x,\bu)$.
\item Planning must occur in state space $X$ with admissible controls $\bu \in U$.
\end{itemize}
\pause
\begin{equation*}
    \dot{\x} = f(\x, \bu)
\end{equation*}
\pause
\begin{itemize}
\item State-to-configuration mapping ensures collision checking still applies.
\end{itemize}
\end{frame}
% --- Original text start ---
% Kinodynamic planning section introducing dynamics model.
% --- Original text end ---

\begin{frame}{Forward vs Steering Extensions}
\begin{itemize}
\item Forward propagation: sample control $u$ and duration $t$, integrate dynamics, and test the new edge.
\item Steering-based: call a dedicated solver to connect $x$ and $x'$ with a feasible trajectory.
\item Differentially flat systems often admit efficient steering functions.
\end{itemize}
\pause
\begin{center}
    \includegraphics[width=0.45\linewidth]{../tex/figs/ch06_figs/driftless.png}\hspace{1em}
    \includegraphics[width=0.45\linewidth]{../tex/figs/ch06_figs/drift.png}
\end{center}
\end{frame}
% --- Original text start ---
% Description of DFMT* results and kinodynamic figures.
% --- Original text end ---

\begin{frame}{Deterministic Sampling Motivation}
\begin{itemize}
\item Randomness complicates certification and reuse in safety-critical deployments.
\item Low-dispersion deterministic sequences can replace i.i.d. samples.
\item Theoretical guarantees mirror probabilistic planners while shrinking required radii $r_n$.
\end{itemize}
\end{frame}
% --- Original text start ---
% Section on deterministic sampling-based motion planning motivations.
% --- Original text end ---

\begin{frame}{$l_2$-Dispersion Metric}
\begin{itemize}
\item Measures the largest empty ball that fits between samples in $S \subset X$.
\end{itemize}
\pause
\begin{equation*}
    D(S) := \sup_{x \in X} \min_{s \in S} \lVert s - x \rVert_2
\end{equation*}
\pause
\begin{itemize}
\item Sequences with $D(S) \leq \gamma n^{-1/d}$ ensure returned paths converge to optimal length.
\end{itemize}
\end{frame}
% --- Original text start ---
% Definition of l2-dispersion and convergence statement for deterministic sampling.
% --- Original text end ---

\begin{frame}{Exercises}
\begin{itemize}
\item Problem 2: implement geometric and Dubins RRT planners.
\item Problem 3: combine A* planning with tracking for a unicycle robot.
\item Extra credit: build a bi-directional RRT-Connect variant.
\item Repository: \url{https://github.com/PrinciplesofRobotAutonomy/AA274A_HW2}
\end{itemize}
\end{frame}
% --- Original text start ---
% Exercise listings for Chapter 06.
% --- Original text end ---

\end{document}
