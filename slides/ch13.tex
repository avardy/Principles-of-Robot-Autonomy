\documentclass{beamer}

% Shared Beamer configuration for Principles of Robot Autonomy decks
\usetheme{Boadilla}
%\usecolortheme{seagull}

% Use Palatino to match the book's Tufte style
\usepackage{mathpazo}
\usefonttheme{serif}

\usepackage{amsmath}
\usepackage{amssymb}
\usepackage{bm}
\usepackage{graphicx}
\usepackage{enumitem}
\usepackage{xparse}
\usepackage{array}

% Increase vertical spacing between list items for readability.
\setlist[itemize]{itemsep=0.75em}
\setlist[enumerate]{itemsep=0.75em}

% Define bold symbol macros if not already defined
\providecommand{\x}{\boldsymbol{x}}
\providecommand{\z}{\boldsymbol{z}}
\providecommand{\q}{\boldsymbol{q}}
\providecommand{\bxi}{\boldsymbol{\xi}}
\providecommand{\bu}{\boldsymbol{u}}
\providecommand{\bmu}{\boldsymbol{\mu}}
\providecommand{\bSigma}{\boldsymbol{\Sigma}}
\providecommand{\y}{\boldsymbol{y}}
\providecommand{\m}{\mathcal{M}}
\providecommand{\bc}{\boldsymbol{c}}
\providecommand{\R}{\mathbb{R}}

% Animated lists by default, with override via optional overlay argument.
\beamerdefaultoverlayspecification{<+->}

% Paragraphs reveal sequentially by default.
\AtBeginDocument{%
  \ifx\paragraph\undefined
    \newcommand{\paragraph}[1]{\textbf{#1}\par\onslide<+->{} }
  \else
    \let\origparagraph\paragraph
    \renewcommand{\paragraph}[1]{\origparagraph{#1}\par\onslide<+->{} }
  \fi
}

% Add other shared packages or theme settings above this line as needed.

\author[COMP 4766]{Adapted from ``Principles of Robot Autonomy'' by D. Gammelli, J. Lorenzetti,
K. Luo, G. Zardini, and M. Pavone}
\date{\today}


\title{Chapter 13 - Localization and Filtering}

\begin{document}

\frame{\titlepage}

\begin{frame}{Why Localization Matters}
\begin{itemize}
\item Local sensing (lidar, cameras) is inherently egocentric; autonomy needs global context.
\item Robots must answer "where am I?" and "have I seen this before?" to plan reliably.
\item Localization and mapping bridge local measurements to global understanding.
\end{itemize}
\end{frame}
% --- Original text start ---
% Opening motivation on global information needs and localization/mapping definition.
% --- Original text end ---

\begin{frame}{Map-Based Localization}
\begin{itemize}
\item Map representation plus belief representation jointly support navigation.
\item Example: floor plan navigation requires knowing the current room before routing.
\item Behavioral hacks (wall-following) can work, but maps enable deliberate planning.
\end{itemize}
\pause
\begin{center}
    \includegraphics[width=0.8\linewidth]{../tex/figs/ch13_figs/map.png}
\end{center}
\end{frame}
% --- Original text start ---
% Discussion of Figure map and map-based vs behavioral approaches.
% --- Original text end ---

\begin{frame}{Belief Representations}
\begin{itemize}
\item Single best-guess poses ignore uncertainty; probabilistic beliefs capture it.
\item Choose between single vs multi-hypothesis and continuous vs discrete forms.
\item Expressiveness trades off with computational load for downstream filters.
\end{itemize}
\pause
\begin{center}
    \includegraphics[width=0.6\linewidth]{../tex/figs/ch13_figs/distributions.png}
\end{center}
\end{frame}
% --- Original text start ---
% Figure on probabilistic representations and surrounding explanation.
% --- Original text end ---

\begin{frame}{Random Variables Refresher}
\begin{itemize}
\item Discrete RV $X$ uses pmf $p(x)$ with $\sum_x p(x)=1$ (coin flips, grid poses).
\item Continuous RV uses pdf $p(x)$ with $\int p(x)dx=1$ (planar position, heading).
\item Measurements, states, and environment parameters all fit into this RV framework.
\end{itemize}
\end{frame}
% --- Original text start ---
% Definitions of discrete and continuous random variables with examples.
% --- Original text end ---

\begin{frame}{Joint, Conditional, Independence}
\begin{itemize}
\item Joint distribution $p(x,y)$ captures simultaneous hypotheses for two RVs.
\item Independence: $p(x,y)=p(x)p(y)$; conditional independence inserts a third RV.
\item Conditionals reweight beliefs: $p(x\mid y)=\frac{p(x,y)}{p(y)}$.
\end{itemize}
\end{frame}
% --- Original text start ---
% Joint distribution, independence, conditional probability, conditional independence sections.
% --- Original text end ---

\begin{frame}{Total Probability and Bayes}
\begin{itemize}
\item Law of total probability marginalizes over hidden variables.
\item Bayes' rule swaps conditionals to fuse new evidence with priors.
\end{itemize}
\pause
\begin{align*}
 p(x) &= \sum_y p(x\mid y)p(y) \\
 p(x\mid y) &= \frac{p(y\mid x)p(x)}{p(y)}
\end{align*}
\end{frame}
% --- Original text start ---
% Sections on law of total probability and Bayes' rule including equations.
% --- Original text end ---

\begin{frame}{Moments: Expectation and Covariance}
\begin{itemize}
\item Expectation $E[X]$ is the first moment (mean) for discrete or continuous RVs.
\item Linearity: $E[aX+b]=aE[X]+b$ simplifies many derivations.
\item Covariance $\text{cov}(X,Y)=E[XY^\top]-E[X]E[Y]^\top$ links co-variation.
\end{itemize}
\end{frame}
% --- Original text start ---
% Expectation and covariance definitions and intuition.
% --- Original text end ---

\begin{frame}{Markov Model Ingredients}
\begin{itemize}
\item State $\x_t$ may include robot pose plus salient map features.
\item Controls $\bu_t$ advance the state; measurements $\z_t$ reveal environment cues.
\item Use sequences $\x_{t_1:t_n}$, $\z_{t_1:t_n}$, $\bu_{t_1:t_n}$ to denote histories.
\end{itemize}
\end{frame}
% --- Original text start ---
% Section on states, measurements, and controls notation.
% --- Original text end ---

\begin{frame}{Probabilistic Dynamics}
\begin{itemize}
\item General model: $p(\x_t\mid \x_{0:t-1},\z_{1:t-1},\bu_{1:t})$ with measurement $p(\z_t\mid \x_{0:t},\cdot)$.
\item If state is complete (Markov), transition simplifies to $p(\x_t\mid \x_{t-1},\bu_t)$.
\item Measurement model often collapses to $p(\z_t\mid \x_t)$.
\end{itemize}
\end{frame}
% --- Original text start ---
% General probabilistic model equations and Markov property simplifications.
% --- Original text end ---

\begin{frame}{Hidden Markov Model}
\begin{itemize}
\item Bayes network links prior state, control, next state, and measurement.
\item Execution order: apply control then sense, matching most real robots.
\item Graph makes conditional independencies explicit for inference.
\end{itemize}
\pause
\begin{center}
    \includegraphics[width=0.55\linewidth]{../tex/figs/ch13_figs/hmm.png}
\end{center}
\end{frame}
% --- Original text start ---
% Figure hmm and explanation.
% --- Original text end ---

\begin{frame}{Belief and Prediction}
\begin{itemize}
\item Belief: $bel(\x_t)=p(\x_t\mid \z_{1:t},\bu_{1:t})$ captures posterior state pdf.
\item Prediction: $\overline{bel}(\x_t)=p(\x_t\mid \z_{1:t-1},\bu_{1:t})$ omits current measurement.
\item Filtering alternates prediction (motion update) and correction (measurement update).
\end{itemize}
\end{frame}
% --- Original text start ---
% Definitions of belief and prediction distributions.
% --- Original text end ---

\begin{frame}{Bayes Filter Steps}
\begin{itemize}
\item Prediction: propagate $bel(\x_{t-1})$ through motion model via $\overline{bel}(\x_t)$.
\item Correction: weight prediction by likelihood $p(\z_t\mid \x_t)$ and renormalize.
\item Recursive structure enables online localization with streaming data.
\end{itemize}
\end{frame}
% --- Original text start ---
% Bayes filter algorithm overview.
% --- Original text end ---

\begin{frame}{Deriving the Update}
\begin{itemize}
\item Apply Bayes' rule to $bel(\x_t)$ to expose likelihood and prediction terms.
\item Use law of total probability to integrate over $\x_{t-1}$ for $\overline{bel}(\x_t)$.
\item Markov property removes historical dependencies, yielding tractable recursions.
\end{itemize}
\end{frame}
% --- Original text start ---
% Derivation section for prediction and correction equations.
% --- Original text end ---

\begin{frame}{Implementation Notes}
\begin{itemize}
\item Integrals in the continuous Bayes filter are rarely closed-form.
\item Numerical quadrature or sampling may be too costly for high-dimensional states.
\item Practical filters (EKF, particle, histogram) approximate the same recursion.
\end{itemize}
\end{frame}
% --- Original text start ---
% Practical considerations paragraph.
% --- Original text end ---

\begin{frame}{Discrete Bayes Filter}
\begin{itemize}
\item Finite state spaces replace integrals with sums over hypothesis bins.
\item Belief stored as probabilities $\{p_{k,t}\}$; prediction sums over transitions.
\item Still needs normalization constant $\eta$ after weighting by measurement likelihood.
\end{itemize}
\end{frame}
% --- Original text start ---
% Discrete Bayes filter algorithm description.
% --- Original text end ---

\begin{frame}{Reminder: Filtering Essentials}
\begingroup
\setbeamercolor{block title}{bg=gray!20,fg=black}
\setbeamercolor{block body}{bg=gray!10,fg=black}
\begin{block}{Key Takeaways}
\begin{itemize}
\item Represent localization beliefs probabilistically to capture multi-modal uncertainty.
\item Hidden Markov models + Bayes filter recursion underpin most localization stacks.
\item Choose approximations (continuous, discrete, sample-based) that balance fidelity and compute.
\end{itemize}
\end{block}
\endgroup
\end{frame}
% --- Original text start ---
% Summary tying localization motivation, Bayes filter, and discrete variant.
% --- Original text end ---

\end{document}
