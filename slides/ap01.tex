\documentclass{beamer}

% Shared Beamer configuration for Principles of Robot Autonomy decks
\usetheme{Boadilla}
%\usecolortheme{seagull}

% Use Palatino to match the book's Tufte style
\usepackage{mathpazo}
\usefonttheme{serif}

\usepackage{amsmath}
\usepackage{amssymb}
\usepackage{bm}
\usepackage{graphicx}
\usepackage{enumitem}
\usepackage{xparse}
\usepackage{array}

% Increase vertical spacing between list items for readability.
\setlist[itemize]{itemsep=0.75em}
\setlist[enumerate]{itemsep=0.75em}

% Define bold symbol macros if not already defined
\providecommand{\x}{\boldsymbol{x}}
\providecommand{\z}{\boldsymbol{z}}
\providecommand{\q}{\boldsymbol{q}}
\providecommand{\bxi}{\boldsymbol{\xi}}
\providecommand{\bu}{\boldsymbol{u}}
\providecommand{\bmu}{\boldsymbol{\mu}}
\providecommand{\bSigma}{\boldsymbol{\Sigma}}
\providecommand{\y}{\boldsymbol{y}}
\providecommand{\m}{\mathcal{M}}
\providecommand{\bc}{\boldsymbol{c}}
\providecommand{\R}{\mathbb{R}}

% Animated lists by default, with override via optional overlay argument.
\beamerdefaultoverlayspecification{<+->}

% Paragraphs reveal sequentially by default.
\AtBeginDocument{%
  \ifx\paragraph\undefined
    \newcommand{\paragraph}[1]{\textbf{#1}\par\onslide<+->{} }
  \else
    \let\origparagraph\paragraph
    \renewcommand{\paragraph}[1]{\origparagraph{#1}\par\onslide<+->{} }
  \fi
}

% Add other shared packages or theme settings above this line as needed.

\author[COMP 4766]{Adapted from ``Principles of Robot Autonomy'' by D. Gammelli, J. Lorenzetti,
K. Luo, G. Zardini, and M. Pavone}
\date{\today}


\title{Appendix 01 - Machine Learning Primer}

\begin{document}

\begin{frame}[plain]
\titlepage
\end{frame}

\begin{frame}{Why Lean on Learning?}
\begin{itemize}
\item First-principles models struggle with occlusions, weather, human intent, or combinatorial object variation.
\item Data-driven methods capture patterns from experiments or simulation when analytic models fall short.
\item Modern autonomy blends physics insight with learning pipelines for perception, control, and decision making.
\end{itemize}
\end{frame}
% --- Original text start ---
% Intro paragraphs motivating ML for autonomy when first-principles models break down.
% --- Original text end ---

\begin{frame}{Problem Taxonomy}
\begin{itemize}
\item \textbf{Supervised}: learn $y = f(x)$ from labeled pairs $\{(x_i, y_i)\}_{i=1}^n$ for regression or classification.
\item \textbf{Unsupervised}: seek patterns or structure directly from unlabeled samples $\{x_i\}$.
\item Robotics heavily favors supervised imitation or perception tasks but still borrows unsupervised tools for features.
\end{itemize}
\end{frame}
% --- Original text start ---
% Definitions of supervised and unsupervised learning plus regression/classification discussion.
% --- Original text end ---

\begin{frame}{Choosing Function Classes}
\begin{itemize}
\item Parametric models (e.g., $f(x)=Wx$ or neural nets) expose finite parameters and analytical forms.
\item Non-parametric methods (e.g., $k$-NN) store data and answer via local neighborhoods or kernels.
\item Practical choice hinges on data volume, desired inductive bias, and computation budget.
\end{itemize}
\end{frame}
% --- Original text start ---
% Paragraph describing parametric versus non-parametric models and examples.
% --- Original text end ---

\begin{frame}{Regression Losses}
\begin{itemize}
\item $\ell_2$: $L = \tfrac{1}{n}\sum_{i=1}^n (f(x_i) - y_i)^2$ favors small average residuals but is outlier-sensitive.
\item $\ell_1$: $L = \tfrac{1}{n}\sum_{i=1}^n \lvert f(x_i) - y_i \rvert$ treats errors uniformly and is robust.
\item Choice encodes tolerance to noise vs. outliers and shapes the optimization landscape.
\end{itemize}
\end{frame}
% --- Original text start ---
% Loss section describing l2 and l1 objectives for regression.
% --- Original text end ---

\begin{frame}{Classification Losses}
\begin{itemize}
\item $0$-$1$ loss counts misclassifications: $L = \tfrac{1}{n}\sum_{i=1}^n \mathbf{1}\{f(x_i) \neq y_i\}$ (non-differentiable).
\item Cross entropy: $L = -\tfrac{1}{n}\sum_{i=1}^n y_i^\top \log f(x_i)$, encourages confident, correct probabilities.
\item Softmax-normalized outputs plus cross entropy give smooth gradients for large-class problems.
\end{itemize}
\end{frame}
% --- Original text start ---
% Paragraph on classification losses, indicator loss, and cross-entropy interpretation.
% --- Original text end ---

\begin{frame}{Model Training Basics}
\begin{itemize}
\item Fix a parametric template $f_\theta$ and minimize chosen loss to fit $\theta$.
\item Analytical solutions exist in special cases like linear least squares with $\theta^* = (X^\top X)^{-1}X^\top Y$.
\item Most robotics-grade models require iterative numerical optimization.
\end{itemize}
\end{frame}
% --- Original text start ---
% Linear least squares example and discussion of analytic vs numerical training.
% --- Original text end ---

\begin{frame}{Gradient and Stochastic Descent}
\begin{itemize}
\item Gradient descent updates $\theta \leftarrow \theta - \eta \nabla_\theta L$, using full dataset gradients.
\item Stochastic gradient descent samples a batch $S$ to estimate the gradient cheaply each iteration.
\item Learning rate, batch size, and variance reduction tricks govern convergence speed and stability.
\end{itemize}
\end{frame}
% --- Original text start ---
% Section on numerical optimization, gradient descent, and stochastic approximation.
% --- Original text end ---

\begin{frame}{Generalization and Regularization}
\begin{itemize}
\item Split data into training/test folds to detect overfitting when training loss $\ll$ test loss.
\item Add $\ell_2$ or $\ell_1$ penalties on $\theta$ to tame model complexity during training.
\item Other tools (dropout, larger datasets) further promote models that transfer to unseen scenarios.
\end{itemize}
\end{frame}
% --- Original text start ---
% Section on training/test splits, overfitting diagnosis, and regularization techniques including dropout.
% --- Original text end ---

\begin{frame}{Neural Network Architecture}
\begin{itemize}
\item Compose layers $h_{k} = f_k(W_k h_{k-1} + b_k)$ ending with \(\hat{y} = f_K(W_K h_{K-1} + b_K)\).
\item Depth, hidden dimension, and activation selection set model capacity.
\item Nonlinear activations keep the stack expressive; differentiability keeps gradients tractable.
\end{itemize}
\end{frame}
% --- Original text start ---
% Neural network structure equations and discussion of parameters, layers, and activations.
% --- Original text end ---

\begin{frame}{Activation Functions}
\begin{itemize}
\item Sigmoid, tanh, ReLU, and leaky ReLU balance smoothness, saturation, and sparse gradients.
\item All are easy to differentiate (or sub-differentiate) for backprop.
\item Selection trades vanishing gradients vs. nonlinearity and computational cost.
\end{itemize}
\begin{center}
    \includegraphics[width=0.75\linewidth]{../book/figs/app01_figs/activations.png}
\end{center}
\end{frame}
% --- Original text start ---
% Activation-function descriptions and Figure activations.
% --- Original text end ---

\begin{frame}{Training Neural Networks}
\begin{itemize}
\item Backpropagation applies chain rule through layered computations to get $\tfrac{\partial L}{\partial \theta}$ efficiently.
\item SGD variants (momentum, Adam, etc.) scale to millions of parameters.
\item Regularization and dropout combat overfitting despite flexible function classes.
\end{itemize}
\end{frame}
% --- Original text start ---
% Sections on training neural networks, backpropagation overview, and regularization/dropout.
% --- Original text end ---

\begin{frame}{Backpropagation as Dynamic Programming}
\begin{itemize}
\item Treat computation as a graph; forward pass caches intermediate values.
\item Reverse-mode differentiation reuses partial derivatives to avoid redundant work.
\item View aligns with auto-diff tools such as PyTorch or TensorFlow used in robotics stacks.
\end{itemize}
\begin{center}
    \includegraphics[width=0.58\linewidth]{../book/figs/app01_figs/backprop.png}
\end{center}
\end{frame}
% --- Original text start ---
% Backpropagation and computational graph section plus Figure backprop.
% --- Original text end ---

\begin{frame}{Example: Scalar Network}
\begin{itemize}
\item Toy model $f(x) = (x+a)(x+b)$ with $\ell_2$ loss showcases backprop savings.
\item Forward pass computes $z_1, z_2, z_3$ once; backward pass reuses gradients to get $\partial L/\partial a, \partial L/\partial b$.
\item Cuts operations from 14 (naive) to 9, illustrating why backprop matters even for tiny graphs.
\end{itemize}
\begin{center}
    \includegraphics[width=0.6\linewidth]{../book/figs/app01_figs/graph.png}
\end{center}
\end{frame}
% --- Original text start ---
% Worked example and computational graph figure illustrating backprop efficiency.
% --- Original text end ---

\begin{frame}{Reminder: ML Foundations}
\begingroup
\setbeamercolor{block title}{bg=gray!20,fg=black}
\setbeamercolor{block body}{bg=gray!10,fg=black}
\begin{block}{Key Takeaways}
\begin{itemize}
\item Match supervised/unsupervised formulations and loss choices to the robotics task at hand.
\item Train with optimization methods that scale, then verify generalization via splits and regularizers.
\item Neural nets plus backprop provide expressive models—keep them grounded with good data and structure.
\end{itemize}
\end{block}
\endgroup
\end{frame}
% --- Original text start ---
% Appendix summary tying task taxonomy, loss/optimization, and neural network training guidance.
% --- Original text end ---

\end{document}
