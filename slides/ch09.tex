\documentclass{beamer}

% Shared Beamer configuration for Principles of Robot Autonomy decks
\usetheme{Boadilla}
%\usecolortheme{seagull}

% Use Palatino to match the book's Tufte style
\usepackage{mathpazo}
\usefonttheme{serif}

\usepackage{amsmath}
\usepackage{amssymb}
\usepackage{bm}
\usepackage{graphicx}
\usepackage{enumitem}
\usepackage{xparse}
\usepackage{array}

% Increase vertical spacing between list items for readability.
\setlist[itemize]{itemsep=0.75em}
\setlist[enumerate]{itemsep=0.75em}

% Define bold symbol macros if not already defined
\providecommand{\x}{\boldsymbol{x}}
\providecommand{\z}{\boldsymbol{z}}
\providecommand{\q}{\boldsymbol{q}}
\providecommand{\bxi}{\boldsymbol{\xi}}
\providecommand{\bu}{\boldsymbol{u}}
\providecommand{\bmu}{\boldsymbol{\mu}}
\providecommand{\bSigma}{\boldsymbol{\Sigma}}
\providecommand{\y}{\boldsymbol{y}}
\providecommand{\m}{\mathcal{M}}
\providecommand{\bc}{\boldsymbol{c}}
\providecommand{\R}{\mathbb{R}}

% Animated lists by default, with override via optional overlay argument.
\beamerdefaultoverlayspecification{<+->}

% Paragraphs reveal sequentially by default.
\AtBeginDocument{%
  \ifx\paragraph\undefined
    \newcommand{\paragraph}[1]{\textbf{#1}\par\onslide<+->{} }
  \else
    \let\origparagraph\paragraph
    \renewcommand{\paragraph}[1]{\origparagraph{#1}\par\onslide<+->{} }
  \fi
}

% Add other shared packages or theme settings above this line as needed.

\author[COMP 4766]{Adapted from ``Principles of Robot Autonomy'' by D. Gammelli, J. Lorenzetti,
K. Luo, G. Zardini, and M. Pavone}
\date{\today}


\title{Chapter 09 - Stereo Vision and SFM}

\begin{document}

\frame{\titlepage}

\begin{frame}{Why Multiple Views?}
\begin{itemize}
\item Single calibrated images cannot recover depth without extra assumptions.
\item Stereo and structure-from-motion (SFM) exploit multiple viewpoints to infer 3D structure.
\item Cameras offer dense, low-cost measurements that complement active ranging sensors.
\end{itemize}
\end{frame}
% --- Original text start ---
% Chapter intro motivating stereo vision and structure-from-motion for 3D recovery.
% --- Original text end ---

\begin{frame}{Stereo Pipeline}
\begin{itemize}
\item Acquire synchronized images from two cameras with known baselines.
\item Fuse images by finding correspondences, then reconstruct depth via triangulation.
\item Disparity (horizontal shift) encodes range: larger disparity means closer objects.
\end{itemize}
\end{frame}
% --- Original text start ---
% Overview of stereo vision steps (fusion and reconstruction) and disparity intuition.
% --- Original text end ---

\begin{frame}{Epipolar Geometry}
\begin{itemize}
\item Points $P$, $p$, $p'$ and camera centers form an epipolar plane.
\item Each image point lies on an epipolar line through its epipole; reduce correspondence search to 1D.
\item Fundamental matrix $F$ encodes this constraint: $p^\top F p' = 0$.
\end{itemize}
\pause
\begin{center}
    \includegraphics[width=0.7\linewidth]{../tex/figs/ch09_figs/stereo.png}
\end{center}
\end{frame}
% --- Original text start ---
% Epipolar constraint explanation and Figure \ref{fig:epi}.
% --- Original text end ---

\begin{frame}{Fundamental Matrix Details}
\begin{itemize}
\item $F = K^{-\top} E K'^{-1}$ with $E = [t]_\times R$ captures relative pose and intrinsics.
\item Estimate $F$ from $n \geq 8$ correspondences by solving $Wf = 0$ under $\lVert f \rVert = 1$.
\item Enforce rank-2 by SVD projection to nearest singular matrix.
\end{itemize}
\end{frame}
% --- Original text start ---
% Fundamental matrix expression and least-squares estimation (Eq. \ref{eq:fopt}).
% --- Original text end ---

\begin{frame}{Image Rectification}
\begin{itemize}
\item Warp images so epipolar lines become horizontal and aligned.
\item Reduces correspondence search to line scans and simplifies disparity computation.
\item Conceptually equivalent to rotating cameras about their optical centers.
\end{itemize}
\pause
\begin{center}
    \includegraphics[width=0.65\linewidth]{../tex/figs/ch09_figs/rectification.png}
\end{center}
\end{frame}
% --- Original text start ---
% Image rectification discussion and Figure \ref{fig:rect}.
% --- Original text end ---

\begin{frame}{Correspondence Challenges}
\begin{itemize}
\item Occlusions hide features from one camera; repetitive textures confuse matching.
\item Photometric differences or lens distortions degrade similarity metrics.
\item Robust stereo stacks priors (smoothness, uniqueness) with epipolar constraints.
\end{itemize}
\end{frame}
% --- Original text start ---
% Paragraph on correspondence problem difficulties (occlusions, repetitive patterns, distortions).
% --- Original text end ---

\begin{frame}{Triangulation and Disparity}
\begin{itemize}
\item With rectified images and baseline $b$, depth follows $z = \frac{bf}{p_u - p'_u}$.
\item Small disparity implies far objects; large disparity marks nearby geometry.
\item Baseline must balance accuracy (larger $b$) against visibility overlap.
\end{itemize}
\pause
\begin{center}
    \includegraphics[width=0.74\linewidth]{../tex/figs/ch09_figs/triangulation.png}
\end{center}
\end{frame}
% --- Original text start ---
% Triangulation derivation (Figure \ref{fig:recttri}) and disparity definition.
% --- Original text end ---

\begin{frame}{Disparity Maps}
\begin{itemize}
\item Compute pixel-wise disparity to create a dense depth proxy.
\item Bright regions correspond to large disparities (close objects); dark to distant geometry.
\item Holes reveal occlusions where no correspondence exists.
\end{itemize}
\pause
\begin{center}
    \includegraphics[width=0.8\linewidth]{../tex/figs/ch09_figs/disparitymap.png}
\end{center}
\end{frame}
% --- Original text start ---
% Figure \ref{fig:disparity} description of disparity map interpretation.
% --- Original text end ---

\begin{frame}{Structure From Motion (SFM)}
\begin{itemize}
\item Single moving camera captures multiple views; intrinsics stay fixed, extrinsics vary per frame.
\item Need to solve for both 3D points $P_j$ and camera matrices $M_k$ from $p_{j,k}^h = M_k P_j^h$.
\item Scale remains ambiguous without extra cues; only relative structure is recovered.
\end{itemize}
\pause
\begin{center}
    \includegraphics[width=0.65\linewidth]{../tex/figs/ch09_figs/sfm.png}
\end{center}
\end{frame}
% --- Original text start ---
% SFM overview and Figure \ref{sfm}.
% --- Original text end ---

\begin{frame}{Applications \\ Visual Odometry}
\begin{itemize}
\item Visual odometry estimates camera motion by chaining SFM over time.
\item Common on planetary rovers where wheel slip breaks encoder-only odometry.
\item Depth + motion estimates enable mapping, localization, and obstacle avoidance.
\end{itemize}
\end{frame}
% --- Original text start ---
% Section on visual odometry applications of SFM.
% --- Original text end ---

\begin{frame}{Reminder: Stereo/SFM Checklist}
\begingroup
\setbeamercolor{block title}{bg=gray!20,fg=black}
\setbeamercolor{block body}{bg=gray!10,fg=black}
\begin{block}{Concept Recap}
\begin{itemize}
\item Calibrate both cameras (intrinsics + extrinsics) before enforcing $p^\top F p' = 0$.
\item Rectify images to simplify matching, then validate depth via disparity maps.
\item For SFM/visual odometry, monitor scale drift and fuse extra sensors when available.
\end{itemize}
\end{block}
\endgroup
\end{frame}
% --- Original text start ---
% Key takeaways from stereo and SFM sections.
% --- Original text end ---

\end{document}
