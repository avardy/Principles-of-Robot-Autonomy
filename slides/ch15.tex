\documentclass{beamer}

% Shared Beamer configuration for Principles of Robot Autonomy decks
\usetheme{Boadilla}
%\usecolortheme{seagull}

% Use Palatino to match the book's Tufte style
\usepackage{mathpazo}
\usefonttheme{serif}

\usepackage{amsmath}
\usepackage{amssymb}
\usepackage{bm}
\usepackage{graphicx}
\usepackage{enumitem}
\usepackage{xparse}
\usepackage{array}

% Increase vertical spacing between list items for readability.
\setlist[itemize]{itemsep=0.75em}
\setlist[enumerate]{itemsep=0.75em}

% Define bold symbol macros if not already defined
\providecommand{\x}{\boldsymbol{x}}
\providecommand{\z}{\boldsymbol{z}}
\providecommand{\q}{\boldsymbol{q}}
\providecommand{\bxi}{\boldsymbol{\xi}}
\providecommand{\bu}{\boldsymbol{u}}
\providecommand{\bmu}{\boldsymbol{\mu}}
\providecommand{\bSigma}{\boldsymbol{\Sigma}}
\providecommand{\y}{\boldsymbol{y}}
\providecommand{\m}{\mathcal{M}}
\providecommand{\bc}{\boldsymbol{c}}
\providecommand{\R}{\mathbb{R}}

% Animated lists by default, with override via optional overlay argument.
\beamerdefaultoverlayspecification{<+->}

% Paragraphs reveal sequentially by default.
\AtBeginDocument{%
  \ifx\paragraph\undefined
    \newcommand{\paragraph}[1]{\textbf{#1}\par\onslide<+->{} }
  \else
    \let\origparagraph\paragraph
    \renewcommand{\paragraph}[1]{\origparagraph{#1}\par\onslide<+->{} }
  \fi
}

% Add other shared packages or theme settings above this line as needed.

\author[COMP 4766]{Adapted from ``Principles of Robot Autonomy'' by D. Gammelli, J. Lorenzetti,
K. Luo, G. Zardini, and M. Pavone}
\date{\today}


\title{Chapter 15 - Nonparametric Filters}

\begin{document}

\begin{frame}[plain]
\titlepage
\end{frame}

\begin{frame}{Why Nonparametric?}
\begin{itemize}
\item Gaussian (EKF/UKF) beliefs are efficient but enforce unimodal structure.
\item Some tasks need multiple hypotheses or sharp discontinuities in belief.
\item Nonparametric filters keep Bayes recursion but drop restrictive priors.
\end{itemize}
\end{frame}
% --- Original text start ---
% Motivation for nonparametric filters vs parametric Gaussian filters.
% --- Original text end ---

\begin{frame}{Families of Nonparametric Filters}
\begin{itemize}
\item Histogram filter quantizes continuous state into finitely many bins.
\item Particle filter represents belief with a cloud of weighted samples.
\item Both approximate $bel(\x_t)$ without assuming shape.
\end{itemize}
\end{frame}
% --- Original text start ---
% Chapter overview introducing histogram and particle filters.
% --- Original text end ---

\begin{frame}{Histogram Filter: State Binning}
\begin{itemize}
\item Decompose domain into bins $\x_{1,t}\cup\cdots\cup \x_{K,t}=\text{dom}(X_t)$.
\item Assign probability masses $p_{k,t}$; within each bin density is constant.
\item Mean representative state $\hat{\x}_{k,t}=|\x_{k,t}|^{-1}\int_{\x_{k,t}}x_t d\x_t$ drives updates.
\end{itemize}
\pause
\begin{equation*}
 p(\x_t)=\frac{p_{k,t}}{|\x_{k,t}|}, \quad \x_t\in \x_{k,t}
\end{equation*}
\end{frame}
% --- Original text start ---
% Histogram filter discretization equations.
% --- Original text end ---

\begin{frame}{Histogram Prediction and Update}
\begin{itemize}
\item Transition: $p(\x_{k,t}\mid \bu_t,\x_{i,t-1})\approx \eta |\x_{k,t}| p(\hat{\x}_{k,t}\mid \bu_t,\hat{\x}_{i,t-1})$.
\item Measurement: $p(\z_t\mid \x_{k,t})\approx p(\z_t\mid \hat{\x}_{k,t})$.
\item Apply discrete Bayes filter to evolve $p_{k,t}$ over bins.
\end{itemize}
\end{frame}
% --- Original text start ---
% Transition and measurement approximation plus pointer to discrete Bayes filter.
% --- Original text end ---

\begin{frame}{Histogram Trade-offs}
\begin{itemize}
\item Resolution vs.
 computation: fine bins capture detail but explode state count.
\item Binning error grows when dynamics or sensors vary sharply within bins.
\item Works best for low-dimensional states (e.g., 1D range-only robots).
\end{itemize}
\end{frame}
% --- Original text start ---
% Discussion on expressiveness and computational complexity.
% --- Original text end ---

\begin{frame}{Particle Filter Intuition}
\begin{itemize}
\item Represent belief by particles $\mathcal{X}_t=\{\x_t^{[1]},\ldots,\x_t^{[M]}\}$.
\item Dense particle regions indicate high posterior probability.
\item Exact only as $M\to \infty$, but $\mathcal{O}(10^3)$ samples often suffice.
\end{itemize}
\end{frame}
% --- Original text start ---
% Definition of particle set and approximate belief representation.
% --- Original text end ---

\begin{frame}{Prediction and Weighing}
\begin{itemize}
\item For each prior particle, sample $\bar{\x}_t^{[m]} \sim p(\x_t\mid \bu_t, \x_{t-1}^{[m]})$.
\item Compute weight $w_t^{[m]}=p(\z_t\mid \bar{\x}_t^{[m]})$ from measurement likelihood.
\item Weighted set $\bar{\mathcal{X}}_t$ approximates $\overline{bel}(\x_t)$.
\end{itemize}
\end{frame}
% --- Original text start ---
% Particle propagation and importance weighting description.
% --- Original text end ---

\begin{frame}{Resampling and Survival}
\begin{itemize}
\item Draw $M$ particles from $\bar{\mathcal{X}}_t$ proportionally to $w_t^{[m]}$.
\item High-likelihood regions keep more "offspring"; others fade away.
\item Resampling prevents degeneracy and focuses compute where it matters.
\end{itemize}
\pause
\begin{center}
    \includegraphics[width=0.7\linewidth]{../book/figs/ch15_figs/particlefilter.png}
\end{center}
\end{frame}
% --- Original text start ---
% Algorithm \ref{alg:particle} description and Figure \ref{fig:Particle_filter}.
% --- Original text end ---

\begin{frame}{Exercises}
\begin{itemize}
\item HW4 extra credit: Monte Carlo Localization with line features.
\item Implement motion sampling, measurement likelihoods, and resampling loop.
\item Repository: \url{https://github.com/PrinciplesofRobotAutonomy/AA274A_HW4}.
\end{itemize}
\end{frame}
% --- Original text start ---
% Exercise description referencing AA274A_HW4 particle filter task.
% --- Original text end ---

\begin{frame}{Reminder: Nonparametric Stack}
\begingroup
\setbeamercolor{block title}{bg=gray!20,fg=black}
\setbeamercolor{block body}{bg=gray!10,fg=black}
\begin{block}{Key Takeaways}
\begin{itemize}
\item Histogram filters discretize space; particles sample it—both avoid strict shapes.
\item Always apply Bayes prediction + correction even when beliefs are sample-based.
\item Resampling or grid resolution choices set the accuracy vs.
 runtime balance.
\end{itemize}
\end{block}
\endgroup
\end{frame}
% --- Original text start ---
% Summary of histogram and particle filters and practical tuning.
% --- Original text end ---

\end{document}
