\documentclass{beamer}

% Shared Beamer configuration for Principles of Robot Autonomy decks
\usetheme{Boadilla}
%\usecolortheme{seagull}

% Use Palatino to match the book's Tufte style
\usepackage{mathpazo}
\usefonttheme{serif}

\usepackage{amsmath}
\usepackage{amssymb}
\usepackage{bm}
\usepackage{graphicx}
\usepackage{enumitem}
\usepackage{xparse}
\usepackage{array}

% Increase vertical spacing between list items for readability.
\setlist[itemize]{itemsep=0.75em}
\setlist[enumerate]{itemsep=0.75em}

% Define bold symbol macros if not already defined
\providecommand{\x}{\boldsymbol{x}}
\providecommand{\z}{\boldsymbol{z}}
\providecommand{\q}{\boldsymbol{q}}
\providecommand{\bxi}{\boldsymbol{\xi}}
\providecommand{\bu}{\boldsymbol{u}}
\providecommand{\bmu}{\boldsymbol{\mu}}
\providecommand{\bSigma}{\boldsymbol{\Sigma}}
\providecommand{\y}{\boldsymbol{y}}
\providecommand{\m}{\mathcal{M}}
\providecommand{\bc}{\boldsymbol{c}}
\providecommand{\R}{\mathbb{R}}

% Animated lists by default, with override via optional overlay argument.
\beamerdefaultoverlayspecification{<+->}

% Paragraphs reveal sequentially by default.
\AtBeginDocument{%
  \ifx\paragraph\undefined
    \newcommand{\paragraph}[1]{\textbf{#1}\par\onslide<+->{} }
  \else
    \let\origparagraph\paragraph
    \renewcommand{\paragraph}[1]{\origparagraph{#1}\par\onslide<+->{} }
  \fi
}

% Add other shared packages or theme settings above this line as needed.

\author[COMP 4766]{Adapted from ``Principles of Robot Autonomy'' by D. Gammelli, J. Lorenzetti,
K. Luo, G. Zardini, and M. Pavone}
\date{\today}


\title{Chapter 16 - Robot Localization}

\begin{document}

\begin{frame}[plain]
\titlepage
\end{frame}

\begin{frame}{Why Map-Based Localization?}
\begin{itemize}
\item Bayes filters from prior chapters lack the notion of a local map needed for range-based localization.
\item Localizing with range data requires knowing which object is sensed and where it sits in the environment frame.
\item We now target estimating robot pose relative to a given map so autonomy tasks stay grounded.
\end{itemize}
\end{frame}
% --- Original text start ---
% The last few chapters introduced some of the most widely used algorithms based on Bayes' filter for probabilistic robot localization and state estimation. However these fundamental algorithms still need further enhancements before application to many robot localization tasks, since in their standard form they don't incorporate a notion of a local \textit{map}. For example, a particle filter could be applied in its original form to a problem of global localization based on GNSS measurements, but localizing based on range measurements requires knowledge about \textit{what} object is being ranged, and \textit{where} that object is with respect to the local environment (i.e. the map).
% In this chapter a more specific definition of mobile robot localization is considered\cite{ThrunBurgardEtAl2005}, namely the problem of determining the pose of a robot relative to a \textit{given map} of the environment.
% --- Original text end ---

\begin{frame}{Coordinate Frames \& Pose}
\begin{itemize}
\item Localization aligns the global map frame with the robot's local coordinate frame via a pose transformation.
\item In planar problems we estimate $\x_t = [x, y, \theta]^\top$ even though no ideal sensor measures it directly.
\item Indirect noisy measurements $\z_t$ plus motion data provide the evidence for that coordinate transform.
\end{itemize}
\end{frame}
% --- Original text start ---
% \notessection{Robot Localization}
% Localization with respect to a map can be interpreted as a problem of coordinate transformation. Maps are described in a global coordinate system, which is independent of a robot’s pose. Localization can then be viewed as the process of establishing a correspondence between the map coordinate system and the robot’s local coordinate system. Knowing this coordinate transformation then enables the robot to express the location of objects of interest within its own coordinate frame (a necessary prerequisite for robot autonomy).
% 
% 
% In 2D problems, knowing the pose $\x_t = [x, y, \theta]^\top $ of a robot is sufficient to establish this correspondence, and an ideal sensor would directly be able to measure this pose. However in practice no such sensor exists, and therefore \textit{indirect} (often noisy) measurements $\z_t$ of the pose are used. Since it is almost impossible to be able to reliably estimate $\x_t$ from a single measurement $\z_t$, localization algorithms typically integrate additional data \textit{over time} to build reliable localization estimates. For example, consider a robot located inside a building where many corridors look alike. In this case a single sensor measurement (e.g. a range scan) is usually insufficient to disambiguate the identity of the corridor from the others.
% --- Original text end ---

\begin{frame}{Accumulating Evidence}
\begin{itemize}
\item Reliable pose estimates emerge by integrating data over time to overcome per-measurement ambiguity.
\item Ambiguous corridors illustrate why single scans rarely pinpoint position.
\item Casting map-based localization inside the Bayesian filtering framework lets us reuse prior algorithms.
\end{itemize}
\end{frame}
% --- Original text start ---
% In this case a single sensor measurement (e.g. a range scan) is usually insufficient to disambiguate the identity of the corridor from the others.
% 
% In this chapter it will be seen how this map-based localization problem can be cast in the Bayesian filtering framework, such that the algorithms from previous chapters can be leveraged.
% --- Original text end ---

\begin{frame}{Localization Taxonomy}
\begin{itemize}
\item We categorize localization tasks by initial pose knowledge, environment changes, agency, and team size.
\item Each axis influences algorithm choices, sensing assumptions, and robustness requirements.
\item Understanding the taxonomy clarifies when Gaussian vs. multi-hypothesis beliefs are appropriate.
\end{itemize}
\end{frame}
% --- Original text start ---
% \subsection{A Taxonomy of Localization Problems}
% To understand the broad scope of challenges related to robot localization, it is useful to develop a brief taxonomy of localization problems. This categorization will divide localization problems along a number of important dimensions pertaining to the nature of the environment (e.g. static versus dynamic), the initial knowledge that a robot may possess, and how information about the environment is gathered (e.g. passive or active, with one robot or collaboratively with several robots).
% --- Original text end ---

\begin{frame}{Local vs. Global vs. Kidnapped}
\begin{itemize}
\item Position tracking starts with known pose, enabling unimodal Gaussian filters with small error.
\item Global localization begins ignorant of pose; particle or other multi-hypothesis filters handle multimodality.
\item Kidnapped robots must detect teleport events, demanding robust non-parametric beliefs that can recover.
\end{itemize}
\end{frame}
% --- Original text start ---
% \subsubsection{Local vs. Global}
% Localization problems can be characterized by the type of knowledge that is available initially, which has a significant impact on what type of localization algorithm is most appropriate for the problem.
% 
% \begin{itemize}
%     \item \textit{Position tracking} problems assume that the initial pose of the robot is known. In these types of problems only incremental updates are required (i.e. the localization error is generally always small), and therefore unimodal Gaussian filters (e.g. Kalman filters) can be efficiently applied.
%     
%     \item \textit{Global localization} problems assume that the initial pose of the robot is unknown. In these scenarios the use of a unimodal parametric belief distribution cannot adequately capture the global uncertainty. Therefore it is more appropriate to use non-parametric, multi-hypothesis filters, such as the particle filter.
%     
%     \item The \textit{kidnapped robot problem} is a variant of the global localization problem (i.e. unknown initial pose) where the robot can get ``kidnapped'' and ``teleported'' to some other location. This problem is more difficult than the global localization problem since the localization algorithm needs to have an awareness that sudden drastic to the robot's pose are possible. While robots are typically not ``kidnapped'' in practice, the consideration of this type of problem is useful for ensuring the localization algorithm is \textit{robust}, since the ability to recover from failure is essential for truly autonomous robots. Similar to the global localization problem, these problems are often best addressed using non-parametric, multi-hypothesis filters.
% \end{itemize}
% --- Original text end ---

\begin{frame}{Static vs. Dynamic Environments}
\begin{itemize}
\item Static worlds move only the robot, simplifying prediction and measurement models.
\item Dynamic settings include moving objects; we augment state or filter sensor data to stay consistent.
\item Accounting for scene dynamics is crucial when landmarks relocate or disappear.
\end{itemize}
\end{frame}
% --- Original text start ---
% \subsubsection{Static vs. Dynamic}
% Environmental changes are another important consideration in mobile robot localization, specifically whether they are static or dynamic.
% \begin{itemize}
%     \item In \textit{static} environments the robot is the only object that moves. Static environments are generally much easier to perform localization in.
%     \item \textit{Dynamic} environments possess objects other than the robot whose locations or configurations change over time. This problem is usually addressed by augmenting the state vector to include the movement of dynamic entities, or by filtering the sensor data to remove the effects of environment dynamics.
% \end{itemize}
% --- Original text end ---

\begin{frame}{Passive vs. Active Gathering}
\begin{itemize}
\item Passive localization assumes motion is independent of localization objectives.
\item Active localization deliberately selects actions that maximize information (e.g., reorient to see more).
\item Hybrids switch between passive cruising and active probing depending on uncertainty.
\end{itemize}
\end{frame}
% --- Original text start ---
% \subsubsection{Passive vs. Active}
% Information collected via measurements is crucial for robot localization. Therefore it is reasonable to consider localization problems where the robot can \textit{explicitly} choose its actions to gather more (or more specific) information from the environment. 
% \begin{itemize}
%     \item \textit{Passive localization} problems assume that the robot's motion is unrelated to its localization process.
%     \item \textit{Active localization} problems consider the ability of the robot to choose its actions (at least partially) to improve its understanding of the environment. For example, a robot in the corner of a room might choose to reorient itself to face the rest of the room, so it can collect environmental information as it moves along the wall. Hybrid approaches are also possible, since it may be inefficient to use active localization all of the time. 
% \end{itemize}
% --- Original text end ---

\begin{frame}{Single vs. Multi-Robot}
\begin{itemize}
\item Single-robot localization gathers and reasons over data on one platform.
\item Multi-robot scenarios share beliefs so one robot's certainty can inform another via known relative pose.
\item Team localization introduces communication and consistency challenges alongside accuracy gains.
\end{itemize}
\end{frame}
% --- Original text start ---
% \subsubsection{Single Robot vs. Multi-Robot}
% It is of course also possible to consider problems where several robots all gather independent information and then share that information with each other.
% \begin{itemize}
%     \item \textit{Single-robot localization} problems are the most commonly studied and utilized approach, and are often simpler because all data is collected on a single platform.
%     \item \textit{Multi-robot localization} problems consider teams of robots that share information in such a way that one robot's belief can be used to influence another robot's belief if the relative location between robots is known. 
% \end{itemize}
% --- Original text end ---

\begin{frame}{Map Representation Basics}
\begin{itemize}
\item States, controls, and measurements follow prior notation, but we add a map $\m$ capturing environment objects.
\item Maps enumerate objects $m_i$ with properties; design choice affects efficiency and expressiveness.
\item We distinguish location-based (cells) and feature-based (landmark) encodings.
\end{itemize}
\pause
\begin{equation*}
\m = \{ m_1 , m_2 , \ldots , m_N \}
\end{equation*}
\end{frame}
% --- Original text start ---
% \subsection{Robot Localization via Bayesian Filtering}
% The parametric (e.g. EKF) and non-parametric (e.g. particle) filters from the previous chapters are all variations of the Bayes filter. In particular they rely on a Markov process assumption and the identification of probabilistic measurement models. In this section it is shown how map-based robot localization can be cast into this framework, such that the previously discussed algorithms can be applied.
% 
% Similar to the general filtering context from the previous chapters, at time $t$ the state is denoted by $\x_t$, the control input is denoted by $\bu_t$, and the measurements are denoted by $\z_t$. For example, a differential drive robot equipped with a laser range-finder (returning a set of range measurements $r_i$ and bearings $\phi_i$), the state, control, and measurements would be:
% \begin{equation}
% \x_{t} = \begin{bmatrix} x \\ y \\ \theta
% \end{bmatrix}, \quad
% \bu_{t} = \begin{bmatrix} v \\ \omega 
% \end{bmatrix}, \quad
% \z_{t} =  \begin{bmatrix} r_1 \\ \phi_1 \\  \vdots
% \end{bmatrix}.
% \end{equation}
% 
% However, the critical new component is the concept of a \textit{map} (denoted as $\m$), which is a list of objects in the environment along with their properties:
% \begin{equation}
% \m = \{ m_1 , m_2 , \ldots , m_N \},
% \end{equation}
% where $m_i$ represents the properties of a specific object. Generally there are two types of maps that will be considered, location-based maps and feature-based maps, which typically have differences in both computational efficiency and expressiveness.
% --- Original text end ---

\begin{frame}{Location-Based Maps}
\begin{itemize}
\item Associate each index with a volumetric cell in a decomposition or occupancy grid.
\item Resolution depends on cell size: finer grids capture detail but raise cost.
\item Explicitly encodes presence or absence of obstacles per location.
\end{itemize}
\pause
\begin{center}
    \includegraphics[width=0.75\linewidth]{../book/figs/ch16_figs/location_based_maps.png}
\end{center}
\end{frame}
% --- Original text start ---
% For location-based maps, the index $i$ associated with object $m_i$ corresponds to a specific \textit{location} (i.e. $m_i$ are volumetric objects). For example, objects $m_i$ in a location-based map might represent cells in a cell decomposition or grid representation of a map (see Figure \ref{fig:LocationBasedMaps}).
% \begin{figure}[ht]
% \centering
% \includegraphics[width=.75\linewidth]{tex/figs/ch16_figs/location_based_maps.png}
% \caption{Two examples of location-based maps, both represent the map as a set of volumetric objects (i.e. cells in these cases).}
% \label{fig:LocationBasedMaps}
% \end{figure}
% One potential disadvantage of the cell-based maps is that their resolution is dependent on the size of the cells, but their advantage is that they can explicitly encode information about presence (or absence) of objects in specific locations.
% --- Original text end ---

\begin{frame}{Feature-Based Maps}
\begin{itemize}
\item Each index identifies a feature (e.g., line segment, landmark) with associated properties.
\item Efficient in structured environments but may omit unmodeled obstacles.
\item Topological or geometric feature sets enable compact yet expressive localization.
\end{itemize}
\pause
\begin{center}
    \includegraphics[width=0.75\linewidth]{../book/figs/ch16_figs/feature_based_maps.png}
\end{center}
\end{frame}
% --- Original text start ---
% For feature-based maps, an index $i$ is a feature index, and $m_i$ contains information about the properties of that feature, including its Cartesian location. These types of maps can typically be thought of as a collection of landmarks. Figure \ref{fig:FeatureBasedMaps} gives two examples of feature-based maps, one which is represented by a set of lines, and another which is represented by nodes and edges like a graph (i.e. a topological map).
% \begin{figure*}[ht]
% \centering
% \includegraphics[width=0.8\linewidth]{tex/figs/ch16_figs/feature_based_maps.png}
% \caption{Two examples of feature-based maps.}
% \label{fig:FeatureBasedMaps}
% \end{figure*}
% Feature-based maps can be more finely tuned to specific environments, for example the line-based map might make sense to use in highly structured environments such as buildings. While feature-based maps can be computationally efficient, their main disadvantage is that they typically do not capture spatial information about all potential obstacles.
% --- Original text end ---

\begin{frame}{Map-Aware Motion Model}
\begin{itemize}
\item Base transition $p(\x_t\mid \bu_t, \x_{t-1})$ ignores map obstacles, so we modulate it with map consistency.
\item Approximate the map-conditioned motion by blending original dynamics with $p(\x_t\mid \m)$.
\item Normalization $\eta$ ensures the resulting density integrates to one.
\end{itemize}
\pause
\begin{equation*}
p(\x_t \mid \bu_t, \x_{t-1}, \m) \approx \eta \frac{p(\x_t\mid \bu_t, \x_{t-1}) p(\x_t \mid \m)}{p(\x_t)}
\end{equation*}
\end{frame}
% --- Original text start ---
% \subsubsection{State Transition Model}
% In the previous chapters on Bayesian filtering the probabilistic state transition model $p(\x_t \mid \bu_t, \x_{t-1})$ describes the posterior distribution over the states that the robot could transition to when executing control $\bu_t$ from $\x_{t-1}$. However in robot localization problems it might be important to take into account how the map $\m$ could affect the state transition since in general:
% \begin{equation*}
% p(\x_t \mid \bu_t, \x_{t-1}) \neq p(\x_t \mid \bu_t, \x_{t-1}, \m).
% \end{equation*}
% For example, $p(\x_t \mid \bu_t, \x_{t-1})$ cannot account for the fact that a robot cannot move through walls since it doesn't know that walls exist!
% 
% However, a common approximation is to make the assumption that:
% \begin{equation} \label{eq:statetransmap}
% p(\x_t \mid \bu_t, \x_{t-1}, \m) \approx \eta \frac{p(\x_{t}\mid \bu_{t}, \x_{t-1} ) p(\x_{t} \mid \m )}{p(\x_{t})},
% \end{equation}
% where $\eta$ is a normalization constant. This approximation can be derived from Bayes' rule by assuming that $p(\m \mid \x_t, \x_{t-1}, \bu_t) \approx p(\m \mid \x_t)$ (which is a tight approximation under high update rates). More specifically:
% \begin{equation*}
% \begin{split}
% p(\x_t \mid \bu_t, \x_{t-1}, \m) &= \frac{p(\m | \x_{t}, \x_{t-1}, \bu_{t}) p(\x_{t} \mid \x_{t-1}, \bu_t)}{p(\m \mid \x_{t-1}, \bu_{t})}, \\
% &= \eta' p(\m | \x_{t}, \x_{t-1}, \bu_{t}) p(\x_{t} \mid \x_{t-1}, \bu_t), \\
% &\approx \eta' p(\m | \x_{t}) p(\x_{t} \mid \x_{t-1}, \bu_t), \\
% & =\eta \frac{p(\x_{t}\mid \bu_{t}, \x_{t-1} ) p(\x_{t} \mid \m )}{p(\x_{t})},
% \end{split}
% \end{equation*}
% where $\eta'$ and $\eta$ are normalization constants (such that the total probability density integrates to one).
% 
% In this approximation the term $p(\x_{t} \mid \m )$ is the state probability conditioned on the map which can be thought of as describing the ``consistency'' of state with respect to the map. The approximation \eqref{eq:statetransmap} can therefore be viewed as making a probabilistic guess using the original state transition model (without map knowledge), and then using the consistency term $p(\x_{t} \mid \m )$ to check the plausibility of the new state $\x_t$ given the map.
% --- Original text end ---

\begin{frame}{Map-Aware Measurement Model}
\begin{itemize}
\item Measurements now depend on both pose and map: $p(\z_t\mid \x_t, \m)$.
\item With $K$ sensor returns per step we model conditional independence once we condition on $\x_t$ and $\m$.
\item This factorization makes multi-measurement updates tractable in EKF and particle implementations.
\end{itemize}
\pause
\begin{equation*}
p(\z_t\mid \x_t, \m) = \prod_{k=1}^K p(\z_t^k \mid \x_t, \m)
\end{equation*}
\end{frame}
% --- Original text start ---
% \subsubsection{Measurement Model}
% The probabilistic measurement model model $p(\z_t \mid \x_t)$ from previous chapters also needs to be modified to take map information into account. This new measurement model can simply be expressed as $p(\z_t \mid \x_t, \m)$ (i.e. measurement is also conditioned on the map). This is obviously important because the local measurements can have significant influence from the environment. For example a range measurement is dependent on what object is currently in the line of sight.
% 
% Additionally, since the suite of sensors on a robot may generate more than one measurement when queried, it is also common to make another measurement model assumption for simplicity. Suppose $K$ measurements are taken at time $t$, such that:
% \begin{equation*}
%     \z_t = \begin{bmatrix}
%     \z_t^1 \\ \vdots \\ \z_t^K
%     \end{bmatrix}.
% \end{equation*}
% Then it can often be assumed that each of the $K$ measurements are conditionally independent from each other (i.e. when conditioned on $\x_t$ and $\m$ the probability of measuring $\z_t^k$ is independent from the other measurements). With this assumption the probabilistic measurement model can be expressed as:
% \begin{equation} \label{eq:condind}
%     p(\z_{t} \mid \x_{t}, \m) = \prod_{k=1}^{K}p(\z_{t}^{k}\mid \x_{t}, \m).
% \end{equation}
% --- Original text end ---

\begin{frame}{Markov Localization Algorithm}
\begin{itemize}
\item Bayes filter steps remain: predict with map-aware motion, then correct with map-aware measurements.
\item Includes controls $\bu_t$, observations $\z_t$, prior $bel(\x_{t-1})$, and map $\m$.
\item Works for global, tracking, and kidnapped variants depending on belief representation.
\end{itemize}
\end{frame}
% --- Original text start ---
% \subsection{Markov Localization}
% With the probabilistic state transition and measurement models that include the map, the Bayes' filter can be directly modified as shown in Algorithm \ref{alg:markovloc}.
% \begin{algorithm}[ht]
%  \KwData{$bel(\x_{t-1}), \bu_{t},\z_{t}, \m$}
%  \KwResult{$bel(\x_{t})$}
%  \ForEach{$\x_t$}{
%     $\overline{bel}(\x_t) = \int p(\x_t\mid \bu_{t}, \x_{t-1},\m) bel(\x_{t-1}) d\x_{t-1}$ \\
%     $bel(\x_t) = \eta p(\z_t\mid \x_{t},\m)\overline{bel}(\x_t)$
%  }
%  \Return $bel(\x_{t})$
%  \caption{Markov Localization Algorithm}
%  \label{alg:markovloc}
% \end{algorithm}
% As can be seen, this algorithm is conceptually identical to the Bayes' filter except for the inclusion of the model $\m$. This algorithm is referred to as the \textit{Markov localization} algorithm, and the localization problem it is trying to solve is generally referred to as simply \textit{Markov localization}\footnote{Recall the use of the Markov property assumption in the derivation of the Bayes' filter.}. 
% 
% The Markov localization algorithm can be used to address global localization, position tracking, and kidnapped robot problems, but generally some implementation details might be different. The choice for the initial (prior) belief distribution $bel(\x_0)$ is one such parameter that may be different depending on the type of localization problem.
% --- Original text end ---

\begin{frame}{Selecting Priors}
\begin{itemize}
\item Position tracking initializes with a concentrated Gaussian $\mathcal{N}(\bar{\x}_0, \bSigma_0)$.
\item Global localization often starts uniform across feasible states $1/|X|$.
\item Prior design sets recovery ability for kidnapped scenarios.
\end{itemize}
\end{frame}
% --- Original text start ---
% Specifically, since the initial belief encodes any prior knowledge about the robot pose, the best choice of distribution depends on what (if any) knowledge is available. For example, in the position tracking problem it is assumed that an initial pose of the robot is known. Therefore choosing a (unimodal) Gaussian distribution $bel(\x_0) \sim \mathcal{N}(\bar{\x}_0, \bSigma_0)$ with a small covariance might be a good choice. Alternatively, for a global localization problem the initial pose is not known. In this case an appropriate choice for the initial belief would be a uniform distribution $bel(\x_0) = 1/\lvert X \rvert$ over all possible states $\x$.
% --- Original text end ---

\begin{frame}{EKF Localization Setup}
\begin{itemize}
\item Assume Gaussian belief $bel(\x_t) \sim \mathcal{N}(\bmu_t, \bSigma_t)$ with nonlinear motion $g(\bu_t, \x_{t-1})$.
\item Map provides landmark set $\m = \{m_j\}$ with positions $(m_{j,x}, m_{j,y})$.
\item Measurements $\z_t^i$ observe specific landmarks via model $h(\x_t, j, \m)$ plus Gaussian noise.
\end{itemize}
\end{frame}
% --- Original text start ---
% \subsection{Extended Kalman Filter (EKF) Localization}
% The extended Kalman filter (EKF) localization algorithm is essentially equivalent to the EKF algorithm presented in previous chapters, except that it also takes the map $\m$ into account. In particular, it still makes a Guassian belief assumption, $bel(\x_t) \sim \mathcal{N}(\bmu_t, \bSigma_t)$, to add structure to the filtering problem. As a brief review, the assumed state transition model is given by:
% \begin{equation*}
% \x_t = g(\bu_t, \x_{t-1}) + \bm{\epsilon}_t,
% \end{equation*}
% where $\bm{\epsilon}_t \sim \mathcal{N}(\bm{0}, \bm{R}_t)$ is Gaussian zero-mean noise. The Jacobian $G_t$ is again defined by $G_t = \nabla_{\x}g(\bu_t, \bmu_{t-1})$, where $\bmu_{t-1}$ is the expected value of the previous belief distribution $bel(\x_{t-1})$. 
% 
% The main difference in EKF localization is the assumption that a feature-based map is available, consisting of point landmarks given by:
% \begin{equation*}
% \m = \{m_1, m_2, \dots, m_N\}, \quad m_j = (m_{j,x}, m_{j,y}),
% \end{equation*}
% where $N$ is the total number of landmarks, and each landmark $m_j$ encapsulates the location $(m_{j,x}, m_{j,y})$ of the landmark in the global coordinate frame. Measurements $\z_t$ associated with these point landmarks at a time $t$ are denoted by:
% \begin{equation*}
% \z_t = \{\z_t^1, \z_t^2, \dots \},
% \end{equation*}
% where $\z_t^i$ is associated with a particular landmark and is assumed to be generated by the measurement model:
% \begin{equation*}
% \z^i_t = h(\x_t, j, \m) + \bm{\delta}_t,
% \end{equation*}
% where $\bm{\delta}_t \sim \mathcal{N}(\bm{0}, \bm{Q}_t)$ is Gaussian zero-mean noise and $j$ is the index of the map feature $m_j \in \m$ that measurement $i$ is associated with.
% --- Original text end ---

\begin{frame}{Data Association Variables}
\begin{itemize}
\item Unknown correspondences motivate discrete variables $c_t^i$ selecting landmark indices or "no match".
\item Given $c_t^i$, compute measurement Jacobian $H_t^{c_t^i}$ for EKF updates.
\item Solving the data association problem is central to EKF localization robustness.
\end{itemize}
\end{frame}
% --- Original text start ---
% One fundamental problem that now needs to be addressed is the \textit{data association problem}, which arises due to uncertainty in which measurements are associated with which landmark. To begin addressing this problem, the correspondences are modeled through a variable $c_t^i \in \{1,\dots,N+1\}$, which take on the values $c_t^i = j$ if measurement $i$ corresponds to landmark $j$, and $c_t^i = N + 1$ if measurement $i$ has no corresponding landmark. 
% Then, given a correspondence $c^i_t$ of measurement $i$ (associated with a specific landmark), the Jacobian $H^i_t$ used in the EKF measurement correction step can be determined. Specifically, for the $i$-th measurement the Jacobian of the new measurement model can be computed by $H^{c^i_t}_t = \nabla_{\x}h(\bar{\bmu}_t,c^i_t,\m)$, where $\bar{\bmu}_t$ is the predicted mean (that results from the EKF prediction step).
% --- Original text end ---

\begin{frame}{EKF Localization: Known Correspondences}
\begin{itemize}
\item Prediction step mirrors standard EKF with $\bar{\bmu}_t$ and $\bar{\bSigma}_t$.
\item Loop over measurements, using their known landmark indices $c_t^i$.
\item Update innovations with $S_t^i$, Kalman gain $K_t^i$, and covariance deflation exactly as in the textbook EKF localization loop.
\end{itemize}
\end{frame}
% --- Original text start ---
% \subsubsection{EKF Localization with Known Correspondences}
% In practice the correspondences between measurements $\z^i_t$ and landmarks $m_j$ are generally unknown. However, it is useful from a pedagogical standpoint to first consider the case where these correspondences $\bc_t = [c_t^1,\dots]^\top $ are assumed to be \textit{known}.
% 
% In the EKF localization algorithm given in Algorithm \ref{alg:ekflocal}, the main difference from the original EKF filter algorithm is that multiple measurements are processed at the same time. Crucially, this is accomplished in a computationally efficient way by exploiting the conditional independence assumption \eqref{eq:condind} for the measurements. In fact, by exploiting this assumption and some special properties of Gaussians, the multi-measurement update can be implemented by just looping over each measurement individually and applying the standard EKF correction.
% \begin{algorithm}[ht]
%  \KwData{$\bmu_{t-1}, \bSigma_{t-1}, \bu_{t},\z_{t}, \bc_t, \m$}
%  \KwResult{$\bmu_t, \bSigma_t$}
%  $\bar{\bmu}_t = g(\bu_t,\bmu_{t-1})$\\
%  $\bar{\bSigma}_t = G_t\bSigma_{t-1} G_t^{T} + \bm{R}_t$\\
%  \ForEach{$\z_t^i$}{
%   $j = c_t^i$\\
%   $S_t^i = H_t^j\bar{\bSigma}_{t}[H^j_t]^{T}+\bm{Q}_t$\\
%   $K^i_t = \bar{\bSigma}_{t}[H^j_t]^{T}[S_t^i]^{-1}$\\
%   $\bar{\bmu}_t = \bar{\bmu}_t + K^i_t(\z^i_t - h(\bar{\bmu}_{t}, j, \m))$\\
%   $\bar{\bSigma}_t = (I - K^i_t H^j_t)\bar{\bSigma}_t$\\
%  }
%  $\bmu_t = \bar{\bmu}_t$\\
%  $\bSigma_t = \bar{\bSigma}_t$\\
%  \Return $\bmu_t, \bSigma_t$
%  \caption{Extended Kalman Filter Localization Algorithm}
%  \label{alg:ekflocal}
% \end{algorithm}
% --- Original text end ---

\begin{frame}{Unknown Correspondences via ML}
\begin{itemize}
\item Maximize data likelihood over correspondence set $\bc_t$ using conditional independence.
\item Each measurement selects a landmark minimizing negative log-likelihood (Mahalanobis distance).
\item Requires evaluating predicted measurements $\hat{\z}_t^k$ and covariances $S_t^k$ for every candidate.
\end{itemize}
\pause
\begin{equation*}
\hat{c}_t^i = \arg \min_k (\z_t^i-\hat{\z}_t^k)^\top [S_t^k]^{-1} (\z_t^i-\hat{\z}_t^k)
\end{equation*}
\end{frame}
% --- Original text start ---
% \subsubsection{EKF Localization with Unknown Correspondences}
% For EKF localization with \textit{unknown} correspondences, the correspondence variables must also be estimated!
% ... (derivation through Mahalanobis distance) ...
% \begin{equation}
% d_t^{ij} = (\z_t^i-\hat{\z}^{j}_t)^\top  [S_t^{j}]^{-1} (\z_t^i-\hat{\z}^{j}_t),
% \end{equation}
% ...
% --- Original text end ---

\begin{frame}{Validation Gates}
\begin{itemize}
\item To avoid brittle ML assignments, reject matches with Mahalanobis distance above threshold $\gamma$.
\item Unmatched measurements can be treated as outliers or new-feature hypotheses.
\item Gates improve robustness when multiple landmarks look similar.
\end{itemize}
\end{frame}
% --- Original text start ---
% One of the disadvantages of using the maximum likelihood estimation is that it can be brittle with respect to outliers and in cases where there are equally likely hypothesis for the correspondence.
% An alternative approach to estimating correspondences that is more robust to outliers is to use a \textit{validation gate}. In this approach the Mahalanobis smallest distance $d_t^{ij}$ must also pass a \textit{thresholding} test:
% \begin{equation*}
% (\z_t^i-\hat{\z}^{j}_t)^\top  [S_t^{j}]^{-1} (\z_t^i-\hat{\z}^{j}_t) \leq \gamma,
% \end{equation*}
% in order for a correspondence to be created.
% --- Original text end ---

\begin{frame}{Range/Bearing Measurement Model}
\begin{itemize}
\item Differential-drive robot senses landmark range and bearing relative to body frame.
\item Nonlinear measurement function $h(\x_t, j, \m)$ outputs $[r, \phi]^\top$.
\item Jacobian $H_t^j$ and diagonal noise $\bm{Q}_t = \text{diag}(\sigma_r^2, \sigma_\phi^2)$ support EKF updates.
\end{itemize}
\pause
\begin{equation*}
h(\x_t, j, \m) = \begin{bmatrix}
\sqrt{(m_{j,x} - x)^2 + (m_{j,y}- y)^2} \\
\text{atan2}(m_{j,y}- y, m_{j,x} - x) - \theta
\end{bmatrix}
\end{equation*}
\end{frame}
% --- Original text start ---
% \begin{example}[Differential Drive Robot with Range and Bearing Measurements] \label{ex:rangeandbearing}
% \theoremstyle{definition}
% Consider a differential drive robot with state $\x = [x, y, \theta]^\top $, and suppose a sensor is available on the robot which measures the range $r$ and bearing $\phi$ of landmarks $m_j \in \m$ relative to the robot’s local coordinate frame. Additionally, multiple measurements corresponding to different features can be collected at each time step:
% \begin{equation*}
% \z_t = \{[r_t^1,\phi_t^1]^\top , [r_t^2,\phi_t^2]^\top , \dots\},
% \end{equation*}
% where each measurement $\z_t^i$ contains the range $r_t^i$ and bearing $\phi_t^i$. 
% 
% Assuming the correspondences are known, the measurement model for the range and bearing is:
% \begin{equation}
% h(\x_t, j, \m)  = \begin{bmatrix}
% \sqrt{(m_{j,x} - x)^{2} + (m_{j,y}- y)^{2}} \\
% \text{atan2}(m_{j,y}- y, m_{j,x} - x) - \theta
% \end{bmatrix}.
% \end{equation}
% The measurement Jacobian $H^j_t$ corresponding to a measurement from landmark $j$ is then given by:
% \begin{equation}
% H^j_t = \begin{bmatrix}
% -\frac{m_{j,x} - \bar{\mu}_{t,x}}{\sqrt{(m_{j,x} - \bar{\mu}_{t,x})^2 + (m_{j,y} - \bar{\mu}_{t,y})^2}} & -\frac{m_{j,y} - \bar{\mu}_{t,y}}{\sqrt{(m_{j,x} - \bar{\mu}_{t,x})^2 + (m_{j,y} - \bar{\mu}_{t,y})^2}} & 0 \\
% \frac{m_{j,y} - \bar{\mu}_{t,y}}{(m_{j,x} - \bar{\mu}_{t,x})^2 + (m_{j,y} - \bar{\mu}_{t,y})^2} & -\frac{m_{j,x} - \bar{\mu}_{t,x}}{(m_{j,x} - \bar{\mu}_{t,x})^2 + (m_{j,y} - \bar{\mu}_{t,y})^2} & -1
% \end{bmatrix}.
% \end{equation}
% It is also common to assume that the covariance of the measurement noise is given by:
% \begin{equation*}
% \bm{Q}_t = \begin{bmatrix}
% \sigma_r^2 & 0 \\ 0 & \sigma_\phi^2
% \end{bmatrix},
% \end{equation*}
% where $\sigma_r$ is the standard deviation of the range measurement noise and $\sigma_\phi$ is the standard deviation of the bearing measurement noise. This diagonal covariance matrix is typically used since these two measurements can be assumed to be uncorrelated.
% \end{example}
% --- Original text end ---

\begin{frame}{Monte Carlo Localization}
\begin{itemize}
\item Represent belief by particles $\mathcal{X}_t = \{\x_t^{[1]}, ..., \x_t^{[M]}\}$ sampled from motion model.
\item Weight particles with map-consistent likelihoods $p(\z_t\mid \bar{\x}_t^{[m]}, \m)$.
\item Resample to focus on high-likelihood hypotheses; add random particles to handle kidnapping if needed.
\end{itemize}
\end{frame}
% --- Original text start ---
% \subsection{Monte Carlo Localization (MCL)}
% Another approach to Markov localization is the Monte Carlo localization (MCL) algorithm. This algorithm leverages the non-parametric particle filter algorithm from the previous chapter, and is therefore much better suited to solving \textit{global} localization problems (unlike EKF localization which only solves position tracking problems). MCL can also be used to solve the kidnapped robot problem through some small modifications, such as injecting new random particles at each step to ensure that a ``particle collapse'' problem does not occur.
% 
% As a brief review, the particle filter represents the belief $bel(\x_t)$ by a set of $M$ particles:
% \begin{equation*}
% \mathcal{X}_t \coloneqq \{\x_t^{[1]}, \x_t^{[2]},..., \x_t^{[M]}\},
% \end{equation*}
% where each particle $\x_t^{[m]}$ represents a hypothesis about the true state $\x_t$. At each step of the algorithm the state transition model is used to propagate forward the particles, and then the measurement model is used to resample particles based on the measurement likelihood. This algorithm is shown in Algorithm \ref{alg:montecarlo}, and is nearly identical to the particle filter algorithm except that the map $\m$ is used in the probabilistic state transition and measurement models.
% \begin{algorithm}[ht]
%  \KwData{$\mathcal{X}_{t-1}, \bu_{t}, \z_{t}, \m$}
%  \KwResult{$\mathcal{X}_{t}$}
%  $\bar{\mathcal{X}}_{t} = \mathcal{X}_t = \emptyset$\\
%  \For{$m=1$ \KwTo $M$}{
%   Sample $\bar{\x}_{t}^{[m]} \sim p(\x_t\mid \bu_t, \x_{t-1}^{[m]},\m)$\\
%   $w_t^{[m]} = p(\z_t \mid \bar{\x}_{t}^{[m]},\m)$\\
%   $\bar{\mathcal{X}}_{t} = \bar{\mathcal{X}}_{t} \cup \big(\bar{\x}_{t}^{[m]}, w_t^{[m]} \big)$\\
%  }
%  \For{$m=1$ \KwTo $M$}{
%   Draw $i$ with probability $\propto w_t^{[i]}$\\
%   Add $\bar{\x}_{t}^{[i]}$ to $\mathcal{X}_t$
%  }
%  \Return $\mathcal{X}_t$
%  \caption{Monte Carlo Localization Algorithm}
%  \label{alg:montecarlo}
% \end{algorithm}
% --- Original text end ---

\begin{frame}{Reminder: Map-Based Localization Stack}
\begingroup
\setbeamercolor{block title}{bg=gray!20,fg=black}
\setbeamercolor{block body}{bg=gray!10,fg=black}
\begin{block}{Key Takeaways}
\begin{itemize}
\item Embed maps inside Bayes filters by modulating motion and measurement models.
\item EKF localization excels with known correspondences; MCL handles multimodal beliefs and kidnapping.
\item Careful prior choice, data association, and map selection keep localization reliable.
\end{itemize}
\end{block}
\endgroup
\end{frame}
% --- Original text start ---
% As can be seen, this algorithm is conceptually identical to the Bayes' filter except for the inclusion of the model $\m$. ... The Markov localization algorithm can be used to address global localization, position tracking, and kidnapped robot problems, but generally some implementation details might be different.
% The extended Kalman filter (EKF) localization algorithm is essentially equivalent to the EKF algorithm presented in previous chapters, except that it also takes the map $\m$ into account.
% Another approach to Markov localization is the Monte Carlo localization (MCL) algorithm. This algorithm leverages the non-parametric particle filter algorithm from the previous chapter, and is therefore much better suited to solving \textit{global} localization problems ... MCL can also be used to solve the kidnapped robot problem ...
% --- Original text end ---

\end{document}
